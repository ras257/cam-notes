\documentclass[../main.tex]{subfiles}
\begin{document}
\chapter{Fundamentals of Forces}
\section{Newton's Second Law}
Once there is more than one particle in in the universe, there will be \textit{interactions} between the particles.
In Newtonian physics such interactions are described by \textit{forces}.\quad
\begin{definition}[Momentum]
  The \textit{momentum} of a point particle is the product of its \textit{mass} and its \textit{velocity}:
  \[
    \vec{p} = m \dot{\vec{x}}
  \]
  where $m$ is the \textit{inertial mass}, often shortened to just \textit{mass}.
\end{definition}
The mass is an additional property of point particles.
It could change with time, but we assume that it is constant unless otherwise specified.
\begin{law}[Newton's Second Law]
  In an inertial frame:
  \[
    \dot{\vec{p}} = \vec{F}
  \]
  That is, the rate of change of momentum is equal to the force.
  \label{newton2}
\end{law}
The force $\vec{F}$ depends on the interaction, however, we require that it must only depend on $\vec{x}$ and $\dot{\vec{x}}$.
This means that Newton's 2nd law is a second order differential equation for $\vec{x}(t)$.
Thus, given $\vec{x}$ and $\dot{\vec{x}}$ of all particles in a system at some time, Newton's equations uniquely determine $\vec{x}(t)$ for all future and past times.
The force cannot depend exclusively on time because there is no preferred time in the universe as we saw in \cref{galileanRelativity}.
\begin{remark}
  Newtonian mechanics has been superseded by both quantum mechanics (generally, small things) and relativity (generally, fast things), but nevertheless accurately describes much of the universe.
\end{remark}
\section{Conservative Forces}
\begin{definition}[Conservative Force]
  A \textit{conservative force} can be written as:
  \[
    \vec{F} = - \nabla V
  \]
  for some \textit{potential} $V$, a scalar function.
\end{definition}
\begin{remark}[Derivative Notation]\par
  From Vector Calculus, the gradient $\nabla V$ is the vector:
  \[
    \nabla V = \left(\pderiv{V}{x_1}, \pderiv{V}{x_2}, \pderiv{V}{x_3}\right)
  \]
  The notation $\partial_i f$ is often used as a shorthand for $\pderiv{f}{x_i}$.
\end{remark}
\subsection{Gravitational Potential Energy}
The \textit{gravitational potential energy} of a particle of mass $m$ at $\vec{x}$ due to a particle of mass $M$ at $\vec{x}_0$ is:
\[
  V = - \frac{GMm}{|\vec{x} - \vec{x}_0|}
\]
where $G \approx \qty{6.67e-11}{\metre^3\kilogram^{-1}\second^{-2}}$ is the \textit{gravitational constant}.

We can find $\nabla|\vec{x} - \vec{x}_0|$ by computing $\partial_i|\vec{x} - \vec{x}_0|^2$ in two ways:
\begin{align*}
  \partial_i(|\vec{x} - \vec{x}_0|^2) &= 2|\vec{x} - \vec{x}_0|\partial_i|\vec{x} - \vec{x}_0| \\
  \partial_i((\vec{x} - \vec{x}_0)_j (\vec{x} - \vec{x}_0)_j) &= 2(\vec{x} - \vec{x}_0)_j \partial_i(x_j - {x_0}_j) = 2(\vec{x} - \vec{x}_0)_j \delta_{i j} = 2(\vec{x} - \vec{x}_0)_i
\end{align*}
Combining the above yields:
\[
  2|\vec{x} - \vec{x}_0|\partial_i |\vec{x} - \vec{x}_0| = 2(\vec{x} - \vec{x}_0)_i \implies  \partial_i|\vec{x} - \vec{x}_0| = \frac{(\vec{x} - \vec{x}_0)_i}{|\vec{x} - \vec{x}_0|}
\]
Therefore:
\[
  \nabla |\vec{x} - \vec{x}_0| = \frac{\vec{x} - \vec{x}_0}{|\vec{x} - \vec{x}_0|}
\]
and so:
\[
  F = - \nabla V = -GMm \frac{\nabla|\vec{x} - \vec{x}_0|}{|\vec{x} - \vec{x}_0|^2} = -GMm \frac{\vec{x} - \vec{x}_0}{|\vec{x} - \vec{x}_0|^3}
\]
If we let $\vec{r} = \vec{x} - \vec{x}_0$ be the vector that points from $\vec{x}_0$ to $\vec{x}$ then:
\[
  F = -\frac{GMm}{r^2} \uvec{r} \text{ where } \uvec{r} = \frac{\vec{r}}{|\vec{r}|}
\]
which is the \textit{inverse square law of gravity}.
The force is going in the opposite direction to $\uvec{r}$ so the force acting on the particle of mass $m$ at $\vec{x}$ is attractive towards the particle of mass $M$ at $\vec{x}_0$, as expected.

Sometimes we write $V = m \Phi$ where $\Phi = -\frac{GM}{|\vec{x} - \vec{x}_0|}$ is called the \textit{gravitational potential}.

Near the surface of the Earth, take $\vec{x}_0 = \vec{0}$ to be the centre of the Earth and $|\vec{x}| = R + z$, where $R$ is the radius of the Earth and $z \ll R$ is the distance above the surface.
\begin{align*}
  \Phi(R + z) &= - \frac{GM}{R + z} \\
              &\approx -\frac{GM}{R}\left[1 - \frac{z}{R} + \frac{z^2}{R^2} + \cdots\right] \\
              &\approx \text{constant} + \frac{GM}{R^2} z + \cdots
\end{align*}
We define $g = \frac{GM}{R^2} \approx \qty{9.81}{\metre\second^{-2}}$ on Earth.
So $\Phi(R + z) \approx \text{constant} + gz$ near the earth.
Using this approximation, when we take the gradient, the constant drops out so $\nabla \Phi = g\uvec{z}$.
Thus $\vec{F} = -m\nabla \Phi = -mg\uvec{z}$ which is a constant downwards force.

This force leads to the simplest nontrivial example of motion due to a force.
Newton's 2nd Law (\cref{newton2}) tells us that:
\[
  m \ddot{\vec{x}} = m \vec{g} \text{ where } \vec{g} = (0, 0, -g)
\]
Considering the $z$ component:
\begin{align*}
  \cancel{m} \ddot{z} &= - \cancel{m}g \\
  \implies \dot{z} &= v_0 - gt \\
  \implies z &= z_0 + v_0 t - \frac{1}{2}gt^2
\end{align*}
where $z_0$ is the initial height and $v_0$ is the initial velocity.
\begin{remark}[Assumptions]
  \begin{itemize}
    \item We treated the Earth as a point particle, assuming that all of its mass is concentrated in a single point.
    \item We ignored non-inertial effects as technically the Earth is accelerating so is not an inertial frame.
  \end{itemize}
\end{remark}
\subsection{Energy}
\begin{definition}[Energy]
  Conservative forces have a \textit{conserved energy}:
  \begin{align*}
    E &= \frac{1}{2}m \dot{\vec{x}} \cdot \dot{\vec{x}} + V(\vec{x}) \\
      &= \frac{1}{2}m \dot{x}_i \dot{x}_i + V(\vec{x})
  \end{align*}
\end{definition}
We can check that it is constant by computing $\deriv{E}{t}$:
\begin{align*}
  \deriv{E}{t} &= m \dot{x}_i \ddot{x}_i + \pderiv{V}{x_i}\dot{x}_i \\
  &= \dot{x}_i\left(m \ddot{x}_i + \pderiv{V}{x_i}\right) \\
  &= 0 \text{ as $\vec{F} = m \ddot{\vec{x}} = -\nabla V$}
\end{align*}
\begin{example}[Escape Velocity]
  Suppose we throw an object into space and want it to never fall back down.
  The minimal velocity the object must have to achieve this is called the \textit{escape velocity}.

  As the object is thrown:
  \[
    E_{\text{initial}} = \frac{1}{2}mv^2 - \frac{GMm}{R}
  \]
  as $V = - GMm/R$ on the surface of Earth.

  To not fall back, the object must reach $|\vec{x}| \to \infty$ without the velocity going to zero.
  Assuming that the particle reaches infinity, we have:
  \begin{align*}
    \frac{1}{2}mv^2 - \frac{GMm}{R} &= E_{\text{initial}} \\
                                    &= E_{\infty} \text{ by conservation of energy}\\
                                    &= \frac{1}{2}mv^{2}_{\infty} - 0 \text{ as $V(\vec{x}) \to 0$ as $|\vec{x}| \to \infty$}
  \end{align*}
  Since it made it to infinity, $v_{\infty}$ must be positive and thus $v^2 > 2GM/R$.
  The escape velocity is then $v_{\text{escape}} = \sqrt{2GM/R}$.
  On Earth, this is about \qty{10}{\kilo\metre\second^{-1}}.

  Note that the mass of the object $m$ cancels in $v_{\text{escape}}$ because the \textit{gravitational mass} (that appears in the inverse square law), is the same as the \textit{inertial mass} (that appears in Newton's 2nd Law -- \cref{newton2}).
  This hypothesis is known as the \textit{principal of equivalence}.
\end{example}
It is useful to write $E = T + V$ where $T = \frac{1}{2} m \dot{\vec{x}} \cdot \dot{\vec{x}}$ is the \textit{kinetic energy} and $V$ is the potential energy.

\begin{definition}[Work Done]
  The \textit{work done} by a force along a trajectory $C$ is defined to be the line integral:
  \[
    W = \int_C \vec{F} \cdot \d{\vec{x}}
  \]
\end{definition}
\begin{proposition}[Path independence of work done for conservative forces]
  Conservative forces have the property that the \textit{work done} by the force as a particle moves along a trajectory only \textbf{depends on the endpoints of the trajectory} and \textbf{not} on the path itself.
\end{proposition}
There are two ways to see this:
\begin{proof}[1]
  \[
    W = \int_C \vec{F} \cdot \d{\vec{x}} = \int_{t_1}^{t_2} \underbrace{\vec{F} \cdot \deriv{\vec{x}}{t}}_{\text{Power}} \d{t}
  \]
  Continued next lecture.
\end{proof}
\end{document}
