\documentclass[../main.tex]{subfiles}
\begin{document}
\chapter{Fundamentals of Forces}
\section{Newton's Second Law}
Once there is more than one particle in in the universe, there will be \textit{interactions} between the particles.
In Newtonian physics such interactions are described by \textit{forces}.\quad
\begin{definition}[Momentum]
  The \textit{momentum} of a point particle is the product of its \textit{mass} and its \textit{velocity}:
  \[
    \vec{P} = m \dot{\vec{x}}
  \]
  where $m$ is the \textit{inertial mass}, often shortened to just \textit{mass}.
\end{definition}
The mass is an additional property of point particles.
It could change with time, but we assume that it is constant unless otherwise specified.
\begin{law}[Newton's Second Law]
  In an inertial frame:
  \[
    \dot{\vec{p}} = \vec{F}
  \]
  That is, the rate of change of momentum is equal to the force.
\end{law}
The force $\vec{F}$ depends on the interaction, but can only depend on $\vec{x}$ and $\dot{\vec{x}}$.
This means that Newton's 2nd law is a second order differential equation for $\vec{x}(t)$.
Thus, given $\vec{x}$ and $\dot{\vec{x}}$ of all particles in a system at some time, Newton's equations uniquely determine $\vec{x}(t)$ for all future and past times.
The force cannot depend exclusively on time because there is no preferred time in the universe as we saw in \cref{galileanRelativity}.
\begin{remark}
  Newtonian mechanics has been superseded by both quantum mechanics (generally, small things) and relativity (generally, fast things), but nevertheless accurately describes much of the universe.
\end{remark}
\end{document}
