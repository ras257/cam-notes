\documentclass[../main.tex]{subfiles}
\begin{document}
\chapter{Systems of Particles}
\label{systemsOfParticles}
\section{General Systems of Particles}
Suppose we have $N$ particles indexed by $i = 1, \ldots, N$ with momenta $\vec{p}_i = m_i \dot{\vec{x}}_i$ and all obeying Newton's law individually, that is, for each particle:
\[
  \dot{\vec{p}}_i = \vec{F}_i
\]
The force $F_i$ on the $i$-th particle can be \textbf{external} or \textbf{due to the other particles}:
\[
  \vec{F}_i = \vec{F}^{\text{ext}}_i + \sum_{j \neq i} \vec{F}_{ij}
\]
where $\vec{F}^{\text{ext}}_i$ is the external force on the $i$-th particle and $\vec{F}_{ij}$ is the force on the $i$-th particle due to the $j$-th particle.

\begin{law}[Newton's 3rd Law]
  The forces between particles are found to obey:
  \label{newtonsThird}
  \[
    \vec{F}_{ij} = - \vec{F}_{ji}
  \]
  This is often stated as:
  \[
    \text{``Every action has an equal an opposite reaction''}
  \]
\end{law}
\begin{remark}
  An example of Newton's 3rd law can be seen in \cref{gravity}.
  If we swap the roles of the two particles around, all that changes is the direction with which the force acts in.
\end{remark}
\subsection{Centre of Mass}
\begin{definition}[Total Mass and Momentum]
  The \textit{total mass $M$} is:
  \[
    M = \sum_{i = 1}^{N} m_i
  \]
  The \textit{total momentum $\vec{P}$} is:
  \[
    \vec{P} = \sum_{i = 1}^{N} \vec{p}_i
  \]
\end{definition}
\begin{definition}[Centre of Mass]
  The \textit{centre of mass} of a system of particles is:
  \[
    \vec{R} = \frac{1}{M}\sum_{i = 1}^{N} m_i\vec{x}_i
  \]
  It is often abbreviated as \textit{CoM}.
\end{definition}
We can relate the total momentum to the centre of mass:
\begin{align*}
  \vec{P} &= \sum_{i = 1}^{N} \vec{p}_i \\
          &= \sum_{i = 1}^{N} m_i \vec{x}_i \\
          &= M \dot{\vec{R}}
\end{align*}
So the centre of mass of a system of particle is like a single particle of mass $M$.

Furthermore, if we find $\dot{\vec{P}}$:
\begin{align*}
  \dot{\vec{P}} = \sum_{i = 1}^{N} \dot{\vec{p}}_i &= \sum_{i = 1}^{N} \left(\vec{F}^\text{ext}_i + \sum_{j = 1, j \neq i}^N \vec{F}_{i j}\right) \\
                                                   &= \sum_{i = 1}^{N} \vec{F}^\text{ext}_i + \sum_{i < j} (\underbrace{\vec{F}_{i j} + \vec{F}_{j i}}_{0}) \text{ using \cref{newtonsThird}}\\
                                                   &= \sum_{i = 1}^{N} \vec{F}^{\text{ext}}_i
\end{align*}
\begin{definition}[Total External Force]
  The \textit{total external force} $\vec{F}$ of a system of particles is:
  \[
    \vec{F} = \sum_{i = 1}^{N}  \vec{F}^{\text{ext}}_i
  \]
\end{definition}
Thus, $\dot{\vec{P}} = \vec{F}$ and since $\vec{P} = M \dot{\vec{R}}$, we then have:
\begin{equation}
  M \ddot{\vec{R}} = \vec{F} = \sum_{i = 1}^{N} \vec{F}^\text{ext}_i \label{comN2}
\end{equation}
So the centre of mass accelerates just like a point particle subject to an external force $\vec{F}$.
The system cannot accelerate itself without an external force.
\begin{remark}
  This property of systems of particle is what allows us to treat the Earth as a point particle as all the internal forces between individual particles cancel each other out when considering the centre of mass of the whole system.
\end{remark}
In particular, if $\vec{F} = \vec{0}$, then $\dot{\vec{P}} = 0$ and so the total momentum is conserved.
\subsection{Total Angular Momentum}
A similar result also holds for the total angular momentum.
\begin{definition}[Total Angular Momentum]
  The \textit{total angular momentum} $\vec{L}$ of a system of particles about a fixed point $\vec{a}$ is:
  \[
    \vec{L} = \sum_{i = 1}^{N} (\vec{x}_i - \vec{a})\times \vec{p}_i
  \]
\end{definition}
\begin{remark}
  Previously in \cref{angularMomentum}, we used $\vec{a} = \vec{0}$ (i.e. about the origin) but here we are using an arbitrary point $\vec{a}$.
\end{remark}
To find the torque of the system $\dot{\vec{L}}$, we take the derivative with respect to $t$:
\begin{align*}
  \dot{\vec{L}} &= \sum_{i = 1}^{N} \dot{\vec{x}}_i \times \vec{p}_i + (\vec{x}_i - \vec{a}) \times \dot{\vec{p}}_i \\
                &= \sum_{i = 1}^{N} \cancelto{\vec{0}}{\dot{\vec{x}}_i \times m \dot{\vec{x}}_i} + (\vec{x}_i - \vec{a}) \times \dot{\vec{p}}_i \\
                &= \sum_{i = 1}^{N} (\vec{x}_i - \vec{a}) \times \left(\vec{F}^{\text{ext}}_i + \sum_{j = 1, j\neq i}^N \vec{F}_{i j}\right) \\
                &= \sum_{i = 1}^{N} (\vec{x}_i - \vec{a}) \times \vec{F}^{\text{ext}}_i + \sum_{i < j} [(\vec{x}_i - \vec{a}) \times \vec{F}_{i j} + (\vec{x}_j - \vec{a}) \times \vec{F}_{j i}]
\end{align*}
We then define the total external torque as:
\begin{definition}[Total Torque]
  The \textit{total external torque} $\vec{G}$ of a system of particles is:
  \[
    \vec{G} = \sum_{i = 1}^{N} (\vec{x}_i - \vec{a}) \times \vec{F}^{\text{ext}}_i
  \]
\end{definition}
Since $\vec{F}_{i j} = -\vec{F}_{j i}$ from \cref{newtonsThird}:
\begin{align*}
  (\vec{x}_i - \vec{a}) \times \vec{F}_{i j} + (\vec{x}_j - \vec{a}) \times \vec{F}_{j i} &= (\vec{x}_i - \vec{a}) \times \vec{F}_{i j} - (\vec{x}_j - \vec{a}) \times \vec{F}_{i j} \\
                                                                                          &= [\vec{x}_i - \vec{x}_j]\times\vec{F}_{i j}
\end{align*}
We then have:
\begin{equation}
  \dot{\vec{L}} = \vec{G} + \sum_{i < j} [\vec{x}_i - \vec{x}_j]\times\vec{F}_{i j} \label{rateOfTotalAngular}
\end{equation}
We would like to show that $\dot{\vec{L}} = \vec{G}$ so that the rate of change of the total angular momentum is just the external torque, however, we cannot just use Newtons 3rd Law to show that the sum in \cref{rateOfTotalAngular} vanishes.

If the force $\vec{F}_{i j}$ between particles $i$ and $j$ comes from a potential that depends on the distance from $\vec{x}_i$ to $\vec{x}_j$ then:
\begin{align*}
  \vec{F}_{i j} &= -\nabla_{\vec{x}_i} (V(|\vec{x}_i - \vec{x}_j|)) \\
                &= -V' (|\vec{x}_i - \vec{x}_j|) \frac{\vec{x}_i - \vec{x}_j}{|\vec{x}_i - \vec{x}_j|}
\end{align*}
as from Vector Calculus: $\nabla_{\vec{u}} f(|\vec{u} - \vec{v}|) = f'(|\vec{u} - \vec{v}|) \nabla_{\vec{u}} |\vec{u} - \vec{v}| = f'(|\vec{u} - \vec{v}|) \frac{\vec{u} - \vec{v}}{|\vec{u} - \vec{v}|}$.

Thus, $\vec{F}_{i j}$ is parallel to $\vec{x}_i - \vec{x}_j$ and so $(\vec{x}_i - \vec{x}_j) \times \vec{F}_{i j} = \vec{0}$.
Therefore, in this case $\dot{\vec{L}} = \vec{G}$.

It turns out that the sum in \cref{rateOfTotalAngular} does indeed vanish for all known forces, however this is more technical for other forces and is not covered in this course.
Since the final sum vanishes, we always have $\dot{\vec{L}} = \vec{G}$, that is, the total torque of the system $\dot{\vec{L}}$ is equal to the external torque.
This means that a system of particles cannot spin itself and needs an external torque to change its total angular momentum $\vec{L}$.
\begin{remark}
  It is natural to take $\vec{a} = \vec{R}$ (i.e. the centre of mass).
  In general, $\vec{R}$ depends on $t$, however, the above proof assumed that $\vec{a}$ was constant, see Example Sheet 3 Q1 for a proof when $\vec{a} = \vec{R}(t)$.
\end{remark}
\subsection{Total Energy}
The \textit{total energy} of the system can also be split up nicely.

It is often convenient to think of a particles position \textit{relative to the centre of mass}.
To do this, we define $\vec{y}_i$ to be such that:
\[
  \vec{x}_i = \vec{R} + \vec{y}_i
\]
We know that:
\begin{align}
  M\vec{R} &= \sum_{i = 1}^{N} m_i \vec{x}_i \nonumber \\
           &= \sum_{i = 1}^{N} m_i\vec{R} + \sum_{i = 1}^{N} \vec{y}_i \nonumber \\
           &= \vec{R}\sum_{i = 1}^{N} m_i + \sum_{i = 1}^{N} \vec{y}_i \nonumber \\
           &= M\vec{R} + \sum_{i = 1}^{N} \vec{y}_i \implies \sum_{i = 1}^{N} \vec{y}_i = \vec{0} \label{relativeSum}
\end{align}
The total kinetic energy $T$ is then:
\begin{align}
  T &= \sum_{i = 1}^{N} \frac{1}{2} m_i \dot{\vec{x}}_i \cdot \dot{\vec{x}}_i \nonumber \\
    &= \sum_{i = 1}^{N} \frac{1}{2} m_i (|\dot{\vec{R}}|^2 + |\dot{\vec{y}}_i|^2 + 2 \dot{\vec{R}} \cdot \dot{\vec{y}}_i) \nonumber \\
    &= \frac{1}{2} |\dot{\vec{R}}|^2\sum_{i = 1}^{N} m_i + \sum_{i = 1}^{N} \frac{1}{2} m_i |\dot{\vec{y}}_i|^2 + 2 \dot{\vec{R}} \cdot\cancelto{\vec{0}}{\left(\sum_{i = 1}^{N} \dot{\vec{y}}_i\right)} \nonumber \\
    &= \underbrace{\frac{1}{2}M |\dot{\vec{R}}|^2}_{\text{CoM K.E.}} + \underbrace{\sum_{i = 1}^{n} \frac{1}{2}m_i |\dot{\vec{y}}_i|^2}_{\text{Internal K.E.}} \label{totalKE}
\end{align}
So we have split the kinetic energy into a centre of mass kinetic energy and sum of ``internal'' kinetic energies with respect to the centre of mass.

To have a conserved total energy $E$, all forces must be conservative.
Here, we will take forces between particles to depend on their separation only:
\begin{align*}
  \vec{F}^{\text{ext}}_i &= -\nabla_{\vec{x}_i} V_i(\vec{x}_i) \\
  \vec{F}_{i j} &= -\nabla_{\vec{x}_i} V_{i j}(|\vec{x}_i  - \vec{x}_j|)
\end{align*}
\begin{remark}[Note]
  We must have $V_{i j} = V_{j i}$, so that Newtons 3rd Law (\cref{newtonsThird}) is obeyed:
  \begin{align*}
    \vec{F}_{i j} &= - \nabla_{\vec{x}_i} V_{i j}(|\vec{x}_i - \vec{x}_j|) \\
                  &= - V_{i j}(|\vec{x}_i - \vec{x}_j|)\frac{\vec{x}_i - \vec{x}_j}{|\vec{x}_i - \vec{x}_j|} \\
                  &= V_{j i}(|\vec{x}_j - \vec{x}_i|)\frac{\vec{x}_j - \vec{x}_i}{|\vec{x}_j - \vec{x}_i|} \\
                  &= \nabla_{\vec{x}_j}V_{i j}(|\vec{x}_j - \vec{x}_i|) = -\vec{F}_{j i} \\
  \end{align*}
\end{remark}
\begin{definition}[Total Energy]
  The \textit{total energy of the system} is defined to be the sum of the total kinetic energy $T$ from above plus the sum of all of the potentials:
  \[
    E = T + \sum_{i = 1}^{N} V_i(\vec{x}_i) + \sum_{i < j} V_{i j}(|\vec{x}_i - \vec{x}_j|)
  \]
\end{definition}
\begin{remark}[Note]
  For each pair of particles, there are two associated forces $\vec{F}_{ij}$ and $\vec{F}_{ji}$, however there is only a single potential $V_{i j} = V_{j i}$ which is why we only take the sum over $i < j$ to avoid double counting potentials.
\end{remark}

We can check that the total energy is conserved by finding $\dot{E}$.

Differentiating $T$ we have:
\[
  \deriv{T}{t} = \sum_{i = 1}^{N} \frac{1}{2} \deriv{}{t}(\dot{\vec{x}}_i \cdot \dot{\vec{x}}_i) = \sum_{i = 1}^{N} m_i \dot{\vec{x}}_i \cdot \ddot{\vec{x}}_i
\]
To differentiate the potential $V_{i j}(|\vec{x}_i - \vec{x}_j|)$ with respect to $t$, we think of it as a function of 6 variables (i.e. three from $\vec{x}_i$ and three from $\vec{x}_j$) and so using the multivariate chain rule:
\begin{align*}
  \deriv{V_{i j}}{t} &= \pderiv{V_{i j}}{(\vec{x}_i)_p}\deriv{(\vec{x}_i)_p}{t} + \pderiv{V_{i j}}{(\vec{x}_j)_q}\deriv{(\vec{x}_j)_q}{t} \text{ ($\Sigma$ convention on $p$ and $q$)}\\
                     &= \dot{\vec{x}}_i \cdot \nabla_{\vec{x}_i} V_{i j} + \dot{\vec{x}}_j \cdot \nabla_{\vec{x}_j} V_{i j}
\end{align*}
Combining these results, we have:
\begin{align*}
  \deriv{E}{t} &= \sum_{i = 1}^{N} m_i \dot{\vec{x}}_i \cdot \ddot{\vec{x}}_i + \sum_{i = 1}^{N} [\dot{\vec{x}}_i \cdot \nabla_{\vec{x}_i} V_i] + \sum_{i < j} [\dot{\vec{x}}_i \cdot \nabla_{\vec{x}_i}V_{i j} + \dot{\vec{x}}_j \cdot \nabla_{\vec{x}_j}V_{i j}] \\
               &= \sum_{i = 1}^{N}  m_i \dot{\vec{x}}_i \cdot \ddot{\vec{x}}_i - \sum_{i = 1}^{N} \vec{F}^{\text{ext}}_i \cdot \dot{\vec{x}}_i - \sum_{i < j} [\vec{F}_{i j} \cdot \dot{\vec{x}}_i + \vec{F}_{ji} \cdot \dot{\vec{x}}_j]
\end{align*}
We can rewrite the final sum as:
\begin{align*}
  \sum_{i < j} [\vec{F}_{i j} \cdot \dot{\vec{x}}_i + \vec{F}_{ji} \cdot \dot{\vec{x}}_j] &= \sum_{i < j} \vec{F}_{i j} \cdot \dot{\vec{x}}_i + \sum_{i < j} \vec{F}_{j i} \cdot \dot{\vec{x}}_j \\
                                                                                          &= \sum_{i < j} \vec{F}_{i j} \cdot \dot{\vec{x}}_i + \sum_{j < i} \vec{F}_{i j} \cdot \dot{\vec{x}}_i \\
                                                                                          &= \sum_{i \neq j} \vec{F}_{i j} \cdot \dot{\vec{x}}_i
\end{align*}
Therefore:
\begin{align*}
  \deriv{E}{t} &= \sum_{i = 1}^{N} \dot{\vec{x}}_i \cdot (m_i \ddot{\vec{x}}_i - \vec{F}^\text{ext}_i) - \sum_{i \neq j} F_{i j} \cdot \dot{\vec{x}}_i \\
               &= \sum_{i = 1}^{N} \dot{\vec{x}}_i \cdot \left[m_i \ddot{\vec{x}}_i - \vec{F}^\text{ext}_i - \sum_{j = 1, j \neq i}^n F_{i j}\right] \\
               &= \sum_{i = 1}^{N} \dot{\vec{x}}_i \cdot [\vec{F}_i - \vec{F}_i] = 0
\end{align*}
and so the total energy of the system $E$ is conserved.
\section{The Two Body Problem}
An important special case of a system of particles is the \textit{two body problem}.
This is when we have two particles with \textbf{no external forces}.

Let $\vec{r} = \vec{x}_1 - \vec{x}_2$ be the relative separation of the two particles.
Their centre of mass $\vec{R}$ satisfies:
\[
  M \vec{R} = m_1 \vec{x}_1 + m_2 \vec{x}_2
\]
and so we can write the positions of the particles as:
\begin{align*}
  \vec{x}_1 &= \vec{R} + \frac{m_2}{M}\vec{r} = \vec{R} + \vec{y}_1 \\
  \vec{x}_2 &= \vec{R} - \frac{m_1}{M}\vec{r} = \vec{R} + \vec{y}_2
\end{align*}
where $M = m_1 + m_2$.

The second terms in each of these expressions are then the positions of the particles relative to the centre of mass of the system (i.e. $\vec{y}_1$ and $\vec{y}_2$ from the last section).
We see that they obey $m_1\vec{y}_1 + m_2\vec{y}_2 = \vec{0}$, which is consistent with \cref{relativeSum}.

Using \cref{totalKE}, we can find the total kinetic energy of the system:
\begin{align}
  T &= \frac{1}{2}M |\dot{\vec{R}}|^2 + \frac{1}{2}m_1 |\dot{\vec{y}}_1|^2 + \frac{1}{2}m_2 |\vec{y}_2|^2 \nonumber \\
    &= \frac{1}{2}M |\dot{\vec{R}}|^2 + \frac{m_1}{2} \frac{m^2_2}{M^2} |\dot{\vec{r}}|^2 + \frac{m_2}{2} \frac{m^2_1}{M^2} |\dot{\vec{r}}|^2 \nonumber \\
    &= \frac{1}{2}M |\dot{\vec{R}}|^2 + \frac{1}{2}\left[\frac{m_1m^2_2 + m^2_1m_2}{(m_1 + m_2)^2}\right]|\dot{\vec{r}}|^2 \nonumber \\
    &= \frac{1}{2}M |\dot{\vec{R}}|^2 + \frac{1}{2} \mu |\dot{\vec{r}}|^2 \label{totalKE2Body}
\end{align}
where $\mu = \frac{m_1 m_2}{m_1 + m_2}$ is the \textit{reduced mass}.
So the kinetic energy of the system can be separated into kinetic energy for the centre of mass with mass $M$, and kinetic energy for the relative separation with effective mass $\mu$.
We have seen already that the centre of mass acts like a point particle of mass $M$ and this calculation suggests that the relative separation $\vec{r}$ acts like a point particle of mass $\mu$.

With no external forces, we know from \cref{comN2} that $\ddot{\vec{R}} = \vec{0}$.
But what about $\ddot{r}$?
If we treat $\vec{r}$ like a particle with mass $\mu$ as suggested by \cref{totalKE2Body}, then we see that:
\begin{align*}
  \mu \ddot{\vec{r}} &= \mu (\ddot{\vec{x}}_1 - \ddot{\vec{x}}_2) \\
                     &= \mu \left(\frac{\vec{F}_{1 2}}{m_1} - \frac{\vec{F}_{2 1}}{m_2}\right) \text{ using N2 (\cref{newton2})}\\
                     &= \mu \left(\frac{1}{m_1} + \frac{1}{m_2}\right)\vec{F}_{1 2} \text{ using N3 (\cref{newtonsThird})}\\
                     &= \frac{\mu}{\mu}\vec{F}_{1 2} = \vec{F}_{1 2}
\end{align*}
So the relative separation also behaves like a single particle problem with force $\vec{F}_{1 2}$.
This means we can use the methods we have already developed throughout the course to analyse this.

If one mass is much larger than the other, then we see that the reduced mass is approximately the smaller mass.
That is, if $m_1 \gg m_2$, then:
\[
  \mu = \frac{m_1 m_2}{m_1 + m_2} \approx \frac{m_1m_2}{m_1} = m_2
\]
So in this limit, the more massive object is essentially stationary and the lighter object orbits around it and so we have reduced the problem to the one body Kepler problem (\cref{keplerSection}) for the relative position $\vec{r}$.
\begin{remark}
  We see that the two body problem can be reduced to the Kepler problem, however, the \textit{three body problem} does not have a closed form solution.
  It is a chaotic system and very small perturbations to the initial conditions grow exponentially.

  Surprisingly enough, when the number of particles is very large, the problem simplifies and can be analysed using statistical physics.
\end{remark}
\section{Rocket Equation and Variable Mass}
The relation $\dot{\vec{P}} = \vec{F}^{\text{ext}}$ from \cref{comN2} is useful when the internal forces of the system are complicated as we know only external forces change the total momentum $\dot{\vec{P}}$ of the system.

Consider a rocket that ejects fuel with speed $u$ relative to the rocket.
So far in the course, we have only considered constant masses, however the mass of this rocket depends on time as the longer the rocket has been in flight for, the more fuel it has ejected.

For ease, consider only motion in the $z$-direction so that we use the scalar quantities $\dot{p}$ and $F^{\text{ext}}$.
We don't care about the mechanics of how the fuel is ejected but the total momentum must obey:
\[
  \dot{p} = \frac{p(t + \delta t) - p(t)}{\delta t} = F^{\text{ext}}
\]
for small $\delta t$.

The total momentum of the system at time $t$ is:
\[
  p(t) = m(t) v(t)
\]
and at time $t + \delta t$:
\begin{align*}
  p(t + \delta t) &= \overbrace{m(t + \delta t)v(t + \delta t)}^{\text{momentum of rocket}} + \overbrace{(m(t) - m(t + \delta t))}^{\text{ejected fuel}}\overbrace{(v(t) - u)}^{\text{ejection velocity}} \\
                  &= m(t)v(t) + [m'(t)v(t) + m(t)v'(t)] \delta t - m'(t)(v(t) - u) \delta t \\
                  &= m(t)v(t) + m(t)v'(t)\delta t + um'(t)\delta t
\end{align*}
\begin{remark}[Note]
  To first order, it does not matter if we consider the fuel having velocity $v(t) - u$ (at the start of the time step) or $v(t + \delta t) - u$ (at the end of the time step).
\end{remark}
Subsisting back into $\dot{p}$, we have:
\[
  \frac{p(t + \delta t) - p(t)}{\delta t} = \frac{mv + m\dot{v}\delta t + u\dot{m}\delta t - mv}{\delta t} = m\dot{v} + u\dot{m}
\]
We then obtain the \textit{Tsiolkovsky Rocket Equation}:
\[
  m \dot{v} + u \dot{m} = F^{\text{ext}}
\]
\begin{remark}[Note]
  There are several examples using this on Example Sheet 3.
\end{remark}

We can rearrange this as:
\[
  m \dot{v} = F^{\text{ext}} - u\dot{m}
\]
This is just Newton's equation for the rocket, we can think of the $-u\dot{m}$ as a \textit{thrust force} on the rocket from the fuel which acts to increase velocity as $\dot{m} < 0$.
\end{document}
