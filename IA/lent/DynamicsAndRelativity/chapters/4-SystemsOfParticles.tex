\documentclass[../main.tex]{subfiles}
\begin{document}
\chapter{Systems of Particles}
\section{General Systems of Particles}
Suppose we have $N$ particles indexed by $i = 1, \ldots, N$ with momenta $\vec{p}_i = m_i x_i$ and all obeying Newton's law individually, that is, for each particle:
\[
  \dot{\vec{p}}_i = \vec{F}_i
\]
The force $F_i$ on the $i$-th particle can be \textbf{external} or \textbf{due to the other particles}:
\[
  \vec{F}_i = \vec{F}^{\text{ext}}_i + \sum_{j \neq i} \vec{F}_{ij}
\]
where $\vec{F}^{\text{ext}}_i$ is the external force on the $i$-th particle and $\vec{F}_{ij}$ is the force on the $i$-th particle due to the $j$-th particle.

\begin{law}[Newton's 3rd Law]
  The forces between particles are found to obey:
  \[
    \vec{F}_{ij} = - \vec{F}_{ji}
  \]
  This is often stated as:
  \[
    \text{``Every action has an equal an opposite reaction''}
  \]
\end{law}
\begin{remark}
  An example of Newton's 3rd law can be seen in \cref{gravity}.
  If we swap the roles of the two particles around, all that changes is the direction with which the force acts in.
\end{remark}
\end{document}
