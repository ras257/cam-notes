\documentclass[../main.tex]{subfiles}
\begin{document}
\chapter{Continuity of Functions}
\section{Limits of Functions}
Consider a function $f: X \to \C$ for some $X \subseteq \C$.
We want to define what we mean by $f(z) \to y$ as $z \to a$, even if $a \notin X$.
\begin{example}
  For example, consider $f: \R \setminus \{0\} \to \R,\ x \mapsto \frac{\sin x}{x}$.
  \begin{center}
  \begin{tikzpicture}[scale=1.13, >=stealth]
    \draw[->] (-5, 0) -- (5, 0) node[right] {$x$};
    \draw[->] (0, -0.8) -- (0, 2) node[above] {$f$};

    \def\kconst{3}
    \draw[thick, domain=0.01:5, smooth, samples=100] plot (\x,{1.5 * sin((\x * \kconst) r)/(\x * \kconst)});
    \draw[thick, domain=-5:-0.01, smooth, samples=100] plot (\x,{1.5 * sin((\x * \kconst) r)/(\x * \kconst)});

    \filldraw[fill=white, draw=black] (0, 1.5) circle (1.3pt);
  \end{tikzpicture}
  \end{center}
  $0$ is not in the domain of $f$ as the function is undefined there.

  We hope to make sense of ``$\lim_{x \to 0} \frac{\sin x}{x}$'' as there are points in the domain that are infinitely close to $0$.
  We say that 0 is an \textit{accumulation point}, as for any threshold $\delta > 0$ we can always find an $x \in \R \setminus \{0\}$ such that $|x - 0| < \delta$.

  We would quite like to be able to find this limit so that we can define a new function $g$ that is identical to $f$ but without the discontinuity at $x = 0$:
  \[
    g(x) = \begin{cases}
    \frac{\sin x}{x} & x \neq 0 \\
    \lim\limits_{x \to 0} \frac{\sin x}{x} & x = 0
    \end{cases}
  \]
\end{example}
\begin{definition}[Accumulation Point]
  We say $a \in \C$ is a \textit{accumulation point} for $X \subseteq \C$ if for any threshold $\delta > 0$, $\exists z \in X \setminus \{a\}$ such that $|z - a| < \delta$.

  If $a \in X$ and the above fails, then we say that $a$ is \textit{isolated}.
\end{definition}
\begin{example}
  \begin{enumerate}
    \item Consider $X = \{0\} \cup [1, 2]$.
      $0$ is an isolated point of $X$ and all of the points in $[1, 2]$ are accumulation points of $X$.
    \item The points on the circle $S = \{z \in \C: |z| = 1\}$ are accumulation points for $D = \{z \in \C : |z| < 1\}$.

      All the points in $D$ are also accumulation points for $D$.
    \item All points in the set $\{z \in \C: |z| \leq 1\}$ are accumulation points of the set itself.
  \end{enumerate}
\end{example}
\begin{lemma}
  Let $X \subseteq \C$ and $a \in \C$.
  $a$ is an accumulation point for $X$ if and only if there exists a sequence $(x_n)$ in $X \setminus \{a\}$ s.t. $\lim_{n \to \infty} x_n = a$.
  \label{accumulationSequence}
\end{lemma}
\begin{proof}
  \begin{proofdirection}{Assume $a$ is an accumulation point}
    Consider the sequence $(x_n)$ where each $x_n$ is defined to be some $z \in X \setminus \{a\}$ such that $|z - a| < \frac{1}{n}$.
    Such a $z$ will always exist as $\delta = \frac{1}{n} > 0$ and $a$ is an accumulation point of $X$.

    For any $\varepsilon > 0$, $\exists N \text{ s.t. } 0 < \frac{1}{N} < \varepsilon$.
    For all $n \geq N$, $\frac{1}{n} \leq \frac{1}{N} < \varepsilon$ so by construction of the sequence $|x_n - a| < \frac{1}{n} < \varepsilon\ \forall n \geq N$.
    Thus we have constructed a sequence $(x_n)$ in $X \setminus \{a\}$ s.t. $\lim_{n \to \infty} x_n = a$.
  \end{proofdirection}
  \begin{proofdirection}{Assume there is such a sequence}
    Since $x_n \to a$, $\forall \delta > 0$, $\exists N$ s.t. $|x_n - a| < \delta\ \forall n \geq N$.
    $x_N \in X \setminus \{a\}$ and satisfies $|x_N - a| < \delta$.
    Therefore $a$ is an accumulation point.
  \end{proofdirection}
\end{proof}
To make sense of ``$f(z) \to y$ as $z \to a$'' for $f: X \subseteq \C \to \C$, we would like the following:
\begin{itemize}
  \item We either want $a \in X$, or if $a \notin X$, we want $a$ to be an accumulation point of $X$ so that we can get arbitrarily close to $x$.
  \item Looking at points $z$ ``near'' $a$, no matter how small our threshold $\varepsilon > 0$, there must exists points in the image of $f$ which are $\varepsilon$-close to $y$.
  That is, $\{z \in X: |f(z) - y| < \varepsilon\}$ must be non-empty.
\item We also want it to contain all the points in the domain of $f$ which are sufficiently close $a$, i.e. $\delta = \delta(\varepsilon)$ close to $a$.
\end{itemize}
\begin{definition}[Limit of a function]
  Let $f: X \subseteq \C \to \C$ and let $a \in X$ or let $a \in \C \setminus X$ such that $a$ is an accumulation point of $X$.

  We say that \textit{$f$ tends to $y \in \C$ as $z$ tends to $a$} if:
  \[
    \forall \varepsilon > 0\ \exists \delta = \delta(\varepsilon) \text{ s.t. } (|z - a| < \delta \text{ for } z \in X \setminus \{a\} \implies |f(z) - y| < \varepsilon)
  \]
  and write $f(z) \to y$ as $z \to a$.
\end{definition}
\begin{remark}
  If $a$ is an isolated point of $X$, then we can always find $\delta$ sufficiently small so that $|z - a| < \delta,\ z \in X \iff z = a$.
  For such $\delta$, $|z - a| < \varepsilon \text{ for } z \in X \setminus \{a\} \implies |f(z) - y| < \varepsilon$ is vacuously true as there are no such $z \in X \setminus \{a\}$ satisfying $|z - a| < \varepsilon$.
  Thus, $f(z) \to f(a)$ as $z \to a$ trivially.
  So, the definition makes sense for isolated points, however, is not very interesting.
\end{remark}
\begin{definition}[Divergence to infinity of a function]
  Let $a \in \C \setminus X$ be an accumulation point for $X$.
  We say that $f$ diverges to $\infty$ if as $z \to a$ if $\lim\limits_{z \to a} \frac{f(z)}{|f(z)|}$ exists and:
  \[
    \forall L > 0,\ \exists \delta = \delta(L) \text{ s.t. } (|z - a| < \delta \text{ for } z \in X \implies |f(z)| > L)
  \]
\end{definition}
\begin{remark}
  For $f(z)$ diverging to $\infty$ as $z \to a \in \C$ for finite $a$, we cannot have $a \in X$ otherwise $f$ would not be a well defined function, hence why the above definition only considers $a \in \C \setminus X$ which are accumulation points.
\end{remark}
\begin{example}
  Consider again $f(x) = \frac{\sin x}{x}$ with domain $\R \setminus \{0\}$.
  We claim that $\frac{\sin x}{x} \to 1$ as $x \to 0$.

  We can show geometrically that:
  \begin{equation}
    \cos x < \frac{\sin x}{x} < 1 \label{trigBound}
  \end{equation}
  for all $x \in (0, \frac{\pi}{2})$ using a unit circle.
  Therefore:
  \[
    \abs{\frac{\sin x}{x} - 1} < 1 - \cos x = 2 \sin^2 \left(\frac{x}{2}\right)
  \]
  Substituting $\frac{x}{2}$ into \cref{trigBound}, we have:
  \[
    2\sin\left(\frac{x}{2}\right) < x \implies 2\sin^2\left(\frac{x}{2}\right) < \frac{x^2}{2}
  \]
  Hence $\forall \varepsilon > 0$, we need to choose $\delta(\varepsilon)$ such that:
  \[
    |x - 0| < \delta \implies \abs{\frac{\sin x}{x} - 1} < \frac{x^2}{2} < \varepsilon
  \]
  Choosing $\delta = \sqrt{2\varepsilon}$ satisfies this so $f(x) \to 1$ as $x \to 0$.
\end{example}
\begin{lemma}
  Let $f: X \subseteq \C \to \C$ and $a \in \C$ be either in $X$ or an accumulation point of $X$.
  \begin{enumerate}
    \item $f(z) \to y$ as $z \to a$ if and only if $f(z_n) \to y$ for every sequence $(z_n)$ on $X$ with $z_n \to a$ for a non-constant $z_n$
    \item $f(z)$ diverges as $z \to a$ if and only if $(f(z_n))_n$ diverges for every $(z_n)_{n \in \N}$ with $z_n \to a$ for a non-constant $z_n$
  \end{enumerate}
\end{lemma}
\begin{proof}[\textbf{i}]
  \begin{proofdirection}{Assume $f(z) \to y$ as $z \to a$}
    Suppose we have a sequence $(z_n)$ on $X$ with $z_n \to a$.

    Since $f(z) \to y$, $\forall \varepsilon > 0\ \exists \delta > 0$ s.t. $|z - a| < \delta \implies |f(z) - y| < \varepsilon$.
    Since $z_n \to a$, for all such $\delta$, $\exists N$ s.t. $|z_n - a| < \delta\ \forall n \geq N$ and so $|f(z_n) - y| < \varepsilon\ \forall n \geq N$.

    Thus given some $\varepsilon > 0$, $\exists N \text{ s.t. } |f(z_n) - y| < \varepsilon\ \forall n > N$ so $f(z_n) \to y$.
  \end{proofdirection}
  \begin{proofdirection}{Assume $f(z_n) \to y$ for every sequence with $z_n \to a$}
    Unfinished. %TODO
  \end{proofdirection}
\end{proof}
\begin{proof}[\textbf{ii}]
  Unfinished. %TODO
\end{proof}
\begin{lemma}[Uniqueness of Limits]
  If $f: X \subseteq \C \to \C$  has a limit at $a \in \C$, then this limit is unique.
\end{lemma}
\begin{proof}
  Suppose $f(z) \to y$ as $z \to a$ and $f(z) \to x$ as $z \to a$.
  Then $\forall \varepsilon > 0$:
  \begin{align*}
    \exists \delta_1 = \delta_1(\varepsilon) \text{ s.t. } |z - a| < \delta_1 \implies |f(z) - y| < \varepsilon \\
    \exists \delta_2 = \delta_2(\varepsilon) \text{ s.t. } |z - b| < \delta_2 \implies |f(z) - x| < \varepsilon
  \end{align*}
  Hence $\forall z \text{ s.t. } |z - a| < \min\{\delta_1, \delta_2\}$:
  \[
    |x - y| = |(x - f(z)) - (y - f(z))| \leq |f(z) - x| + |f(z) - y| < 2\varepsilon
  \]
  Since $\varepsilon > 0$ is arbitrary, $|x - y| \leq 0$ so $x = y$.
\end{proof}
\begin{remark}
  The limit is unique so we write $\lim\limits_{z \to a} f(z)$ to mean the unique limit of $f(z)$ as $z \to a$.
\end{remark}
\begin{lemma}[Limit Laws]
  Let $f, g: X \subseteq \C \to \C$ and $a \in \C$ either in $X$ or an accumulation point.
  Suppose $\lim_{z \to a} f(z) = y$ and $\lim_{z \to a} g(z) = x$ then:
  \label{limitLaws}
  \begin{enumerate}
    \item $\lim_{z \to a} (f(z) + g(z)) = y + x$
    \item $\lim_{z \to a} (f(z)g(z)) = yx$
    \item If $g(z) \neq 0$ and $x \neq 0$, then $\lim_{z \to a} \frac{f(z)}{g(z)} = \frac{y}{x}$.
  \end{enumerate}
\end{lemma}
\begin{proof}
  Suppose $f(z) \to y$ as $z \to a$ and $f(z) \to x$ as $z \to a$.
  Then $\forall \varepsilon > 0$:
  \begin{align*}
    \exists \delta_1 = \delta_1(\varepsilon) \text{ s.t. } |z - a| < \delta_1 \implies |f(z) - y| < \varepsilon \\
    \exists \delta_2 = \delta_2(\varepsilon) \text{ s.t. } |z - a| < \delta_2 \implies |g(z) - x| < \varepsilon
  \end{align*}
  \begin{enumerate}
    \item $\forall z \text{ s.t. } |z - a| < \min\{\delta_1, \delta_2\}$ we have:
      \[
        |f(z) + g(z) - (y + x)| \leq |f(z) - y| + |g(z) - x| < 2\varepsilon
      \]
      which is a constant multiple of $\varepsilon$, so as $\varepsilon$ is arbitrary, $\lim_{z \to a} (f(z) + g(z)) = y + x$.
    \item We can also pick $\varepsilon = 1$ so that:
      \[
        \exists \delta_3 \text{ s.t. } |z - a| < \delta_3 \implies |f(z) - y| < \varepsilon \implies |f(z)| < 1 + |y|
      \]
      Then $\forall z \text{ s.t. } |z - a| < \min\{\delta_1, \delta_2, \delta_3\}$:
      \begin{align*}
        |f(z)g(z) - yx| &= |f(z)g(z) - f(z)x + f(z)x - yx| \\
                        &\leq |f(z)||g(z) - x| + |x||f(z) - y| \\
                        &< (1 + |y|)\varepsilon + |x|\varepsilon \\
                        &= \varepsilon(1 + |y| + |x|)
      \end{align*}
      which is a constant multiple of $\varepsilon$, so $\lim_{z \to a} f(z)g(z) = yx$.
    \item To show this, we can show that $\lim_{z \to a} \frac{1}{g(z)} = \frac{1}{x}$ and then use \textbf{(ii)}.
      We can pick $\varepsilon = \abs{\frac{x}{2}} > 0$ so that:
      \[
        \exists \delta_3 \text{ s.t. } |z - a| < \delta_3 \implies |g(z) - x| < \abs{\frac{x}{2}} \implies |g(z)| > \abs{\frac{x}{2}} > 0
      \]
      Then $\forall z \text{ s.t. } |z - a| < \min\{\delta_1, \delta_2, \delta_3\}$:
      \begin{align*}
        \abs{\frac{1}{g(z)} - \frac{1}{x}} &= \frac{1}{|x||g(z)|}|g(z) - x| \\
                                       &< \frac{\varepsilon}{|x||\frac{x}{2}|}
      \end{align*}
      Therefore, $\lim_{z \to a} \frac{1}{g(z)} = \frac{1}{x}$ and so the result follows using \textbf{(ii)}.
      \end{enumerate}
\end{proof}
\begin{remark}
  The strategies used here are similar to those used when proving \cref{limitLaws} and \cref{reciprocalConvergence}, however, instead of bounding the whole sequence, we just introduce a new threshold $\delta$ such that if $|z - a| < \delta$, then $|f(z)|$ is bounded by some constant.
\end{remark}
\section{Continuity of Functions}
What does it mean for a function to be continuous at a point $a$?
\begin{definition}[Continuity]
  Let $f: X \subseteq \C \to \C$ and $a \in X$.
  We say that $f$ is \textit{continuous} at $a$ if:
  \[
    \forall \varepsilon > 0\ \exists \delta = \delta(\varepsilon) \text{ s.t. } |z - a| < \delta \text{ for } z \in X \implies |f(z) - f(a)| < \varepsilon
  \]
  We say that $f$ is \textit{continuous on $X$} if it is continuous at every point $a \in X$.
\end{definition}
\begin{remark}
  For any $a \in X$, either:
  \begin{itemize}
    \item $a$ is an isolated point, then, somewhat tautologically, $f$ is continuous at $a$ because $|z - a| < \delta \implies z = a \implies f(z) = f(a)$ for $\delta$ sufficiently small.
    \item $a$ is an accumulation point, then $f$ is continuous if and only if $\lim\limits_{z \to a} f(z) = f(a)$.
  \end{itemize}
\end{remark}
\begin{example}
  \begin{enumerate}
    \item The function $f(z) = z$, the identity function, is continuous at any $a \in \C$ as $\forall \varepsilon > 0$:
      \[
        |z - a| < \delta(\varepsilon) \implies |f(z) - f(a)| = |z - a| < \varepsilon
      \]
      by picking $\delta(\varepsilon) = \varepsilon$.
    \item The function $f(z) = |z|$ is continuous  at any $a \in \C$ as $\forall \varepsilon > 0$:
      \[
        |z - a| < \delta(\varepsilon) \implies |f(z) - f(a)| = ||z| - |a|| \leq |z - a| < \varepsilon
      \]
      by picking $\delta(\varepsilon) = \varepsilon$.
    \item The function $f(x) = \sin x$ is continuous as a real function.

      Fix some arbitrary $a \in \R$, then for any $x \in \R$:
      \begin{align*}
        |f(x) - f(a)| &= |\sin x - \sin a| \\
                      &= 2 \abs{\cos\left(\frac{x+a}{2}\right)\sin\left(\frac{x - a}{2}\right)} \\
                      &\leq 2 \abs{\sin\left(\frac{x - a}{2}\right)} \\
                      &\leq |x - a| \text{ as $\sin x \leq |x|\ \forall x \in \R$}
      \end{align*}
      Again, picking $\delta(\varepsilon) = \varepsilon$, we have $\forall \varepsilon > 0$:
      \[
        |x - a| < \delta \implies |f(x) - f(a)| \leq |x - a| < \delta = \varepsilon
      \]
      and so $f$ is continuous at every $a \in \R$.
  \end{enumerate}
\end{example}
There is also another notion of continuity using sequences.
\begin{definition}[Sequential Continuity]
  Let $f: X \subseteq \C \to \C$ and $a \in X$.
  We say that $f$ is sequentially continuous at $a$ if $f(z_n) \to f(a)$ for every sequence $(z_n)$ on $X$ with $z_n \to a$.
\end{definition}
\begin{proposition}[Equivalence of Sequential and Standard Continuity]
  Let $f: X \subseteq \C \to \C$.
  $f$ is continuous at $a \in X$ if and only if it is sequentially continuous at $a$.
\end{proposition}
\begin{proof}
  \begin{proofdirection}{Assume $f$ is continuous}
    By continuity of $f$, $\forall \varepsilon > 0,\ \exists \delta = \delta(\varepsilon) \text{ s.t. } |z - a| < \delta \implies |f(z) - f(a)| < \varepsilon$.

    Take any sequence $(z_n)$ on $X$ with $z_n \to a$.
    Since it converges to $a$, $\exists N = N(\delta(\varepsilon)) \text{ s.t. } |z_n - a| < \delta\ \forall n \geq N \implies |f(z_n) - f(a)| < \varepsilon\ \forall n \geq N$.

    Thus for arbitrary $\varepsilon > 0,\ \exists N(\varepsilon) \text{ s.t. } |f(z_n) - f(a)| < \varepsilon \ \forall n \geq n$ so $f(z_n) \to f(a)$.
  \end{proofdirection}
  \begin{proofdirection}{Assume $f$ is sequentially continuous}
    For the purposes of contradiction, assume that $f$ is not continuous at $a$.
    Then $\exists \varepsilon > 0 \text{ s.t. } \forall \delta > 0\ \exists z = z(\delta) \text{ s.t. } |z - a| < \delta$ but $|f(z) - f(a)| \geq \varepsilon$.

    Consider $\delta = \frac{1}{n}$.
    As in the proof for \cref{accumulationSequence}, we can find always find $z_n$ such that $|z_n - a| < \frac{1}{n} = \delta$ since $a$ is an accumulation point.
    We define the sequence $(z_n)$ to be all such $z_n$ for $n \in \N$.
    Thus we have constructed a sequence with $z_n \to a$, however $\forall n, |f(z_n) - f(a)| \geq \varepsilon$ so $f(z_n) \centernot\to f(a)$.
    Hence $(z_n)$ contradicts sequential continuity at $a$.
  \end{proofdirection}
\end{proof}
\begin{example}[Using Sequential Continuity]
  \begin{enumerate}
    \item The function:
      \[
        f(x) = \begin{cases}
          \sin\left(\frac{1}{x}\right) & x \neq 0 \\
          0 & x = 0
        \end{cases}
      \]
      is not continuous.

      Consider the sequence $(x_n)$ on $\R$ given by:
      \[
        x_n = \frac{1}{(2n + \frac{1}{2})\pi}
      \]
      $x_n \to 0$, so if $f$ is sequentially continuous, then we would expect $f(x_n) \to 0$.
      However, $f(x_n) = \sin((2n + \frac{1}{2})\pi) = 1\ \forall n \implies f(x_n) \to 1$ and so $f$ is not sequentially continuous and therefore also not continuous.
    \item The \textit{Dirichlet function}:
      \[
        f(x) = 1_\Q(x) = \begin{cases}
        1 & \text{ if } x \in \Q \\
        0 & \text{ if } x \notin \Q
        \end{cases}
      \]
      is discontinuous at every point in $\R$.

      \begin{itemize}
        \item For $a \in \Q$, there exists a sequence $(x_n)$ on $\R$ with $x_n \to a$.
          But $f(x_n) = 0\ \forall n \implies f(x_n) \to 0$ but $0 \neq 1 = f(a)$ and so $f$ is not continuous at $a$.
        \item For $a \in \R \setminus \Q$, there exists a sequence $(x_n)$ on $\R \setminus \Q$ with $x_n \to a$.
          But $f(x_n) = 1\ \forall n \implies f(x_n) \to 1$ but $1 \neq 0 = f(a)$ and so $f$ is not continuous at $a$
      \end{itemize}
      Therefore it is not continuous for all points in $\Q \cup (\R \setminus \Q) = \R$.

      This can also be shown using the non sequential definition of continuity, again utilising that both irrational and rational numbers are dense in the reals.
  \end{enumerate}
\end{example}
\begin{lemma}[Continuity Laws]
  If $f, g: X \subseteq \C \to \C$ are continuous at $a \in X$, then so are:
  \begin{enumerate}
    \item $\lambda f + g$ for fixed $\lambda \in \C$
    \item $fg$
    \item $\frac{1}{f}$ provided that $f(z) \neq 0\ \forall z \in X$.
  \end{enumerate}
\end{lemma}
\begin{proof}
  Since $f$ and $g$ are continuous at $a$, $\forall \varepsilon > 0$:
  \begin{align*}
    \exists \delta_1 = \delta_1(\varepsilon) \text{ s.t. } |z - a| < \delta_1 \implies |f(z) - f(a)| < \varepsilon \\
    \exists \delta_2 = \delta_2(\varepsilon) \text{ s.t. } |z - a| < \delta_2 \implies |g(z) - f(a)| < \varepsilon
  \end{align*}
  \begin{enumerate}
    \item $\forall z \text{ s.t. } |z - a| < \min\{\delta_1, \delta_2\}$, we have:
      \begin{align*}
        |\lambda f(z) + g(z) - \lambda f(a) - g(a)| &\leq |\lambda||f(z) - f(a)| + |g(z) - g(a)| \\
                                                    &< \varepsilon(|\lambda| + 1)
      \end{align*}
      This is a constant multiple of $\varepsilon$, so $\lambda f + g$ is continuous at $a$.
    \item We can also pick $\varepsilon = 1$ so that:
      \[
        \exists \delta_3 \text{ s.t. } |z - a| < \delta_3 \implies |f(z) - f(a)| < \varepsilon \implies |f(z)| < 1 + |f(a)|
      \]
      Then $\forall z \text{ s.t. } |z - a| < \min\{\delta_1, \delta_2, \delta_3\}$:
      \begin{align*}
        |f(z)g(z) - f(a)g(a)| &= |f(z)g(z) - f(z)g(a) + f(z)g(a) - f(a)g(a)| \\
                        &\leq |f(z)||g(z) - f(a)| + |g(a)||f(z) - f(a)| \\
                        &< (1 + |f(a)|)\varepsilon + |g(a)|\varepsilon \\
                        &= \varepsilon(1 + |f(a)| + |g(a)|)
      \end{align*}
      which is a constant multiple of $\varepsilon$, so $fg$ is continuous at $a$.
    \item We can pick $\varepsilon = \abs{\frac{f(a)}{2}} > 0$ so that:
      \[
        \exists \delta_3 \text{ s.t. } |z - a| < \delta_3 \implies |f(z) - f(a)| < \abs{\frac{f(a)}{2}} \implies |f(z)| > \abs{\frac{f(a)}{2}} > 0
      \]
      Then $\forall z \text{ s.t. } |z - a| < \min\{\delta_1, \delta_2, \delta_3\}$:
      \begin{align*}
        \abs{\frac{1}{f(z)} - \frac{1}{f(a)}} &= \frac{1}{|f(a)||f(z)|}|f(z) - f(a)| \\
                                       &< \frac{\varepsilon}{|f(a)||\frac{f(a)}{2}|}
      \end{align*}
      which is a constant multiple of $\varepsilon$, so $\frac{1}{f}$ is continuous at $a$.
  \end{enumerate}
\end{proof}
\begin{remark}
  The above proof is very similar to the proof of limit laws in \cref{limitLaws}.
\end{remark}
Function composition also preserves continuity.
\begin{lemma}[Continuity Under Composition]
  Let $U, V \subseteq \C$ and $f: U \to V$ and $g: V \to \C$.
  If $f$ is continuous at $a \in U$ and $g$ is continuous at $f(a) \in V$, then $g \circ f: U \to \C$ is continuous at $a$.
\end{lemma}
\begin{proof}
  For arbitrary $\varepsilon > 0$, since $g$ is continuous at $f(a)$:
  \[
    \exists \delta_1 = \delta_1(\varepsilon) \text{ s.t. } |y - f(a)| < \delta_1 \text{ for } y \in V \implies |g(y) - g(f(a))| < \varepsilon
  \]
  Since $f$ is continuous at $a$, using $\delta_1$ in place of $\varepsilon$ in the definition:
  \[
    \exists \delta_2(\delta_1) = \delta_2(\varepsilon) \text{ s.t. } |z - a| < \delta_2 \text{ for } z \in U \implies |f(z) - f(a)| < \delta_1
  \]
  Bringing this all together, $\forall \varepsilon > 0$:
  \begin{align*}
    \exists \delta_2 = \delta_2(\varepsilon&) \text{ s.t. } |z - a| < \delta_2 \text{ for } z \in U \\
    &\implies |f(z) - f(a)| < \delta_1 \text{ for $f(z) \in V$} \\
    &\implies |g(f(z)) - g(f(a))| < \varepsilon
  \end{align*}
  Thus, $g \circ f$ is continuous at $a$.
\end{proof}
\section{Extreme Value Theorem}
\begin{definition}[Closed Set]
  A set $X \subseteq \C$ is \textit{closed} if all the sequences $(x_n)$ in $X$ which converge in $\C$ have their limit in $X$.
\end{definition}
\begin{remark}
  We use this definition here as it applies to $\C$ and any arbitrarily defined set, however, it is often useful to relate this definition to the real interval $[a, b]$ because that is what we will be primarily using.
\end{remark}
\begin{example}
  The interval $[0, 1]$ is closed whereas the interval $(0, 1)$ is no.
  Neither are mixed intervals like $[0, 1)$ or $(0, 1]$.

  Intervals with infinite endpoints can be closed, for example, $[0, \infty)$ is closed while $(0, \infty)$ is not.
\end{example}
\begin{definition}[Bounded Set]
  A set $X \subseteq \C$ is \textit{bounded} if there exists $\exists M > 0$ such that:
  \[
    X \subseteq \{z \in \C: |z| \leq M\}
  \]
  or equivalently, if $\exists M > 0$ such that:
  \[
    \sup_{z \in X} |z| \leq M
  \]
\end{definition}
\begin{example}
  The intervals $[0, 1], (0, 1), [0, 1), [0, 1]$ are all bounded but $[0, \infty)$ and $(0, \infty)$ are not bounded.
\end{example}
\begin{proposition}[Continuity Preserves Bounded Closed Sets]
  Let $X \subseteq \C$ be a bounded closed set.
  If $f: X \to \C$ is continuous, then $f(X) \subseteq \C$ is also a bounded closed set.
\end{proposition}
\begin{proof}
  Covered next lecture.
\end{proof}
\begin{theorem}[Extreme Value Theorem]
  Let $X \subseteq \R$ be a bounded closed set.
  If $f: X \to \R$ is continuous, then $\exists a, b \in X$ such that:
  \[
    f(a) = \sup f(X) = \sup_{z \in X} f(z),\ f(b) = \inf f(X) = \inf_{z \in X} f(z)
  \]
\end{theorem}
\begin{proof}
  Covered next lecture.
\end{proof}
\end{document}
