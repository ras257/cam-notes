\documentclass[../main.tex]{subfiles}
\begin{document}
\chapter{Numerical Sequences and Series}
\section{Sequences}
\begin{definition}[Sequence]
  A \textit{sequence} $(x_n)_{n \in \N}$ on a set $X$ is an enumerated list (i.e. $x_1, x_2, x_3, \ldots$) where each element is in $X$.
\end{definition}
In this chapter, we will consider only sequences where $X \subseteq \R$ or $X \subseteq \C$.
A concrete and important example is $X = \R$ which are known as the \textit{real sequences}.

\subsection{Convergence and Divergence}
We would like to think about what it means for a sequence $x_n$ to converge to a point $x \in X$, and what it means for $x_n$ to diverge to infinity?

\textbf{On convergence}
\begin{itemize}
  \item We need $|x_n - x|$ to be small, more specifically, smaller than any given threshold $\varepsilon > 0$ that we choose.
  \item For this comparison, only the ``tail'' of the sequence, i.e. the ``large $n$'' behaviour, matters.
    We can always ignore the first $N$ terms of the sequence, for some $N$ dependent on $\varepsilon$.
\end{itemize}
\textbf{On divergence to infinity}
\begin{itemize}
  \item We need $|x_n|$ to clear any threshold $L > 0$ that we set for it.
  \item Again, only the ``tail'' matters.
    We can always ignore the first $N$ terms, for some $N$ dependent on $L$.
\end{itemize}
Now that we know what we what restrictions we want the definitions to impose, we can more rigorously state them:
\begin{definition}[Convergence]
  We say that the sequence $(x_n)$ \textit{converges} to some finite $x \in \C$ if:
  \[
    \forall \varepsilon > 0\ \exists N = N(\varepsilon) \text{ s.t. } |x_n - x| < \varepsilon\ \forall n \geq N
  \]
  We then write $x_n \to x$.
\end{definition}
\begin{definition}[Divergence to infinity]
  We say that the sequence $(x_n)$ \textit{diverges to infinity} if:
  \[
    \forall L > 0\ \exists N = N(L) \text{ s.t. } |x_n| > L\ \forall n \geq N
  \]
\end{definition}
\begin{remark}[Remarks]
  \begin{enumerate}
    \item We can make the modulus inequality non strict (i.e. $|x_n - x| \leq \varepsilon$ for convergence or $|x_n| \geq L$ for divergence to infinity) for an equivalent definition.
    \item We can also replace $\varepsilon$ by any constant positive multiple of $\varepsilon$ for an equivalent definition.
    \item If $(x_n)$ is real, then it diverges to infinity if either $(x_n)$ or $(-x_n)$ satisfy:
      \[
        \forall L > 0\ \exists N = N(L) \text{ s.t. } x_n > L\ \forall n \geq N
      \]
  \end{enumerate}
\end{remark}
\begin{definition}[Floor and Ceiling functions]
  The \textit{floor function} takes as input any $x \in \R$ and outputs the \textbf{greatest integer less than or equal} to $x$.
  It is denoted $\floor{x}$.

  Similarly, the \textit{ceiling function} takes as input any $x \in \R$ and outputs the \textbf{greatest integer greater than or equal} to $x$.
  It is denoted $\ceil{x}$.
\end{definition}
\begin{example}
  \begin{enumerate}
    \item $x_n = \frac{1}{n}$, then $x_n \to 0$ as $\forall \varepsilon > 0$, we have:
      \[
        |x_n - 0| = \frac{1}{n} < \varepsilon\ \forall n \geq N = N(\varepsilon) = 1 + \ceil{\frac{1}{\varepsilon}}
      \]
    \item $x_n = \frac{1}{2^{n}}$ then $x_n \to 0$ as $\forall \varepsilon > 0$, we have:
      \[
        |x_n - 0| = \frac{1}{2^{n}} < \varepsilon\ \forall n \geq N = N(\varepsilon) = \max\left\{1 + \log_2\left(\frac{1}{\varepsilon}\right), 1\right\}
      \]
      Note that we use the $\max$ to avoid issues when $\log_{2}(1/\varepsilon) < 0$.
    \item $x_n = i n$ diverges to infinity as $\forall L > 0$, we have:
      \[
        \frac{x_n}{i} = n > L\ \forall n \geq N = 1 + \ceil{L}
      \]
    \item $x_n = (-1)^{n}$ does not diverge to $\infty$ (as $|x_n| \leq 1\ \forall n$), but also does not converge (See Numbers and Sets -- Example 6.5).
  \end{enumerate}
\end{example}
\subsection{Limit Laws and Properties}
\begin{lemma}[Uniqueness of Limits]
  If it exists, the limit of $(x_n)$ is unique.
\end{lemma}
\begin{proof}
  Suppose $x_n \to a$ and $x_n \to b$.
  Take $\varepsilon > 0$, we then have:
  \begin{align*}
    \exists N_1 &= N_1(\varepsilon) \text{ s.t. } |x_n - a| < \varepsilon\ \forall n \geq N_1 \\
    \exists N_2 &= N_2(\varepsilon) \text{ s.t. } |x_n - b| < \varepsilon\ \forall n \geq N_2
  \end{align*}
  In particular $|x_n - a|, |x_n - b| < \varepsilon\ \forall n \geq N = \max\{N_1, N_2\}$, thus, using the triangle inequality:
  \[
    |a - b| = |a - x_n + x_n - b| \leq |x_n - a| + |x_n - b| < 2\varepsilon\ \forall n \geq N
  \]
  Now since $\varepsilon$ can be arbitrarily small and positive, we conclude that $|a - b| = 0$ and so $a = b$.
\end{proof}
\begin{remark}[Notation]
  Since the limit is unique, we write:
  \[
    \lim_{n \to \infty} x_n = x
  \]
  to mean the unique limit of $x_n$ as $n \to \infty$.
\end{remark}
\begin{proposition}[Sandwich Theorems]
  Let $(x_n), (y_n), (z_n)$ be real sequences.
  \label{sandwichThm}
  Then:
  \begin{enumerate}
    \item If $x_n \leq y\ \forall n$ and $x_n \to x$, then $x \leq y$
    \item If $x_n \to x$, $z_n \to x$ and $x_n \leq y_n \leq z_n\ \forall n$, then $y_n \to x$ also.
  \end{enumerate}
\end{proposition}
\begin{proof}[\textbf{i}]
  Suppose, for contradiction, that $x > y$.
  Then $x - y > 0$, so we can take $\varepsilon = x - y$.
  Thus $\exists N \text{ s.t. } |x_n - x| < x - y \ \forall n \geq N$.
  However, $x_n \leq y < x$ so:
  \[
    |x_n - x| = x - x_n < x - y \implies x_n > y\ \forall n \geq N
  \]
  which is a contradiction.
\end{proof}
\begin{proof}[\textbf{ii}]
  Take $\varepsilon > 0$, we then have:
  \begin{align*}
    \exists N_1 &= N_1(\varepsilon) \text{ s.t. } |x_n - a| < \varepsilon\ \forall n \geq N_1 \\
    \exists N_2 &= N_2(\varepsilon) \text{ s.t. } |x_n - b| < \varepsilon\ \forall n \geq N_2
  \end{align*}
  We can then write:
  \begin{align*}
    |y_n - x| &= \frac{1}{2}|(x_n - x) + (z_n - x) + (y_n - x_n) + (y_n - z_n)| \\
              &< \frac{1}{2}(2\varepsilon + |y_n - x_n| + |y_n - z_n|) \\
              &= \frac{1}{2}(2\varepsilon + y_n - x_n + z_n - y_n) \\
              &= \varepsilon + \frac{1}{2}(z_n - x_n) \\
              &= \varepsilon + \frac{1}{2}|z_n - x_n| \\
              &\leq \varepsilon + \frac{1}{2}(|z_n - x| + |x_n - x|) \\
              &< 2\varepsilon
  \end{align*}
  So $y_n \to x$.
\end{proof}
\begin{remark}
  Note that $x_n < y\ \forall n$ and $x_n \to x$ does not necessarily imply $x < y$.

  For example, $x_n = 1 - \frac{1}{n} < 1\ \forall n$ but $x_n \to 1$.
\end{remark}
\begin{lemma}
  Let $(x_n)$ be a complex sequence.
  Then $x_n \to x$ if and only if $\Re(x_n) \to \Re(x)$ and $\Im(x_n) \to \Im(x)$.
  \label{complexConvergence}
\end{lemma}
\begin{proof}
  Recall that for $z \in \C$, $|z| = \sqrt{(\Re(z))^2 + (\Im(z))^2}$.
  \begin{proofdirection}{Assume $x_n \to x$}
    If we fix $\varepsilon > 0$, by definition of convergence:
    \[
      \exists N \text{ s.t } |x_n - x| < \varepsilon\ \forall n \geq N
    \]
    We see that $|\Re(z)| \leq |z|$ and $|\Im(z)| \leq |z|$.
    Therefore for all $n \geq N$:
    \begin{align*}
      |\Re(x_n) - \Re(x)| &= |\Re(x_n - x)| \leq |x_n - x| < \varepsilon \\
      |\Im(x_n) - \Im(x)| &= |\Im(x_n - x)| \leq |x_n - x| < \varepsilon
    \end{align*}
    So $\Re(x_n) \to \Re(x)$ and $\Im(x_n) \to \Im(x)$.
  \end{proofdirection}
  \begin{proofdirection}{Assume $\Re(x_n) \to \Re(x)$ and $\Im(x_n) \to \Im(x)$}
    If we fix $\varepsilon > 0$, by definition of convergence:
    \begin{align*}
      \exists N_1 &= N_1(\varepsilon) \text{ s.t. } |\Re(x_n) - \Re(x)| < \varepsilon\ \forall n \geq N_1 \\
      \exists N_2 &= N_2(\varepsilon) \text{ s.t. } |\Im(x_n) - \Im(x)| < \varepsilon\ \forall n \geq N_2
    \end{align*}
    We see that:
    \[
      |z| = |\Re(z) + i \Im(z)| \leq |\Re(z)| + |i||\Im(z)| = |\Re(z)| + |\Im(z)|
    \]
    and so:
    \[
      |x_n - x| \leq |\Re(x_n) - \Re(x)| + |\Im(x_n) - \Im(x)| < 2\varepsilon\ \forall n \geq \max\{N_1, N_2\}
    \]
  \end{proofdirection}
\end{proof}
\begin{definition}[Bounded Sequence]
  We say that $(x_n)$ is bounded if $\exists M > 0$ such that $|x_n| \leq M$ or equivalently $\sup\limits_{n \geq 1} |x_n| \leq M$.
\end{definition}
\begin{lemma}[Boundedness of Convergent Sequences]
  If $x_n \to x$ then $(x_n)$ must be bounded.
  That is $\exists M$ s.t. $|x_n| \leq M\ \forall n$.
  \label{convergenceBounded}
\end{lemma}
\begin{proof}
  Take $\varepsilon = 1$, then $\exists N \text{ s.t. } |x_n - x| \leq 1\ \forall n \geq N$ and thus by the triangle inequality:
  \[
    |x_n|  = |x_n - x + x| \leq |x_n - x| + |x| \leq 1 + |x|\ \forall n \geq N
  \]
  For $n < N$, $|x_n| \leq \max\{|x_1|, \ldots, |x_{N - 1}|\} < \infty$

  Therefore the whole sequence is bounded by:
  \[
    |x_n| \leq \max\{|x_1|, \ldots, |x_{N - 1}|, 1 + |x|\}\ \forall n
  \]
\end{proof}
\begin{lemma}[Addition and Multiplication of Limits]
  If $x_n \to x$ and $y_n \to y$, then:
  \label{limitLaws}
  \begin{enumerate}
    \item $x_n + y_n \to x + y$
    \item $x_n y_n \to xy$
  \end{enumerate}
\end{lemma}
\begin{proof}
  Since $x_n \to x$ and $y_n \to y$, $\forall \varepsilon > 0\ \exists N_1 = N_1 (\varepsilon),\ N_2 = N_2(\varepsilon)$ such that:
  \[
    |x_n - x| < \varepsilon\ \forall n \geq N_1 \text{ and } |y_n - y| < \varepsilon\ \forall n \geq N_2
  \]

  \textbf{Addition of Limits (i):}
  \[
    |x_n + y_n - (x + y)| \leq |x_n - x| + |y_n - y| < 2\varepsilon\ \forall n \geq \max\{N_1, N_2\}
  \]
  which is a constant multiple of $\varepsilon$, so we are done.

  \textbf{Multiplication of Limits (ii):}
  \begin{align*}
    |x_n y_n - xy| &= |x_n y_n - xy_n + xy_n - xy| \\
                   &= |x| |y_n - y| + |y_n||x - x_n|
  \end{align*}
  Hence $|x_n y_n - xy| \leq \varepsilon(|x| + |y_n|)\ \forall n \geq \max\{N_1, N_2\}$.
  By \cref{convergenceBounded}, for some $M \geq 0$, we have:
  \[
    |x_ny_n - xy| \leq \varepsilon(|x| + |y_n|) \leq \varepsilon(|x| + M)\ \forall n \geq \max\{N_1, N_2\}
  \]
  which is a constant multiple of $\varepsilon$, so we are done.
\end{proof}
\begin{lemma}[Convergence of Reciprocal Sequence]
  If $x_n \neq 0\ \forall n$ and $x_n \to x \neq 0$, then $\frac{1}{x_n} \to \frac{1}{x}$
\end{lemma}
\begin{proof}
  Since $x_n \to x$ and $x \neq 0$, we can take $\varepsilon = |x/2|$ so $\exists N$ s.t. $|x_n - x| < |x/2|\ \forall n \geq N$.
  Using the triangle inequality, for such $n$ we have:
  \[
    |x_n - (x_n - x)| \leq |x_n| + |x_n - x| \implies |x_n| \geq |x| - |x_n - x| > |x| - \abs{\frac{x}{2}} = \abs{\frac{x}{2}}
  \]
  Since there are finitely many terms with index $< N$, we can construct the bound:
  \[
    |x_n| \geq \min\{|x_1|, \ldots, |x_{N - 1}|, |x/2|\}\ \forall n
  \]
  So $|x_n| \geq M\ \forall n$ for some $M > 0$.

  Now for any $\varepsilon > 0$, $\exists N(\varepsilon)$ s.t. $\forall n > N$ we have the following:
  \begin{align*}
    \abs{\frac{1}{x_n} - \frac{1}{x}} &= \abs{\frac{x_n - x}{x_n x}} \\
                                      &< \frac{\varepsilon}{|x_n||x|} \\
                                      &\leq \frac{\varepsilon}{M|x|}
  \end{align*}
  which is a constant multiple of $\varepsilon$, so we are done.
\end{proof}
\begin{remark}
  This proof also shows that if $x_n \neq 0\ \forall n$ and $x_n \to x \neq 0$, then $\exists M > 0$ such that $|x_n| \geq M\ \forall n$.
\end{remark}
\begin{remark}[Advice]
  We often specify that something must be true for all terms of the sequence, however if it is not, we can just define a new sequence that removes terms from the start of the sequence so that the property is satisfied for all $n$.
\end{remark}
\subsection{Monotone Sequences}
\begin{definition}[Monotone Sequence]
  We say a real sequence $(x_n)$ is \textit{monotone} if either:
  \begin{itemize}
    \item It is \textit{increasing} $x_n \leq x_{n + 1}\ \forall n$ (strictly increasing if $<$)
    \item It is \textit{decreasing} $x_n \geq x_{n + 1}\ \forall n$ (strictly decreasing if $>$)
  \end{itemize}
\end{definition}
\begin{proposition}[Monotone Convergence Theorem]
  Every bounded monotone real sequence converges.
  \label{monotoneConvergence}
\end{proposition}
\begin{proof}
  Suppose $(x_n)$ is increasing.
  Then the set $\{x_n: n \geq 1\}$ is a non-empty set which is bounded above as $(x_n)$ is bounded.
  By the least upper bound axiom, it has a supremum, say $\ell$.

  Given $\varepsilon > 0$, $\ell - \varepsilon$ cannot be an upper bound for the set because $\ell$ is the least upper bound.
  Therefore $\exists N$ s.t. $x_N > \ell - \varepsilon$.
  Thus, since $(x_n)$ is monotonic $\ell - \varepsilon < x_n \leq \ell$ for $\forall n \geq N$.
  Hence $\forall n \geq N$, $|x_n -\ell| < \varepsilon$ so $x_n \to \ell$.

  The decreasing case is the same but instead considering $(-x_n)$.
\end{proof}
\begin{remark}
  We only required the sequence to be bounded above if it is monotonically increasing and bounded below if it is monotonically decreasing.
\end{remark}
Sequences that are bounded but not monotone do not necessarily have to converge, for example the sequence $x_n = (-1)^{n}$ is bounded but does not converge.
However, if we remove even terms we get the subsequence $x_{2k + 1} = -1$, and if we remove all the odd terms, we get the subsequence $x_{2k} = 1$.
These are constant sequences so converge.
\section{Bolzano-Weierstrass Theorem}
To prove the Bolzano-Weierstrass Theorem, we first need to rigorously introduce the notion of a subsequence and prove a few preliminary lemmas.
\begin{definition}[Subsequence]
  Given a sequence $(x_n)$, a \textit{subsequence} of $(x_n)$ is a sequence of the form $(x_{n_k})_{k \in \N}$ where $(n_k)_{k \in \N}$ is a sequence of naturals satisfying $n_1 < n_2 < n_3 < \cdots$ (strictly monotonically increasing).
\end{definition}
\begin{lemma}[Convergence of Subsequences]
  If $x_n \to x$ then any subsequence $(x_{n_k})_{k \in \N}$ must converge to the same limit.
\end{lemma}
\begin{proof}
  Since $n_k < n_{k + 1} \implies n_{k + 1} \geq n_k + 1$, by induction $n_k \geq k$.
  If we take $\varepsilon > 0$, then $\exists N = N (\varepsilon) \text{ s.t. } |x_n - x| < \varepsilon\ \forall n \geq N$.
  So if $k \geq N$, then $n_k \geq k \geq N$ so $|x_{n_k} - x| < \varepsilon$.
  Hence $\lim\limits_{k \to \infty} x_{n_k} = x$.
\end{proof}
\begin{proposition}[Nested Interval Property]
  Let $(I_n)_{n \in \N}$ be a sequence of nested intervals in $\R$, that is, $\forall n\ I_{n + 1} \subseteq I_n$ and $I_n = [a_n, b_n]$ for $a_n \leq b_n$.
  \label{nestedInterval}

  If $b_n - a_n = |I_n| \to 0$ as $n \to \infty$ then $\bigcap_{n \in \N} I_n$ contains exactly one point.
\end{proposition}
\begin{proof}
  Since $I_{n + 1} \subseteq I_n$ and $a_{n + 1} \geq a_n$ and $b_{n + 1} \leq b_n$.
  Therefore $a_1 \leq a_n \leq b_n \leq b_1\ \forall n$.

  So the sequence $(a_n)$ is increasing and bounded above by $b_1$ and the sequence $(b_n)$ is decreasing and bounded below by $a_1$.
  Thus, by \cref{monotoneConvergence}, $a_n \to a$ and $b_n \to b$ for some $a, b \in \R$.
  Since $a_n \leq b_n \forall n$ we can use \cref{sandwichThm} to preserve the non-strict inequality to yield, $a \leq b$.

  \textbf{Existence}\par
  For all $k \geq n$, $a_k \in [a_k, b_k] \subseteq [a_n, b_n]$ so $a_n \leq a_k \leq b_n$.
  For fixed $n$, as $k \geq n$, we can take the limit as $k \to \infty$ so using \cref{sandwichThm} again $a_n \leq a \leq b_n$.
  Thus $a \in I_n\ \forall n$ so $a \in \bigcap_{n \in \N} I_n$.

  \textbf{Uniqueness}\par
  By \cref{limitLaws}, $b_n - a_n \to b - a$.
  By assumption, $b_n - a_n \to 0$ and limits are unique, therefore $b - a = 0 \implies b = a$.

  Hence $x \in \bigcap_{n} I_n \iff x \in I_n\ \forall n \iff a_n \leq x \leq b_n\ \forall n$.
  Using \cref{sandwichThm} again, $a \leq x \leq b = a \implies x = a$.
  So the intersection contains only $a$.
\end{proof}
\begin{theorem}[Bolzano-Weierstrass Theorem]
  If $(x_n)$ is a real and bounded sequence, then it has at least one convergence subsequence.
\end{theorem}
\begin{proof}
  Suppose $(x_n)$ is bounded by $M \geq 0$.
  We wish to construct a sequence of nested intervals from which we can sample our subsequence since that will ensure that our subsequence will converge  to the unique intersection point of the nested intervals.

  Let $a_1 = -M$ and $b_1 = M$ so $I_1 = [-M, M] \supset \{x_n: n \in \N\}$ as $|x_n| \leq M\ \forall n$.
  If we let $c = \frac{a_1 + b_1}{2}$, then at least one of the halves $[a_1, c]$ and $[c, b_1]$ will have infinitely many elements of $(x_n)$.
  Take $I_2$ to be a half that has infinitely many elements.
  Continue this process inductively to get $(I_n)_{n \in \N}$ which are nested by construction and all have infinitely many elements of the sequence.
  By \cref{nestedInterval}, $\exists!\ x$ s.t. $x \in \bigcap_{n} I_n$.

  We then choose $(x_{n_k})$ as follows:
  \begin{itemize}
    \item Since $I_1$ has infinitely many elements, we can pick $n_1$ s.t. $x_{n_1} \in I_1$.
    \item By construction, $I_2$ has infinitely many elements of $(x_n)$ of index $> n_1$ so we can pick $n_2 > n_1$ s.t $x_{n_2} \in I_2 \subset I_1$.
    \item Continue this process inductively to get $(x_{n_k})_{k \in \N}$.
  \end{itemize}
  We have $x_{n_k} \in I_k\ \forall k \implies x_{n_k} \in \bigcap_{n \leq k} I_n$.
  Therefore, as $k \to \infty$, the intersection becomes $\{x\}$ so $x_{n_k} \to x$.
\end{proof}
\begin{remark}[Notation]
  We write $\exists!$ to mean ``there exists unique'' and $\centernot \exists$ to mean ``there does not exist''.
\end{remark}
\begin{remark}[Remarks]
  \begin{enumerate}
    \item We see from $x_n = (-1)^{n}$ that there can be more than one convergent subsequence and their limits can disagree.
    \item The Bolzano-Weierstrass theorem also holds for complex sequences, see Q1b on Example Sheet 1.
  \end{enumerate}
\end{remark}
\section{Cauchy Sequences}
A Cauchy sequence is a sequence where the elements get closer together as the sequence progresses.
\begin{definition}[Cauchy Sequence]
  We say a sequence $(x_n)$ on $X \subseteq \C$ is \textit{Cauchy} if:
  \label{cauchyDef}
  \[
    \forall \varepsilon > 0\ \exists N = N(\varepsilon) \text{ s.t. } |x_n - x_m| < \varepsilon\ \forall n, m \geq N
  \]
\end{definition}
\begin{remark}[Remarks]
  \begin{itemize}
    \item Note that the definition compares two elements that are arbitrarily far apart in the tail of the sequence, \textbf{not} just consecutive elements.

      In fact, $|x_n - x_{n + 1}| \to 0$ as $n \to \infty$ is \textbf{not} enough to conclude that $(x_n)$ is Cauchy (See Q1c Example Sheet 1).
    \item Similarly to the convergence definition, changing $\varepsilon$ to any positive multiple of $\varepsilon$ gives an equivalent definition.
  \end{itemize}
\end{remark}
\begin{example}[Cauchy Sequences]
  \label{cauchyExample}
  \begin{enumerate}
    \item $x_n = \frac{1}{n}$, assume WLOG $m \geq n$.
      Then $\forall \varepsilon > 0$:
      \[
        \abs{\frac{1}{m} - \frac{1}{n}} = \frac{1}{n} - \frac{1}{m} = \left(1 - \frac{n}{m}\right) \frac{1}{n} \leq \frac{1}{n} < \varepsilon\ \forall n \geq N(\varepsilon) = \ceil{\frac{1}{\varepsilon}} + 1
      \]
      Since $m \geq n$, the above is true $\forall m, n \geq N(\varepsilon)$ so it is Cauchy.
    \item $x_n = (-1)^{n}$ is not Cauchy.
      If we take $n = 2k, m = 2k + 1$ for any $k \in \N$ then $|x_n - x_m| = 2$.
      This violates the definition when $\varepsilon = 1$ so it cannot be Cauchy.
    \item $(x_n)$ on $\Q$ defined by the truncation of the decimal expansion of $\sqrt{2}$, that is:
      \[
        x_1 = 1,\ x_2 = 1.4,\ x_3 = 1.41,\ x_4 = 1.414, \ldots
      \]
      This sequence is Cauchy.

      Again, WLOG, assume $m \geq n$, then $|x_m - x_n| < 10^{-(n - 1)}$.
      Note that $10^{-(n - 1)} \to 0 \text{ as } n \to \infty$, so for any fixed $\varepsilon > 0$, $\exists N$ s.t. $10^{-(n - 1)} < \varepsilon \forall n\ \geq N$.
      For such $n$, $n, m \geq N$ and $|x_m - x_n| < 10^{- (n - 1)} < \varepsilon$ so $(x_n)$ is Cauchy.

      However, this sequence does not converge over $\Q$, but it converges over $\R$ to $\sqrt{2}$).
  \end{enumerate}
\end{example}
\begin{remark}[Advice]
  In example \textbf{iii} above, we saw that we can show a sequence is Cauchy by showing that for $m \geq n$, $|x_m - x_n|$ is bounded by some positive real sequence $(a_n)$ that converges to 0 as $n \to \infty$.
\end{remark}
\begin{lemma}
  If $(x_n)$ is Cauchy, then it is bounded.
\end{lemma}
\begin{proof}
  Take \cref{cauchyDef} with $\varepsilon = 1$.
  Then, $\exists N$ s.t. $|x_n - x_N| < 1\ \forall n \geq N$, by fixing $m = N$.
  For such $n$, $|x_n| < 1 + |x_N|$ which is a finite number independent of $n$.
  Therefore:
  \[
    \sup\limits_{n \geq 1} |x_n| \leq \max\{|x_1|, \ldots, |x_{N - 1}|, 1 + |x_N|\}
  \]
  so $(x_n)$ is bounded.
\end{proof}
\begin{lemma}
  A complex sequence $(x_n)$ is Cauchy if and only if the real sequences $(\Re(x_n))$ and $(\Im(x_n))$ are Cauchy.
  \label{complexCauchy}
\end{lemma}
\begin{proof}
  This proof is very similar to the proof of \cref{complexConvergence}.
  \begin{proofdirection}{Assume $(x_n)$ is Cauchy}
    If we fix $\varepsilon > 0$, by definition of Cauchy sequences $\exists N \text{ s.t } |x_n - x_m| < \varepsilon\ \forall n, m \geq N$.
    Note that $|\Re(z)| \leq |z|$ and $|\Im(z)| \leq |z|$.
    Therefore for all $n, m \geq N$:
    \begin{align*}
      |\Re(x_n) - \Re(x_m)| &= |\Re(x_n - x_m)| \leq |x_n - x_m| < \varepsilon \\
      |\Im(x_n) - \Im(x_m)| &= |\Im(x_n - x_m)| \leq |x_n - x_m| < \varepsilon
    \end{align*}
    So $(\Re(x_n))$ and $(\Im(x_n))$ are Cauchy.
  \end{proofdirection}
  \begin{proofdirection}{Assume $(\Re(x_n))$ and $(\Im(x_n))$ are Cauchy}
    If we fix $\varepsilon > 0$, again, by definition:
    \begin{align*}
      \exists N_1 &= N_1(\varepsilon) \text{ s.t. } |\Re(x_n) - \Re(x_m)| < \varepsilon\ \forall n, m \geq N_1 \\
      \exists N_2 &= N_2(\varepsilon) \text{ s.t. } |\Im(x_n) - \Im(x_n)| < \varepsilon\ \forall n, m \geq N_2
    \end{align*}
    Note that $|z| \leq |\Re(z)| + |\Im(z)|$ so:
    \[
      |x_n - x_m| \leq |\Re(x_n) - \Re(x_m)| + |\Im(x_n) - \Im(x_m)| < 2\varepsilon\ \forall n, m \geq \max\{N_1, N_2\}
    \]
    which is a constant multiple of $\varepsilon$ so $(x_n)$ is Cauchy.
  \end{proofdirection}
\end{proof}
\begin{lemma}
  If $x_n \to x$, then $(x_n)$ is Cauchy.
\end{lemma}
\begin{proof}
  For fixed $\varepsilon > 0$ we have:
  \begin{align*}
    |x_n - x_m| &= |x_n - x + x - x_m| \\
                &\leq |x_n  - x| + |x_m - x| \\
                &< 2\varepsilon\ \forall n, m \geq N \text{ because $x_n \to x$}
  \end{align*}
  which is a constant multiple of $\varepsilon$ so $(x_n)$ is Cauchy.
\end{proof}
The \cref{cauchyExample} \textbf{iii} shows that there exists Cauchy sequences on $\Q$ that do not converge on $\Q$.
So, if the converse of the above is true, it can only be true for certain ``nice'' subsets $X \subseteq \C$.
\begin{theorem}[Completeness of $\R$ and $\C$]
  Every cauchy sequence on $\R$ or $\C$ is convergent.
\end{theorem}
\begin{proof}
  We have seen that if $(x_n)$ is Cauchy, then it is bounded.
  So by the Bolzano Weierstrass Theorem (which also works for sequences on $\C$), we can find a subsequence $(x_{n_k})$ which converges to some $x \in \R$.
  \begin{align*}
    |x_n - x| &= |x_n - x_{n_k} + x_{n_k} - x| \\
              &\leq \underbrace{|x_n - x_{n_k}|}_{\text{use Cauchy}} + \underbrace{|x_{n_k} - x|}_{\text{use convergence}}
  \end{align*}
  Take an $\varepsilon > 0$.
  Since $(x_n)$ is Cauchy:
  \[
    \exists N_1 = N_1(\varepsilon) \text{ s.t. } |x_n - x_{n_k}| < \varepsilon\ \forall n, n_k \geq N_1
  \]
  Since $(x_{n_k})$ converges:
  \[
    \exists N_2 =  N_2(\varepsilon) \text{ s.t. } |x_{n_k} - x| < \varepsilon\ \forall k \geq N_2
  \]
  As $(x_{n_k})$ is a subsequence, $(n_k)$ must be a strictly monotonically increasing sequence on $\N$.
  Therefore, we can always find a $k \geq N_2$ s.t. $n_k \geq N_1$ and thus:
  \[
    |x_n - x| < 2\varepsilon\ \forall n \geq N_1
  \]
  So $x_n \to x$.
\end{proof}
\begin{remark}[Alternative Proof]
  We can do this without having to use the Bolzano-Weierstrass Theorem on $\C$ and just use the version on $\R$.

  Recall that a sequence on $\C$ is convergent if and only if its real and imaginary parts are (\cref{complexConvergence}), and similarly, a sequence on $\C$ is Cauchy if and only if its imaginary parts are (\cref{complexCauchy}).

  We can then carry out the same proof using a real sequence, since, if a sequence on $\C$ is Cauchy, its real and imaginary parts are cauchy, so converge, and so the original sequence must converge.
\end{remark}
\begin{remark}[Remarks]
  \begin{itemize}
    \item We can use this to prove convergence of $\R$ or $\C$ sequences without having to guess what the limit is as we can just show that they are convergent.
    \item Next year, in Analysis II, we will define ``completeness'' and we will see that $\Q$ is not complete, which matches with the discussed example.
  \end{itemize}
\end{remark}
\section{Series and Convergence}
\begin{definition}[Series and Convergence of Series]
  Let $(a_n)_{n \in \N}$ be a sequence on $\R$ or $\C$.
  We say that $\sum_{n=1}^{\infty} a_n$ is a \textit{series}.

  We say a series converges if the sequence of \textit{partial sums} $(s_k)_{k \in \N}$, defined by $s_k = \sum_{n = 1}^{k} a_n$, converges to some finite $s \in \R$ or $\C$ as $k \to \infty$.
  In this case, $s$ is called the sum of the series and we write $s = \sum_{n = 1}^{\infty} a_n$.
\end{definition}
\begin{example}[Examples of Series]
  \begin{enumerate}
    \item The series $\sum\limits_{n = 1}^{\infty} n$ does not converge as its partial sums diverge:
      \[
        s_k = \sum_{n = 1}^{k} n = \frac{1}{2}k(k + 1) \to \infty \text{ as } k \to \infty
      \]
    \item The geometric series is $\sum\limits_{n = 1}^{\infty} r^{n}$.
      It converges if and only if $|r| < 1$, since its partial sums are:
      \[
        s_k = \begin{cases}
        \frac{1 - r^{k}}{1 - r} & \text{ if } r\neq1 \\
        k & \text{ if } r = 1
        \end{cases}
      \]
      and $r^{k}$ converges if and only if $|r| < 1$.
  \end{enumerate}
\end{example}
\end{document}
