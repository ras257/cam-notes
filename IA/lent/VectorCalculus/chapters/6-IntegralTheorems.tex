\documentclass[../main.tex]{subfiles}
\begin{document}
\chapter{Integral Theorems}
\section{Green's Theorem in the Plane}
\begin{theorem}[Green's Theorem]
  Given an area $A \subset \R^2$ with piecewise smooth boundary $C = \partial A$ and continuously differentiable functions $P(x, y)$ and $Q(x, y)$, then:
  \[
    \int_{A} \left(\pderiv{Q}{x} - \pderiv{P}{y}\right) \d{x} \d{y} = \oint_{C} (P \d{x} + Q \d{y})
  \]
  where $C$ is traversed in the ``positive sense''.

  The ``positive sense'' usually means anti-clockwise, however there are some edge cases involving multiple boundary curves that will be discussed later.
\end{theorem}
\begin{proof}
  Initially, we assume that $A$ is a simple shape.
  This means that for each $y$ there is a single range of values of $x$, namely $x^{-}(y) \leq x \leq x^{+}(y)$.
  Likewise, for each $x$ there is a range $y^{-1}(x) \leq y \leq y^{+}(x)$.
  The diagram below shows an example $y$ and its associated range of values of $x$.
  \begin{center}
  \begin{tikzpicture}[>=stealth]
    \draw[thick, fill=gray!50, postaction={decorate}, decoration={
          markings,
          mark = between positions 0 and 1 step 0.2 with {\arrow{<}}
        }] (-1,1) to[closed, curve through = { (-0.5, 1.3) (0, 1.8) (1, 1.9) (2.5, 0) (1.2, -1.9) (0, -1.6) }] (-1, -1);
    \draw (0, -2) -- (0, 2);
    \draw (-2, 0) -- (3, 0);
    \draw[dotted, thick] (-2, -1.9) -- (3, -1.9) node[right, scale=0.8] {$y_{\text{min}}$};
    \draw[dotted, thick] (-2, 1.97) -- (3, 1.97) node[right, scale=0.8] {$y_{\text{max}}$};
    \filldraw[fill=white, draw=black] (0.55, 1.97) circle (1.5pt) node[above] {$P_1$};
    \filldraw[fill=white, draw=black] (1.1, -1.9) circle (1.5pt) node[below] {$P_2$};
    \node at (2.5, 1.3) {$C^{+}$};
    \node at (-1.3, 1.2) {$C^{-}$};
    \node at (0.6, 0.5) {$A$};

    \draw[dashed] (-0.65, 1.2) -- (2.02, 1.2);
    \fill (0, 1.2) circle (1.5pt) node[below right, scale=0.8] {$y$};
    \draw[dashed] (-0.65, 1.2) -- (-0.65, 0) node[below, scale=0.8] {$x^{-}(y)$};
    \draw[dashed] (2.02, 1.2) -- (2.02, 0) node[below, scale=0.8] {$x^{+}(y)$};
  \end{tikzpicture}
  \end{center}
  We can then divide $C$ into a right hand part $C^{+}$, on which $x = x^{+}(y)$ for each $y$ in the range $y_{\text{min}} \leq y \leq y_{\text{max}}$ and left hand part $C^{-}$ on which $x = x^{-}(y)$.
  In the above diagram, $C^{+}$ goes from $P_2 \to P_1$ and $C^{-}$ goes from $P_1 \to P_2$.

  We can now evaluate the area integral of $\pderiv{Q}{x}$:
  \begin{align*}
    \int_{A} \pderiv{Q}{x} \d{x} \d{y} &= \int_{y_{\text{min}}}^{y_{\text{max}}} \left(\int_{x^{-}(y)}^{x^{+}(y)} \pderiv{Q}{x} \d{x}\right) \d{y} \text{ by definition of area integrals} \\
                                       &= \int_{y_{\text{min}}}^{y_{\text{max}}} [Q(x^{+}(y), y) - Q(x^{-}(y), y)] \d{y} \text{ using FTC} \\
                                       &= \int_{y_{\text{min}}}^{y_{\text{max}}} Q(x^{+}(y), y) \d{y} + \int_{y_{\text{max}}}^{y_{\text{min}}} Q(x^{-}(y), y) \d{y} \\
                                       &= \int_{C^{+}} Q(x, y) \d{y} + \int_{C^{-}} Q(x, y) \d{y} \\
                                       &= \oint Q \d{y}
  \end{align*}
  Similarly, for $\pderiv{P}{y}$:
  \begin{align*}
    \int_{A} \pderiv{P}{y} \d{y} \d{x} &= \int_{x_{\text{min}}}^{x_{\text{max}}} \left(\int_{y^{-}(x)}^{y^{+}(x)} \pderiv{P}{y} \d{y}\right) \d{x} \\
                                       &= \int_{x_{\text{min}}}^{x_{\text{max}}} [P(x, y^{+}(x)) - P(x, y^{-}(x))] \d{x} \text{ using FTC} \\
                                       &= -\int_{x_{\text{max}}}^{x_{\text{min}}} P(x, y^{-}(x)) \d{x} - \int_{x_{\text{min}}}^{x_{\text{max}}} P(x, y^{+}(x)) \d{x} \\
                                       &= -\int_{C^{+}} P(x, y) \d{x} - \int_{C^{-}} P(x, y) \d{x} \\
                                       &= -\oint P \d{x}
  \end{align*}
  So for simple shapes, we are done.

  If $A$ is a more complicated shape (i.e. multiple ranges of $x$ for each $y$ and/or vice versa), then we can split it up into multiple areas $A_i$, each of which is a simple shape, and apply the calculation above to each part to obtain the same result.
  The details of this will be explained below.
\end{proof}
\begin{remark}[Result for non-simple shapes]
  \nonexaminable
  If we have split $A_i$ into disjoint areas then:
  \[
    \int_{A} \pderiv{Q}{x} \d{x} \d{y} = \sum_{i} \int_{A_i} \pderiv{Q}{x} \d{x} \d{y}
  \]
  Furthermore:
  \[
    \oint_{\partial A} Q \d{y} = \sum_{i} \oint_{\partial A_i} Q \d{y}
  \]
  Although each $\partial A_i$ contains a segment not part of the full $\partial A$, these parts cancel out as we have the same segment traversed in different directions in different parts of the shape.

  For example, the following shape has been split into two simple shapes that are infinitesimally close:
  \begin{center}
  \begin{tikzpicture}[>=stealth]
    \draw[thick, fill=gray!50, postaction={decorate}, decoration={
          markings,
          mark = between positions 0.2 and 0.8 step 0.2 with {\arrow{<}}
        }] (0.1, -0.1) to[curve through = { (-0.2, -0.5) (-1, -1) (-1.5, 0) (-0.5, 1.3) (0, 1.8) }] (0.5, 1.9);
    \draw[thick, postaction={decorate}, decoration={
          markings,
          mark = between positions 0.2 and 0.8 step 0.3 with {\arrow{>}}
        }] (0.1, -0.1) -- (0.5, 1.9);

    \begin{scope}[xscale=-1, rotate=23, xshift=-0.4cm, yshift=0.08cm]
      \draw[thick, fill=gray!50, postaction={decorate}, decoration={
            markings,
            mark = between positions 0.2 and 0.8 step 0.2 with {\arrow{>}}
          }] (0.1, -0.1) to[curve through = { (-0.2, -0.5) (-1, -1) (-1.5, 0) (-0.5, 1.3) (0, 1.8) }] (0.5, 1.9);
      \draw[thick, postaction={decorate}, decoration={
            markings,
            mark = between positions 0.2 and 0.8 step 0.3 with {\arrow{<}}
          }] (0.1, -0.1) -- (0.5, 1.9);
    \end{scope}
    \draw (0, -2) -- (0, 2);
    \draw (-2, 0) -- (3, 0);
    \node at (0, 0.9) {$L_1$};
    \node at (-0.9, -0.3) {$A_1$};
    \node at (0.75, 0.8) {$L_2$};
    \node at (0.9, -0.6) {$A_2$};
    \node[scale=0.5] at (0.8, 2.1) {Infinitesimally close};
  \end{tikzpicture}
  \end{center}
  So the sum of the integrals around the boundary of each half is:
  \begin{align*}
    \oint_{\partial A_1} Q \d{y} + \oint_{\partial A_2} Q \d{y} &= \oint_{\partial A} Q \d{y} + \int_{L_1} Q \d{y} + \int_{L_2} Q \d{y} \\
                                                                &= \oint_{\partial A} Q \d{y} + \int_{L_1} Q \d{y} - \int_{L_1} Q \d{y} \\
                                                                &= \oint_{\partial A} Q \d{y}
  \end{align*}
  We can split up $\int_{A} \pderiv{P}{y} \d{x} \d{y}$ using a similar technique and so the result follows for non-simple shapes too.
\end{remark}
\end{document}
