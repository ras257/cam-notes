\documentclass[../main.tex]{subfiles}
\begin{document}
\chapter{Vector Differential Operators}
An operator is a map between vector spaces.
\section{Del}
\begin{definition}
  The fundamental differential operator is \textit{del}:
  \[
    \nabla = \begin{pmatrix}
    \partial/\partial x_1 \\
    \partial/\partial x_2 \\
    \partial/\partial x_3 \\
    \end{pmatrix}
  \]
\end{definition}
This operator is applied in various ways to scalar and vector fields.
In this context, a \textit{field} is simply a quantity that depends on $\vec{x}$.
\begin{remark}
  Del is often called ``grad'' although this is not strictly correct as grad is a specific operator derived from del.
\end{remark}
We can also write:
\[
  \nabla = \vec{e}_1 \pderiv{}{x} + \vec{e}_2 \pderiv{}{y} + \vec{e}_3 \pderiv{}{z} = \vec{e}_i \pderiv{}{x_i}
\]
where $\{\vec{e}_1, \vec{e}_2, \vec{e}_3\}$ is the standard basis for $\R^{3}$.
\section{Gradient}
\begin{definition}[Gradient]
  When del is applied to a scalar field $f(\vec{x})$, it is known as the \textit{gradient operator}:
  \[
    \nabla f = \begin{pmatrix}
    \partial f/\partial x \\
    \partial f/\partial y \\
    \partial f/\partial z \\
    \end{pmatrix}
  \]
  or in suffix notation:
  \[
    (\nabla f)_i = \pderiv{f}{x_i},\ \nabla f = \pderiv{f}{x_i}\vec{e}_i
  \]
\end{definition}
\begin{example}
  \begin{enumerate}
    \item If $f(\vec{x}) = |\vec{x}| = x^2 + y^2 + z^2$, then $\nabla f = (2x, 2y, 2z) = 2\vec{x}$.

      Alternatively, we can use suffix notation and summation convention:
      \[
        \pderiv{}{x_i}(x_j x_j) = 2x_j \pderiv{x_j}{x_i} = 2x_j \delta_{i j} = 2x_i
      \]
  \end{enumerate}
\end{example}
\end{document}
