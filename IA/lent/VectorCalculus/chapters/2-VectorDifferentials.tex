\documentclass[../main.tex]{subfiles}
\begin{document}
\chapter{Vector Differential Operators}
An operator is a map between vector spaces.
\section{Del}
\begin{definition}
  The fundamental differential operator is \textit{del}:
  \[
    \nabla = \begin{pmatrix}
    \partial/\partial x_1 \\
    \partial/\partial x_2 \\
    \partial/\partial x_3 \\
    \end{pmatrix}
  \]
\end{definition}
This operator is applied in various ways to scalar and vector fields.
In this context, a \textit{field} is simply a quantity that depends on $\vec{x}$.
\begin{remark}
  Del is often called ``grad'' although this is not strictly correct as grad is a specific operator derived from del.
\end{remark}
We can also write:
\[
  \nabla = \vec{e}_1 \pderiv{}{x} + \vec{e}_2 \pderiv{}{y} + \vec{e}_3 \pderiv{}{z} = \vec{e}_i \pderiv{}{x_i}
\]
where $\{\vec{e}_1, \vec{e}_2, \vec{e}_3\}$ is the standard basis for $\R^{3}$.
\section{Gradient}
\begin{definition}[Gradient]
  When del is applied to a scalar field $f(\vec{x})$, it is known as the \textit{gradient operator}:
  \[
    \nabla f = \begin{pmatrix}
    \partial f/\partial x \\
    \partial f/\partial y \\
    \partial f/\partial z \\
    \end{pmatrix}
  \]
  or in suffix notation:
  \[
    (\nabla f)_i = \pderiv{f}{x_i},\ \nabla f = \pderiv{f}{x_i}\vec{e}_i
  \]
\end{definition}
\begin{example}[Using Grad]
  \label{gradExample}
  \begin{enumerate}
    \item If $f(\vec{x}) = |\vec{x}| = x^2 + y^2 + z^2$, then $\nabla f = (2x, 2y, 2z) = 2\vec{x}$.

      Alternatively, we can use suffix notation and summation convention:
      \[
        \pderiv{}{x_i}(x_j x_j) = 2x_j \pderiv{x_j}{x_i} = 2x_j \delta_{i j} = 2x_i
      \]
    \item $\nabla |\vec{x} - \vec{x}_0|^2 = 2(\vec{x} - \vec{x}_0)$.
      We can evaluate this component wise, or use suffix notation and summation convention like above.
      However, if we shift the origin by letting $\vec{x}' = \vec{x} - \vec{x}_0$, then using multivariate chain rule \cref{MVC}:
      \[
      \pderiv{}{x_i'} = \pderiv{x_j}{x_i'} \pderiv{}{x_j} = \pderiv{}{x_i'}(x'_j + {x_0}_j) \pderiv{}{x_j} = \delta_{i j} \pderiv{}{x_j} = \pderiv{}{x_i}
      \]
      so $\nabla_{\vec{x}} |\vec{x} - \vec{x}_0|^2 = \nabla_{\vec{x}'} |\vec{x}'| = 2\vec{x}' = 2(\vec{x} - \vec{x}_0)$.
    \item For constant $\vec{a}$, $\nabla(\vec{a} \cdot \vec{x}) = \nabla(a_1 x + a_2 y + a_3 z) = (a_1, a_2, a_3) = \vec{a}$.
      Alternatively, $\pderiv{ }{x_i} (a_j x_j) = a_j \delta_{i j} = a_i$.
  \end{enumerate}
\end{example}
Similarly to $f'$ in 1D, if we need to specify where $\nabla f$ is evaluated at, we write $\nabla f(\vec{x}_1)$.
\begin{example}
  If $f(\vec{x}) = x + y^2 + z^{4}$ then $\nabla f(\vec{a}) = \nabla f(\vec{b})$ if and only if $\vec{a} - \vec{b}$ is parallel to the $x$-axis.

  $\nabla f(\vec{x}) = (1, 2y, 4z^3)$ so $\nabla f(\vec{a}) = \nabla f(\vec{b}) \iff (1, 2a_2, 4a^{3}_{3}) = (1, 2b_2, 4b^{3}_{3})$, that is $a_2 = b_2$ and $b_3$, so $\vec{a} - \vec{b} = (a_1 - b_1, 0, 0) \parallel \vec{e}_1$.
\end{example}
\begin{remark}[Notation]
  $\nabla f(2\vec{x})$ means find $\nabla f$ and then evaluate it at $2\vec{x}$, whereas $\nabla (f(2\vec{x}))$ means evaluate at $2\vec{x}$ and then find $\nabla$ of that.

  It is often easier to avoid ambiguity by defining a new variable, for example, $\vec{y} = 2\vec{x}$ so $\nabla (f(2\vec{x})) = 2 \nabla f(\vec{y})$.

  Similarly, in 2D we can unambiguously write:
  \[
    \at{\pderiv{}{u} f(u, v)}{u = 2x, v = 2y}
  \]
  instead of $f_x(2x, 2y)$ where it is ambiguous whether the partial is taken before or after substituting $(2x, 2y)$.
\end{remark}
We can rewrite the multivariate chain rule using grad:
\[
  \deriv{}{t}f(\vec{x}(t)) = \dot{\vec{x}} \cdot \nabla f
\]
and in terms of infinitesimals as:
\[
  \d{f} = \nabla f \cdot \d{\vec{x}}
\]
This gives us an alternative way to define $\nabla f$ that is coordinate independent, that is, we do not have to choose a particular basis before using it.
Given a function $f$, we can define $\nabla f$ to be the unique vector such that satisfies the above for every possible choice of $\d{\vec{x}}$.
\begin{example}
  Suppose we want to find $\nabla f$ for $f(\vec{x}) = |\vec{x}|^2$, then:
  \begin{align*}
    \d{f} = f(x + \d{\vec{x}}) - f(\vec{x}) &= \vec{x} \cdot \vec{x} + 2\vec{x} \cdot \d{\vec{x}} + \cancelto{0}{\d{\vec{x}} \cdot \d{\vec{x}}} - |\vec{x}|^2 \\
                                            &= 2\vec{x} \cdot \d{\vec{x}}
  \end{align*}
  so $\nabla f = 2\vec{x}$ by definition which agrees with \cref{gradExample}.
\end{example}
Sometimes we wish to differentiate with respect to the components of a vector other than $\vec{x}$.
We denote this $\nabla_{\vec{y}} \equiv \vec{e}_i \pderiv{}{y_i}$.
\begin{example}
  Consider $f(\vec{x}, \vec{y}) = |\vec{x} - 2\vec{y}|^2$.
  We know from \cref{gradExample} that $\nabla |\vec{x} - \vec{x}_0|^2 = 2(\vec{x} - \vec{x}_0)$ for a constant vector $\vec{x}_0$.
  Since $\vec{y}$ is a constant vector with respect to $\vec{x}$, $\nabla f = 2(\vec{x} - 2\vec{y})$.

  However if we are taking grad with respect to $\vec{y}$, $\vec{x}$ is a constant vector so:
  \[
    \nabla_{\vec{y}} |\vec{x} - 2\vec{y}|^2 = -2 \cdot 2 (\vec{x} - 2\vec{y}) = -4(\vec{x} - 2\vec{y})
  \]
  This can also be checked using suffix notation.
\end{example}

\section{Divergence}
\begin{definition}[Divergence]
If we take the dot product of del with a vector field $\vec{F}(\vec{x})$, then we get the \textit{divergence operator}:
\[
  \nabla \cdot \vec{F} = \begin{pmatrix}
    \partial/\partial x_1 \\
    \partial/\partial x_2 \\
    \partial/\partial x_3 \\
    \end{pmatrix} \cdot
    \begin{pmatrix}
    F_1 \\
    F_2 \\
    F_3 \\
    \end{pmatrix} =
    \pderiv{F_1}{x} + \pderiv{F_2}{y} + \pderiv{F_3}{z}
\]
or in suffix notation:
\[
  \nabla \cdot \vec{F} = \pderiv{F_i}{x_i}
\]
\end{definition}
To obtain the suffix notation expression we can use:
\begin{align*}
  \nabla \cdot \vec{F} &= \left(\vec{e}_i \pderiv{}{x_i}\right) \cdot (F_j \vec{e}_j) \\
                       &= \vec{e}_i \cdot \pderiv{F_j}{x_i} \vec{e}_j \\
                       &= \pderiv{F_j}{x_i} \delta_{i j} = \pderiv{F_j}{x_i}\\
\end{align*}
$\partial/\partial x_i$ is a scalar operator so can be moved ``through'' the dot product as:
\[
  \left(\vec{a} \pderiv{}{x_i}\right) \cdot \vec{b} =
  \left[\begin{pmatrix}
  a_1 \\
  a_2 \\
  a_3 \\
  \end{pmatrix} \pderiv{}{x_i}\right] \cdot \vec{b} =
  \begin{pmatrix}
  a_1 \pderiv{}{x_1} \\
  a_2 \pderiv{}{x_2} \\
  a_3 \pderiv{}{x_3} \\
  \end{pmatrix} \cdot
  \begin{pmatrix}
  b_1 \\
  b_2 \\
  b_3 \\
  \end{pmatrix}
  = \vec{a} \cdot \begin{pmatrix}
  \pderiv{b_1}{x_1} \\
  \pderiv{b_2}{x_2} \\
  \pderiv{b_3}{x_3} \\
  \end{pmatrix} =
  \vec{a} \cdot \left(\pderiv{}{x_i} \vec{b}\right)
\]
Roughly speaking, the divergence of a vector field at $\vec{x}$, measures the amount at which the field is expanding around $\vec{x}$.
Later in the course we will derive a coordinate-independent definition of $\nabla \cdot \vec{F}$.
\begin{example}[Using Divergence]
  \begin{enumerate}
    \item $F(\vec{x}) = \vec{x} = (x, y, z)$ so $\nabla \cdot \vec{F} = \pderiv{x}{x} + \pderiv{y}{y} + \pderiv{z}{z} = 3$, or using suffix notation, $F_i = x_i$ so $\nabla \cdot \vec{F} = \delta_{i i} = 3$.
      \begin{center}
      \begin{tikzpicture}[scale=1.13]
        \begin{scope}
          \clip (-2,-2) rectangle (2,2);
          \foreach \x in {-2, -1.5, ..., 2} {
              \foreach \y in {-2, -1.5, ..., 2} {
                  \draw[-{Stealth[length=3pt,width=4.6pt]}] (\x, \y) -- (\x*1.5,\y*1.5);
              }
          }
        \end{scope}
      \end{tikzpicture}
      \end{center}
      We see that the vector field is expanding so agrees with our rough interpretation of the divergence.
    \item $\vec{F}(\vec{x}) = (-y, x, 0)$ so $\nabla \cdot \vec{F} = -\pderiv{y}{x} + \pderiv{x}{y} = 0$.
    \item If $A$ is a $3 \times 3$ matrix then.
      \[
        \nabla \cdot (A \vec{x}) = \pderiv{}{x_i}(A_{i j }x_j) = A_{i j}\delta_{i j} = \tr A
      \]
      This works as we are using $\vec{F}(\vec{x}) = A\vec{x}$ so it is acting on a vector field.
      This means that the trace of a matrix loosely represents how much the vector field associated with $A$ is expanding around any point.
  \end{enumerate}
\end{example}
\section{Curl}
\begin{definition}[Curl]
  The \textit{curl} of a vector field $\vec{F}(\vec{x})$ is defined as:
  \[
    \nabla \times \vec{F} = \begin{pmatrix}
      \partial/\partial x_1 \\
      \partial/\partial x_2 \\
      \partial/\partial x_3 \\
      \end{pmatrix} \times
      \begin{pmatrix}
      F_1 \\
      F_2 \\
      F_3 \\
      \end{pmatrix} =
      \begin{pmatrix}
      \pderiv{F_3}{y} - \pderiv{F_2}{z} \\
      \pderiv{F_1}{z} - \pderiv{F_3}{x} \\
      \pderiv{F_2}{x} - \pderiv{F_1}{y} \\
      \end{pmatrix}
  \]
  or in suffix notation:
  \[
    (\nabla \times \vec{F})_i = \levi_{i j k} \pderiv{}{x_j} F_k \text{ or equivalently } \nabla \times \vec{F} = \levi_{i j k} \pderiv{F_k}{x_j} \vec{e}_i
  \]
\end{definition}
\begin{remark}
  Curl is only defined in 3D due to the presence of a cross product, unlike grad and div which are defined in any dimension.
\end{remark}
We can also use the determinant form for the cross product to write:
\[
  \nabla \times \vec{F} = \begin{vmatrix}
  \vec{e}_1 & \vec{e}_2 & \vec{e}_3 \\
  \pderiv{}{x} & \pderiv{}{y} & \pderiv{}{z} \\
  F_1 & F_2 & F_3 \\
  \end{vmatrix}
\]
However, this only works if the determinant is expanded along the top row but is useful for memorisation nevertheless.

Roughly speaking, the curl of a vector field at a point $\vec{x}$ measures the rotation of the field around $\vec{x}$ although this will be explored more rigorously later.
The modulus $|\nabla \times \vec{F}|$ gives the magnitude of the rotation while its direction gives the axis.
\begin{example}[Using Curl]
  \begin{enumerate}
    \item With $\vec{F}(\vec{x}) = (x, y, z)$:
      \[
        \nabla \times \vec{F} = \begin{pmatrix}
        \pderiv{}{x} \\
        \pderiv{}{y} \\
        \pderiv{}{z} \\
        \end{pmatrix} \times \begin{pmatrix}
        x \\
        y \\
        z \\
        \end{pmatrix} = \vec{0}
      \]
      or, $(\nabla \times \vec{F})_i = \levi_{i j k} \pderiv{}{x_j} x_k = \levi_{i j k} \delta_{j k} = 0$.
    \item With $\vec{F}(\vec{x}) = (-y, x, 0)$, we have:
      \[
        \nabla \times \vec{F} = \begin{pmatrix}
        \pderiv{}{y}(0) - \pderiv{}{z}(x) \\
        \pderiv{}{z}(-y) - \pderiv{}{x}(0) \\
        \pderiv{}{x}(x) - \pderiv{}{y}(-y) \\
        \end{pmatrix} =
        \begin{pmatrix}
        0 \\
        0 \\
        2 \\
        \end{pmatrix}
      \]
    \item If $A$ is a $3\times 3$ symmetric matrix then.
      \[
        (\nabla \times (A\vec{x}))_i = \levi_{i j k} \pderiv{}{x_j}A_{k l}x_l = \levi_{i j k} A_{k j}
      \]
      Since $A_{k j}$ is a symmetric and $\levi_{i j k}$ is anti-symmetric, $\levi_{i j k} A_{k j} = 0$ and thus $\nabla \times (A\vec{x}) = \vec{0}$.
  \end{enumerate}
\end{example}
\end{document}
