\documentclass[../main.tex]{subfiles}
\begin{document}
\chapter{Discrete Random Variables}
\section{Definition}
\begin{definition}[Discrete Probability Distribution]
  Suppose we have a probability space $(\Omega, \salg, \P)$ where $\Omega$ is \textbf{finite or countable}:
  \[
    \Omega = \{\omega_1, \omega_2, \ldots\},\ \salg = \powerset{\Omega}
  \]
  If we know $\P(\{\omega_i\})$ for all $i$, then for any $A \in \salg$:
  \[
    \P(A) = \P\left(\bigcup_{\omega_i \in A} \{\omega_i\}\right) = \sum_{\omega_i \in A} \P(\{\omega_i\})
  \]
  We then call $(\P(\{\omega_i\}))_i$ a \textit{discrete probability distribution} and we write $p_i = \P(\{\omega_i\})$.
\end{definition}
Since the $p_i$ are all probabilities, we must have $p_i \geq 0\ \forall i$.
Furthermore, since $\P(\Omega) = 1$, we must also have $\sum_{i} p_i = 1$
\begin{remark}[Summary]
   If we have a probability space $(\Omega, \salg, \P)$ where the $\Omega$ is at most countably infinite, then the \textit{discrete probability distribution} refers to the probabilities that $\P$ assigns to each singleton $\{\omega_i\}$ of $\Omega$.
\end{remark}
\section{Important Discrete Distributions}
\subsection{Bernoulli Distribution}
\begin{definition}[Bernoulli Distribution]
  The \textit{Bernoulli distribution} with parameter $p \in [0, 1]$, models a probability space with two outcomes where:
  \[
    \Omega = \{0, 1\},\ p_0 = \P(\{0\}) = 1 - p,\ p_1 = \P(\{1\}) = p
  \]
\end{definition}
This means Bernoulli distributions model the outcome of a single trial with two outcomes, for example the toss of a biased coin.
If we associate tails with 0 and heads with 1 then:
\[
  \P(\text{Heads}) = p_1 = p,\ \P(\text{Tails}) = p_0 = 1 - p
\]
\begin{remark}
  A coin with a probability $p$ of heads is sometimes called a $p$-coin.
\end{remark}
\subsection{Binomial Distribution}
\begin{definition}
  The \textit{binomial distribution} with parameters $n \in \N$ and $p \in [0, 1]$ is denoted $\binomial(n, p)$.

  It models number of successes in $n$ independent trials, each with an outcome of either success, with probability $p$, or failure, with probability $1 - p$.

  $\Omega$ all the possible numbers of trials where the outcome was success, i.e. $\Omega = \{1, \ldots, n\}$ as we can have anywhere from 1 to $n$ successful trials.

  The probability $p_k = \P(\{k\})$ is:
  \[
    \P(\text{$k$ successes}) = \binom{n}{k}p^{k}(1 - p)^{n - k} \text{ for $k \in \Omega$}
  \]
  as there are $\binom{n}{k}$ choices for when the successes should be.
\end{definition}
\begin{remark}[Intuition]
  We can think of the binomial distribution as modelling the number of heads we get if we toss a $p$-coin $n$ times.
\end{remark}
\subsection{Multinomial Distribution}
\begin{definition}
  The \textit{multinomial distribution} with parameters $n \in \N$ and $p_1, \ldots, p_k \in [0, 1]$ such that $\sum_{i = 1}^{k} p_i = 1$, is denoted $\multinomial(n, p_1, \ldots, p_k)$.

  It models the outcome of $n$ independent trials, each with $k$ outcomes of probability $p_i$ for $1 \leq i \leq k$.

  $\Omega$ is the set of all tuples whose entries are the number of times $n_i$ each of the $k$ outcomes occurred:
  \[
    \Omega = \left\{(n_1, \ldots, n_k) \in \N^{k} : \sum_{i = 1}^{k} n_i = n\right\}
  \]
  We require their sum to be $n$ as we did $n$ trials in total.

  The probability of a particular tuple, $p_{n_1, \ldots, n_k} = \P(\{(n_1, \ldots, n_k)\})$ is:
  \begin{align*}
    p_{n_1, \ldots, n_k} &= \binom{n}{n_1}p^{n_1}_{1} \cdot \binom{n - n_1}{n_2}p^{n_2}_{2} \cdots \binom{n - n_1 - \cdots - n_{k - 1}}{n_k} p^{n_k}_{k} \\
                         &= \binom{n}{n_1, \ldots, n_k} \cdot p^{n_1}_{1} \cdots p^{n_k}_{n}
  \end{align*}
  This uses the \textit{multinomial coefficient} that we saw in \cref{multinomialCoefficients}.
\end{definition}
\begin{remark}[Intuition]
  We can think of the multinomial distribution as modelling the results of tossing $n$ balls independently into $k$ boxes where each ball goes into box $i$ with probability $p_i$.
  Therefore:
  \[
    \P(n_1 \text{ balls in box 1}, \ldots, n_k \text{ balls in box $k$}) = p_{n_1, \ldots, n_k}
  \]
\end{remark}
\end{document}
