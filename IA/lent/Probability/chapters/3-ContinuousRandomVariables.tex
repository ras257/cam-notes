\documentclass[../main.tex]{subfiles}
\begin{document}
\chapter{Continuous Random Variables}
\section{Introducing Continuous Probability}
\begin{remark}[Recap]
  Back in \cref{probabilitySpaceDef}, we stared by defining a probability space $(\Omega, \salg, \P)$.
  We then moved on to random variables.
  These are functions $X: \Omega \to \R$ that satisfy:
  \[
    \forall x \in \R,\ \{X \leq x\} = \{\omega \in \Omega: X(\omega) \leq x\} \in \salg
  \]
  We also defined the \textit{probability distribution function} (pdf) (\cref{pdfDefinition}) of $X$ as:
  \[
    F: \R \to [0, 1],\ F(x) = \P(X \leq x)
  \]
\end{remark}
In the case of a discrete random variables, the pdf is just a sum of probabilities and so is a \textit{step function}:
\begin{center}
\begin{tikzpicture}[>=stealth]
  \draw[->] (0, 0) -- (5.5, 0);
  \draw[->] (0, 0) -- (0, 5);

  \filldraw[fill=white, draw=black] (1, 1) circle (1.5pt);
  \clip (0, -0.5) rectangle (5.2, 5.2);
  \foreach \i in {1, 2, 3, 4, 5} {
    \draw[dashed] (\i, \i + 1) -- (\i, 0) node[below] {$x_\i$};
    \draw (\i-1, \i) -- (\i, \i);
    \filldraw[fill=white, draw=black] (\i, \i) circle (1.5pt);
    \fill[black] (\i, \i + 1) circle (1.5pt);
  }
\end{tikzpicture}
\end{center}
We can show some properties of pdfs that hold regardless of if we are working with discrete random variables.
\begin{proposition}[Properties of PDFs]
  \begin{enumerate}
    \item If $x \geq y$, then $F(x) \leq F(y)$.
    \item $\forall a, b \in \R$, if $a < b$ then $\P(a < X \leq b) = F(b) - F(a)$
  \end{enumerate}
\end{proposition}
\begin{proof}
  Continued next lecture.
\end{proof}
\end{document}
