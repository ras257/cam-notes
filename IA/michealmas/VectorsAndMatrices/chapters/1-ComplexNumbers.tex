\documentclass[../main.tex]{subfiles}
\begin{document}
\chapter{Complex Numbers}
\section{Definition of Complex Numbers}
\begin{definition}[Complex Number]
  We construct $\C$ by adding the element $i$ to $\R$, where $i^2 = -1$.

  Then any complex number $z \in \C$ can be written as:
  \[
    z = x + iy \quad x, y \in \R
  \]
\end{definition}
We denote the real part of $z$ as $\Re(z) = x$ and the imaginary part as $\Im(z) = y$.
\section{Properties of Complex Numbers}
For complex numbers $z_1 = x_1 + i y_1$ and $z_2 = x_2 + i y_2$ then we can carry out addition component-wise:
\[
  z_1 \pm z_2 = (x_1 \pm x_2) + i(y_ 1 \pm y_2) 
\]
and noting that $i^2 = -1$, we can carry out multiplication:
\begin{equation}
  z_1 z_2 = (x_1 x_2 - y_1 y_2) + i(x_1 y_2 + x_2 y_1)
  \label{complexMult}
\end{equation}
Both addition and multiplication of complex numbers are associative and commutative.

The identity for the operation $+$ in $\C$ is 0 and note that $(\C, +, 0)$ forms and abelian group. 

For $z \neq 0$ the inverse of $z$ is given by:
\begin{equation}
  z^{-1} = \frac{x - iy}{x^2 + y^2}
  \label{inverse}
\end{equation}
Note that $(\C \setminus \{0\}, \cdot, 1)$ forms and abelian group.

Moreover, the distributive property is satisfied. i.e. for $z_1, z_2, z_3 \in \C$ then:
\[
  (z_1 + z_2)z_3 = z_1 z_2 + z_2 z_3
\]
\begin{definition}[Complex Conjugate]
  For any $z \in \C$ with $z = x + iy$ we define its \textit{complex conjugate} by:
  \[
    \bar{z} = z^{*} = x - iy
  \]
\end{definition}
Note that this allows us to write:
\begin{align*}
  \Re(z) &= \frac{z + z^{*}}{2} \\
  \Im(z) &= \frac{z - z^{*}}{2i} \\
  |z|^2 &= z z^{*}
\end{align*}
It also leads to a second form for \cref{inverse}:
\[
  z^{-1} = \frac{z^{*}}{|z|^2}
\]
Some properties satisfied by the complex conjugate:
\begin{itemize}
  \item $(z^{*})^{*} = z$
  \item $(z_1 + z_2)^{*} = z^{*}_{1} + z^{*}_{2}$
  \item $(z_1 z_2)^{*} = z^{*}_{1} z^{*}_{2}$
\end{itemize}
\begin{definition}[Modulus of a Complex Number]
  For any $z \in \C$ with $z = x + iy$ we define its \textit{modulus} by a real and non-negative number $|z|$, such that:
  \[
    |z|^2 = x^2 + y^2
  \]
  The modulus of $z$ is sometimes denoted by $r$.
\end{definition}
\begin{definition}[Argument of a Complex Number]
  For any $z \in \C \setminus \{0\}$ we define its \textit{argument} by a real number denoted $\theta$ or $\arg(z)$ such that:
  \[
    z = r(\cos \theta + i \sin \theta)
  \]
\end{definition}
Writing $z$ as above is called the ``polar form'' of $z$.

From this it follows that (for $z \neq 0$):
\[
  \cos \theta = \frac{x}{r} = \frac{x}{\sqrt{x^2 + y^2}}\text{ and }\sin \theta = \frac{y}{r} = \frac{y}{\sqrt{x^2 + y^2}}
\]
and thus:
\[
  \tan \theta = \frac{y}{x}
\]
If $\theta$ is the argument of $z$ then any $\theta + 2n\pi, n \in \Z$ is also an argument of $z$.
To make the argument of $z$ unique, we restrict the range of $\theta$ to $-\pi < \theta \leq \pi$ and call this argument the ``principal argument''.
\begin{remark}[Other remarks about $\C$]
  \begin{itemize}
    \item $\R \subset \C$, since if we take $a \in \R$ then we can construct $a + i \cdot 0 \in \C$.
    \item A complex number in the form $0 + i \cdot b$ (i.e. $\Re(z) = 0$) is called ``purely imaginary''.
    \item The representation of a complex number in terms of its real and imaginary part is unique.
    \item $(\C, +, \cdot)$ is a field.
  \end{itemize}
\end{remark}
\begin{theorem}[Fundemental Theorem of Algebra]
  A polynomial of degree $n$ with coefficients $c_i \in \C$ can be written as a product of $n$ linear factors. i.e.:
  \[
    p(z) = c_n z^n + \cdots + c_0 = c_n(z - \alpha_n) \cdots (z - \alpha_0)
  \]
\end{theorem}
Hence $p(z) = 0$ has at least one root in $\C$ and $n$ roots $\alpha_i$ counted with multiplicity.
\begin{proposition}
  The modulus satisfies the following properties:
  \begin{enumerate}
    \item $|z_1 z_2| = |z_1||z_2|$
    \item $|z_1 + z_2| \leq |z_1| + |z_2|$ (Triangle Inequality)
    \item $|z_1 - z_2| \geq ||z_1| - |z_2||$
  \end{enumerate}
\end{proposition}
\begin{proof}[\textbf{i}]
  \begin{align*}
      |z_1 z_2| &= \sqrt{(x_1 x_2 - y_1 y_2)^2 + (x_1 y_2 + x_2 y_1)^2}\\
                &= \sqrt{x^{2}_{1} x^{2}_{2} -2x_1x_2y_1y_2 + y^{2}_{1}y^{2}_{2} + x^{2}_{1}y^{2}_{2} + 2x_1x_2y_1y_2 + x^{2}_{2}y^{2}_{1}} \\
                &= \sqrt{x^{2}_{1} x^{2}_{2} + y^{2}_{1} y^{2}_{2} + x^{2}_{1} y^{2}_{2} + x^{2}_{2} y^{2}_{1}} \\
                &= \sqrt{\left(x^{2}_{1} + y^{2}_{1}\right)\left(x^{2}_{2} + y^{2}_{2}\right)} \\
                &= \sqrt{x^{2}_{1} + y^{2}_{1}}\sqrt{x^{2}_{2} + y^{2}_{2}} \\
                &= |z_1||z_2|
  \end{align*}
\end{proof}
\begin{proof}[\textbf{ii}]
  Since both sides of the inequality are non-negative, $|z_1 + z_2| \leq |z_1| + |z_2| \iff |z_1 + z_2|^2 \leq (|z_1| + |z_2|)^2$.

  Consider:
  \begin{align*}
      (|z_1| + |z_2|)^2 - |z_1 + z_2|^2 &= x^{2}_{1} + y^{2}_{1} + 2|z_1|| z_2| + x^{2}_{2} + y^{2}_{2} \\
                                        & \quad - (x^{2}_{1} + 2x_1x_2 + x^{2}_{2} + y^{2}_{1} + 2y_1y_2 + y^{2}_{2})\\
                                        &= 2(|z_1||z_2| - x_1 x_2 - y_1 y_2)
  \end{align*}
  We want to prove that the above is positive:
  \begin{align*}
    |z_1||z_2| - x_1 x_2 - y_1 y_2 &\geq 0 \\
    |z_1 z_2| &\geq x_1 x_2 + y_1 y_2
  \end{align*}
  Either $x_1 x_2 + y_1 y_2 \leq 0$ and we are done or:
  \begin{align*}
    (x^{2}_{1} + y^{2}_{1})(x^{2}_{2} + y^{2}_{2}) &\geq (x_1 x_2 + y_1 y_2)^2 \\
    x^{2}_{1} x^{2}_{2} + x^{2}_{1} y^{2}_{2} + x^{2}_{2} y^{2}_{1} + y^{2}_{1} y^{2}_{2}&\geq x^{2}_{1} x^{2}_{2} + 2x_1 x_2 y_1 y_2 + y^{2}_{1} y^{2}_{2} \\
    x^{2}_{1} y^{2}_{2} - 2x_1 x_2 y_1 y_2 + x^{2}_{2} y^{2}_{1} &\geq 0 \\
    (x_1 y_2 - x_2 y_1)^2 &\geq 0
  \end{align*}
  Thus $(|z_1| + |z_2|)^2 \geq |z_1 + z_2|^2$ and therefore $|z_1 + z_2| \leq |z_1| + |z_2|$.
\end{proof}
\begin{proof}[\textbf{iii}]
  In the triangle inequality, set $z_2 \to z_3 - z_1$.
  This leads to:
  \begin{align*}
    |z_1 + z_3 - z_1| &\leq |z_1| + |z_3 - z_1|\\
    |z_3| - |z_1| &\leq |z_3 - z_1|
  \end{align*}
  Now we have that $|z_1 - z_2| \leq |z_1| - |z_2|$.

  Notice that if we swap $z_1$ and $z_2$ we have that $|z_2 - z_1| \leq |z_2| - |z_1|$.
  But $|z_1 - z_2| = |z_2 - z_1|$ so $|z_1 - z_2|$ is greater than or equal to both $|z_1| - |z_2|$ and $-(|z_1| - |z_2|)$.

  This means it must be greater than the absolute value of $|z_1| - |z_2|$ so we have the required result $|z_1 - z_2| \geq ||z_1| - |z_2||$.
\end{proof}
\section{The Argand Diagram}
The Argand diagram is sometimes called the ``Complex plane''.

To plot a complex number $z = x + iy$ on the Argand diagram, we plot the real part of $z$ along the $x$-axis and the imaginary part of $z$ along the $y$-axis, i.e. as a coordinate $(x, y)$ on a 2D-plane.
\begin{center}
\begin{tikzpicture}[scale=0.8]
  \draw (-3.5,0) -- (3.5,0) node[right] {\(\Re\)};
  \draw (0,-3.5) -- (0,3.5) node[right] {\(\Im\)};

  \coordinate (Z) at (2,2.5);

  \filldraw (Z) circle (1pt) node[above right] {\(z = x + iy\)} node[below right] {\((x, y)\)};
\end{tikzpicture}
\end{center}

By representing complex numbers in the Argand diagram then the addition of complex numbers can be treated as vector addition (tip-to-tail):
\begin{center}
\begin{tikzpicture}[scale=0.8]
  \draw (-3.5,0) -- (3.5,0) node[right] {\(\Re\)};
  \draw (0,-3.5) -- (0,3.5) node[right] {\(\Im\)};

  \coordinate (Z1) at (1,1.5);
  \coordinate (Z2) at (1,0.5);
  \coordinate (sum) at (2, 2);

  \draw[->] (0,0) -- (Z1);
  \draw[->] (0,0) -- (Z2);
  \draw[->] (0, 0) -- (sum);

  \draw[dotted] (Z1) -- (sum);
  \draw[dotted] (Z2) -- (sum);

  \filldraw (Z1) node[above left] {\(z_1\)};
  \filldraw (Z2) node[below right] {\(z_2\)};
  \filldraw (sum) node[above right] {\(z_1 + z_2\)};
\end{tikzpicture}
\end{center}

Note that in the Argand diagram the complex conjugate of a complex number $z$ is its reflection in the real axis:
\begin{center}
\begin{tikzpicture}[scale=0.8]
  \draw (-3.5,0) -- (3.5,0) node[right] {\(\Re\)};
  \draw (0,-3.5) -- (0,3.5) node[right] {\(\Im\)};

  \coordinate (Z) at (2,2.5);
  \coordinate (Zconj) at (2,-2.5);

  \draw[dashed] (Z) -- (Zconj);

  \filldraw (Z) circle (1pt) node[above right] {\(z = x + iy\)};
  \filldraw (Zconj) circle (1pt) node[below right] {\(z^{*} = x - iy\)};
\end{tikzpicture}
\end{center}

This makes it easier to visually derive the following properties:
\begin{itemize}
  \item $(z_1 + z_2)^{*} = z^{*}_{1} + z^{*}_{2}$
  \item $(z_1 z_2)^{*} = z^{*}_{1} z^{*}_{2}$
  \item $|z^{*}| = |z|$
\end{itemize}
\begin{theorem}[De Moivre's Theorem]
  For $n \in \Z$ then:
  \[
    (\cos \theta + i \sin \theta)^n = \cos n\theta + i \sin n\theta
  \] 
\end{theorem}
We first need to prove the following lemma:
\begin{lemma}
  If $z_1 = r_1(\cos \theta_1 + i \sin \theta_1)$ and $z_2 = r_2(\cos \theta_2 + i \sin \theta_2)$, then:
  \[
    z_1 z_2 = r_1 r_2 (\cos (\theta_1 + \theta_2) + i\sin(\theta_1 + \theta_2))
  \]
\end{lemma}
\begin{proof}
  Using \cref{complexMult} we have that:
  \begin{align*}
    z_1 z_2 &= r_1 r_2 (\cos \theta_1 \cos \theta_2 - \sin \theta_1 \sin \theta_2 + i(\cos \theta_1 \sin \theta_2 + \cos \theta_2 \sin \theta_1)) \\
            &= r_1 r_2 (\cos(\theta_1 + \theta_2) + i\sin(\theta_1 + \theta_2))
  \end{align*} 
\end{proof}
\section{Exponential and Trigonometric Form}
\section{Logarithm and Complex Powers}
\section{Lines and Circles}
\end{document}
