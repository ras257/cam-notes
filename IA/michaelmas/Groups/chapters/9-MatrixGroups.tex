\documentclass[../main.tex]{subfiles}
\begin{document}
\chapter{Matrix Groups}
\section{General and Special Linear Groups}
Let $M_n(\R)$ be the set of $n \times n$ matrices with real entries.
From Vectors and Matrices we know that matrix multiplication is associative and the identity matrix provides an identity for matrix multiplication.
However, not all matrices are invertible, recall that:
\begin{lemma}
  A matrix $A \in M_n(\R)$ has an inverse if and only if $\det A \neq 0$.
\end{lemma}
\begin{proof}
  See Vectors and Matrices 3.5.1
\end{proof}
\begin{definition}[General Linear Group]
  The \textit{General Linear Group} is defined as:
  \[
    \GL_n(\R) = \{A \in M_n(\R) : \det A \neq 0\}
  \]
  This is a group by the above discussion.
\end{definition}
\begin{lemma}
  For matrices $A, B \in M_n(\R)$:
  \[
    \det(AB) = \det A \det B
  \]
\end{lemma}
\begin{proof}
  See Vectors and Matrices 3.5.5
\end{proof}
This says that $\det$ is a homomorphism:
\[
  \det: \GL_n(\R) \to \R_\times,\ A \mapsto \det A
\]
\begin{remark}[Notation]
  $\R_\times$ is the group of $(\R \setminus \{0\}, 1, \times)$ under multiplication.
\end{remark}
\begin{definition}[Special Linear Group]
  The \textit{special linear group} is defined to be:
  \begin{align*}
    \SL_n(\R) &= \ker(\det) \\
              &= \{A \in M_n(\R) : \det A = 1\}
  \end{align*}
\end{definition}
Since $\SL_n$ is the kernel of a homomorphism, it must be a normal subgroup of $\GL_n$, that is:
\[
  \SL_n(\R) \lhd \GL_n(\R)
\]
So by the Isomorphism Theorem (\cref{isomorphismTheorem}), we have:
\[
  \GL_n(\R) / \SL_n(\R) \cong \im (\det)
\]
Consider following determinant for any $x \in \R_\times$:
\[
  \det \begin{pmatrix}
  x & 0 & \cdots & 0 & 0 \\
  0 & 1 & \cdots & 0 & 0 \\
  \vdots & \vdots & \ddots & \vdots & \vdots \\
  0 & 0 & \cdots & 1 &  0\\
  0 & 0 & \cdots & 0 & 1 \\
  \end{pmatrix} =
  \det \left(\begin{array}{c|c}
    x & O \\ \hline
    O & I
  \end{array}\right) = x
\]
Therefore, $\im(\det) = \R_\times$ so:
\[
  \GL_n(\R)/\SL_n(\R) = \R_\times
\]

All of the above makes just as much sense when $\R$ is replaced with $\C$ so we also have:
\[
  \GL_n(\C) \text{ and } \SL_n(\C)
\]
and again:
\[
  \GL_n(\C) / \SL_n(\C) \cong \C_\times
\]
\section{Change of Basis}
There is a natural action of $\GL_n(\R)$ on $M_n(\R)$ by conjugation, that is:
\[
  \GL_n(\R) \acts M_n(\R) \text{ by } P(A) = PAP^{-1}
\]
\begin{proposition}
  Let $V$ be an $n$ dimensional vector space (over $\R$) and $\alpha: V \to V$ a linear map.
  If $A \in M_n(\R)$ that represents $\alpha$ in some basis, then the orbit:
  \[
    GL_n(\R)A = \{PAP^{-1} : P \in \GL_n(\R)\}
  \]
  consists of \textbf{all} the matrices that represent $\alpha$ in any basis.
\end{proposition}
\begin{proof}
  A basis $\{\vec{v}_1, \ldots, \vec{v}_n\}$ for $V$ defines an isomorphism of vector spaces:
  \[
    \phi: \R^{n} \to V,\ (\lambda_1, \ldots, \lambda_n) \mapsto \sum_{i = 1}^{n} \lambda_i \vec{v}_i
  \]
  This relation between $\R^{n}$ and $V$ can be expressed diagrammatically:
  \begin{center}
  \begin{tikzcd}
  \R^n \arrow{d}[swap]{A} \arrow{r}{\phi}[swap]{\cong} & V \arrow{d}{\alpha} \\
  \R^n \arrow{r}{\phi}[swap]{\cong} & V
  \end{tikzcd}
  \end{center}
  The claim that $A$ represents $\alpha$ in this basis means that:
  \[
    \alpha = \phi A \phi^{-1}
  \]
  Continued next lecture.
\end{proof}
\end{document}
