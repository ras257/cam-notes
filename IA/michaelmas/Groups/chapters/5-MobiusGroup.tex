\documentclass[../main.tex]{subfiles}
\begin{document}
\chapter{The M\"obius Group}
\section{Introduction and Definitions}
This chapter is all about a group called the \textit{M\"obius Group}.
It is \textit{almost} a group of bijections of $\C$ \textbf{but} we need to add a formal ``point at infinity''.
\subsection{Riemann Sphere and M\"obius Transforms}
\begin{definition}[Riemann Sphere]
  The \textit{Riemann sphere}, $\Cinf$, is defined to be:
  \[
    \Cinf = \C \cup \{\infty\}
  \]
\end{definition}
We define $\pi(z)$ to be the \textit{stereographic projection} of $z \in \C$ onto $S^2$ (The unit sphere).
If we draw a line through $z$ and the north pole, this line will intersect the sphere exactly once, this is $\pi(z)$.
The stereographic projection identified $\C$ with all the points of the sphere except the north pole, which we define to be $\infty$.
\begin{definition}[M\"obius Transform]
  Let $a, b, c, d \in \C$ such that $ad - bc \neq 0$.
  The \textit{m\"obius transform}, $\mu: \Cinf \to \Cinf$ is defined as:

  For $c \neq 0$:
  \[
    z \mapsto \begin{cases}
    \frac{az + b}{cz + d} & \text{if $z \in \C \setminus \{-\frac{d}{c}\}$} \\
    \infty & \text{if $z = -\frac{d}{c}$} \\
    \frac{a}{c} & \text{if $z = \infty$}
    \end{cases}
  \]
  For $c = 0$:
  \[
    z \mapsto \begin{cases}
    \frac{az + b}{d} & \text{if $z \in \C$} \\
    \infty & \text{if $z = \infty$}
    \end{cases}
  \]
\end{definition}
For brevity, we usually just write:
\[
  \mu(z) = \frac{az + b}{cz + b}
\]
and interpret the cases of $0$ and $\infty$ appropriately.
\section{M\"obius Group}
\begin{definition}[M\"obius Group]
  The set $\mob$:
  \[
    \mob = \{f: \Cinf \to \Cinf: f \text{ is a M\"obius transform}\}
  \]
  together with composition is the \textit{M\"obius Group}.
\end{definition}

\begin{proposition}
  The composition of two M\"obius transforms is also a M\"obius transform.
  \label{mobComposition}
\end{proposition}
\begin{proof}
  If we take:
  \[
    \mu_1(z) = \frac{a_1 z + b_1}{c_1 z + d_1},\
    \mu_2(z) = \frac{a_2 z + b_2}{c_2 z + d_2}
  \]
  Then we see that $\mu_1 \circ \mu_2$ is:
  \begin{align*}
    \mu_1 \circ \mu_2 (z) &= \frac{a_1\mu_2(z) + b_1}{c_1\mu_2(z) + d_1} \\
                          &= \frac{a_1\left(\frac{a_2 z + b_2}{c_2 z + d_2}\right) + b_1}{c_1\left(\frac{a_2 z + b_2}{c_2 z + d_2}\right) + d_1} \\
                          &= \frac{a_1(a_2 z + b_2) + b_1(c_2 z + d_2)}{c_1(a_2 z + b_2) + d_1(c_2 z + d_2)} \\
                          &= \frac{\overbrace{(a_1 a_2 + b_1 c_2)}^{a'}z + \overbrace{(a_1 b_2 + b_1 d_2)}^{b'}}{\underbrace{(c_1 a_2 + d_1 c_2)}_{c'}z + \underbrace{(c_1 b_2 + d_1 d_2)}_{d'}}
  \end{align*}
  So $\mu_1 \circ \mu_2$ is in the correct form.
  We now need to check that it satisfies $a'd' - b'c' \neq 0$:
  \begin{align*}
    a'd' - b'c' &= (a_1 a_2 + b_1 c_2)(c_1 b_2 + d_1 d_2) - (a_1 b_2 + b_1 d_2)(c_1 a_2 + d_1 c_2) \\
                &= \cancel{a_1 a_2 c_1 b_2} + b_1 b_2 c_1 c_2 + a_1 a_2 d_1 d_2 + \cancel{d_1 d_2 b_1 c_2}\\
                &\quad- \cancel{a_1 a_2 c_1 b_2} - b_1 c_1 a_2 d_2 - a_1 d_1 b_2 c_2 - \cancel{d_1 d_2 b_1 c_2} \\
                &= a_2 d_2(a_1 d_1 - b_1 c_1) - b_2 c_2(a_2 d_2 - b_2 c_2) \\
                &= (a_1 d_1 - b_1 c_1)(a_2 d_2 - b_2 c_2)
  \end{align*}
  Since $\mu_1$ and $\mu_2$ are M\"obius transforms, $a_1 d_1 - b_1 c_2 \neq 0$ and $a_2 d_2 - b_2 c_2 \neq 0$.
  Therefore $a'd' - b'c' \neq 0$, so $\mu_1 \circ \mu_2 \in \mob$.
\end{proof}
\begin{remark}
  $a', b', c', d'$ described above are the entries obtained when multiplying the following two matrices:
  \[
    \begin{pmatrix}
    a_1 & b_1 \\
    c_1 & d_1 \\
    \end{pmatrix}
    \begin{pmatrix}
    a_2 & b_2 \\
    c_2 & d_2 \\
    \end{pmatrix} =
    \begin{pmatrix}
    a_1a_2 + b_1c_2 & a_1b_2 + b_1d_2 \\
    c_1a_2 + d_1c_2 & c_1b_2 + d_1d_2 \\
    \end{pmatrix}
  \]
\end{remark}
\begin{theorem}[M\"obius Group]
  $(\mob, \circ, \id)$ is a group.
\end{theorem}
\begin{proof}
  \begin{itemize}
    \item \textbf{Associativity -} Composition of functions is associative, see \cref{assocLemma}.
    \item \textbf{Identity -}
      \[
        \id: z \mapsto z = \frac{1\cdot z + 0}{0 \cdot z + 1} \in \mob
      \]
    \item \textbf{Inverses -} Let:
      \[
        \mu: z \mapsto \frac{az + b}{cz + d},\ \nu: z \mapsto \frac{dz - b}{-cz + a}
      \]
      Then by \cref{mobComposition}:
      \[
        \mu \circ \nu = \left(\frac{ad - bc}{ad - bc}\right)z =z
      \]
      So $\mu^{-1} = \nu \in \mob$.
    \item \textbf{Closure -} See \cref{mobComposition}. %TODO
  \end{itemize}
\end{proof}
\section{Fixed Points}
One important way to study M\"obius transforms is via their fixed points.
\begin{definition}
  Suppose $f: X \to X$ is a permutation.
  Any $x \in X$ such that $f(x) = x$ is called a \textit{fixed point} of $f$.
\end{definition}
\begin{remark}[Notation]
  We denote the set of fixed points of a function $f$ as $\Fix(f)$.
\end{remark}
\begin{lemma}[3-point Lemma for $\mob$]
  If $\mu \in \mob$ fixes three distinct points $w_1, w_2, w_3 \in \Cinf$, then $\mu = \id$.
  \label{threePointMob}
\end{lemma}
\begin{proof}
  Let $\mu(z) = \frac{az + b}{cz + d}$.
  A fixed point $w_i$ satisfies the equation:
  \[
    w_i = \frac{aw_i + b}{cw_i + d}
  \]
  \begin{proofcases}
    \begin{case}{One $w_i$ is infinite}
      WLOG suppose $w_1 = \infty$, then $c = 0$ so $w_2$ and $w_3$ satisfy:
      \[
        w_i = \frac{aw_i + b}{d} \iff (a-d)w_i + b = 0
      \]
      This is a linear equation with at least 2 distinct roots so must be trivial, that is, $a = d$ and $b = 0$.
      So $\mu(z) = \frac{az}{d} = z = \id$.
    \end{case}
    \begin{case}{$w_i \neq \infty$}
      Now suppose every $w_i \neq \infty$.
      \begin{align*}
        w_i &= \frac{aw_i + b}{cw_i + d} \\
        cw^{2}_{i} + dw_i &= aw_i + b \\
        cw^{2}_{i} + (d - a)w_i - b &= 0
      \end{align*}
      This is a quadratic with at least 3 roots so must be trivial, that is, $b = c = 0$ and $a = d$.
      So again $\mu(z) = \id$.
    \end{case}
  \end{proofcases}
\end{proof}
\begin{proposition}
  Every $\mu \in \mob$ has at least one fixed point in $\Cinf$.
\end{proposition}
\begin{proof}
  \begin{proofcases}
    \begin{case}{$c = 0$}
      Then, by definition, $\mu(\infty) = \infty$ so $\infty \in \Fix(\mu)$.
    \end{case}
    \begin{case}{$c \neq 0$}
      A fixed point of $\mu$ satisfies:
      \[
        \mu(z) = \frac{az + b}{cz + d} = z \\
        \iff az + b = cz^2 + dz \\
        \iff cz^2 + (d - a)z - b = 0
      \]
      Since $c \neq 0$, this is a quadratic so has either one repeated or two distinct complex roots.
      In either case, there will be at least one $z$ satisfying $\mu(z) = z$.
    \end{case}
    In both cases, $\mu$ fixes at least one point.
  \end{proofcases}
\end{proof}
\begin{example}
  \begin{enumerate}
    \item If $\mu(z) = z + 1$, then $\Fix(\mu) = \{\infty\}$.
    \item If $\mu(z) = 2z$, then $\Fix(\mu) = \{0, \infty\}$.
  \end{enumerate}
\end{example}
\begin{lemma}[Triple Transitivity]
  For any triples of distinct points $z_1, z_2, z_3 \in \Cinf$ and $w_1, w_2, w_3 \in \Cinf$ there exists a $\mu \in \mob$ such that $\mu(z_i) = w_i$ for $i = 1, 2, 3$.
  \label{tripleTransitivity}
\end{lemma}
\begin{proof}
  Let
  \[
    \alpha(z) = \left(\frac{z - z_1}{z - z_3}\right) \left(\frac{z_2 - z_3}{z_2 - z_1}\right)
  \]
  \begin{remark}[Note]
    If one of $z_i$ is infinite, then we can modify $\alpha$ to be:
    \[
      z_1 = \infty \to \frac{z_2 - z_3}{z - z_3},\ z_2 = \infty \to \frac{z - z_1}{z - z_3}, z_3 = \infty \to \frac{z - z_1}{z_2 - z_1}
    \]
    and the following still holds true.
  \end{remark}
  We can now see that:
  \[
    \alpha: z_1 \mapsto 0,\ z_2 \mapsto 1,\ z_3 \mapsto \infty
  \]
  Similarly, we can write down a $\beta \in \mob$ such that:
  \[
    \beta: w_1 \mapsto 0,\ w_2 \mapsto 1,\ w_3 \mapsto \infty
  \]
  Thus we have:
  \[
    \beta^{-1} \circ \alpha: z_1 \mapsto w_1,\ z_2 \mapsto w_2,\ z_3 \mapsto w_3
  \]
  So $\mu = \beta^{-1} \circ \alpha$ works.
\end{proof}
\begin{remark}
  The above $\mu$ is unique as if we had some other $\nu \in \mob$ with
  \[
    \nu: z_1 \mapsto w_1,\ z_2 \mapsto w_2,\ z_3 \mapsto w_3
  \]
  then $\Fix(\nu^{-1} \circ \mu) = \{z_1, z_2, z_3\}$ so by \cref{threePointMob}, $\nu^{-1} \circ \mu = \id$ so $\nu = \mu$.

  We say that the action of $\mob$ on $\Cinf$ is \textit{sharply triple transitive}.
\end{remark}
\section{Cross Ratio}
\begin{definition}[Cross Ratio]
  Let $z_1, z_2, z_3, z_4 \in \Cinf$ be distinct.
  Because $\mob \acts \Cinf$ sharply triple transitively, there is a unique $\alpha \in \mob$ such that:
  \[
    \alpha(z_1) = 0,\ \alpha(z_2) = 1,\ \alpha(z_3) = \infty
  \]
  The \textit{cross-ratio} is then defined to be:
  \[
    [z_1, z_2, z_3, z_4] = \alpha(z_4)
  \]
\end{definition}
\begin{remark}[Note]
  We see from \cref{tripleTransitivity} that:
  \[
    [z_1, z_2, z_3, z_4] = \left(\frac{z_4 - z_1}{z_4 - z_3}\right)\left(\frac{z_2 - z_3}{z_2 - z_1}\right)
  \]
\end{remark}
\end{document}
