\documentclass[../main.tex]{subfiles}
\begin{document}
\chapter{Introduction To Groups}
\section{Examples and Definitions}
You can think about groups in two ways:
\begin{itemize}
  \item Algebra
  \item Symmetries
\end{itemize}
\subsection{A Symmetries Viewpoint}
Consider the symmetries of an equilateral triangle:
\begin{itemize}
  \item The ``do-nothing'' symmetry (identity)
  \item Rotational symmetry of order 3. (Rotate $120^\circ$ clockwise or anti-clockwise)
  \item Reflective symmetry across three different axes
\end{itemize}
This gives a total of 6 symmetries.
Informally a symmetry is an action that does not change the overall shape.
\begin{remark}
  A regular $n$-gon has $2n$ symmetries.
\end{remark}
We can compose multiple symmetries to get composite symmetries.
If we label the vertices of the triangle we can identify which of the original symmetries the composite symmetry is equivalent to.

To invert a reflection you can just repeat the same reflection, they are ``self-inverse''.
\begin{remark}[Important Features]
  \begin{itemize}
    \item Symmetries can be \textbf{composed} to get another symmetry (\textbf{closed})
    \item There is an \textbf{identity} ``do-nothing'' symmetry
    \item Every symmetry has an \textbf{inverse}
    \item Composition of symmetries is \textbf{associative} ($a \circ (b \circ c) = (a \circ b) \circ c$)
  \end{itemize}
\end{remark}
\begin{remark}[Warning]
  Composition of symmetries is \textbf{not} always \textbf{commutative}.
  
  For example, if we rotate and then reflect it is not the same as if we reflect and then rotate.
\end{remark}
\subsection{An Algebraic Viewpoint}
\begin{definition}[Binary Operation]
  A \textit{binary operation} on a set $X$ is a function $\cdot: X \times X \to X$.
  This means for $x, y \in X$, $(x, y) \mapsto x \cdot y$
\end{definition}
\begin{definition}[Group]
  A \textit{group} is a triple $(G, \cdot, e)$ where:
  \begin{itemize}
    \item $G$ is a set
    \item $\cdot$ is a binary operation on $G$
    \item $e \in G$
  \end{itemize}
  such that the following 4 group axioms are satisfied:
  \begin{enumerate}
    \item \textbf{Closure -} For all $a, b \in G$, $a \cdot b \in G$
    \item \textbf{Associativity -} For all $a, b, c \in G$, $(a \cdot b) \cdot c = a \cdot (b \cdot c)$
    \item \textbf{(Right) Identity -} There exists an $e \in G$ such that for all $a \in G$, $a \cdot e = a$
    \item \textbf{(Right) Inverses -} For all $a \in G$ there exists $b \in G$ such that $a \cdot b = e$
  \end{enumerate}
\end{definition}
We noticed earlier that the symmetries of the equilateral triangle form a group.
We can also think about this definition as encompassing algebra with a single operation.
\begin{example}
  $(\Z, +, 0)$ forms a group.
  \begin{enumerate}
    \item \textbf{Closure -} For all $a, b \in \Z$, $a + b \in \Z$ \tick
    \item \textbf{Associativity -} We know that addition is associative \tick
    \item \textbf{Identity -} 0 is the identity as for all $a \in \Z$, $a + 0 = a$ \tick
    \item \textbf{Inverses -} The inverse of an element $a \in \Z$ is $-a$ (which is also in $\Z$) as $a + (-a) = 0$ \tick
  \end{enumerate}
\end{example}
The definition of a group has some important consequences:
\begin{proposition}
  Let $(G, \cdot, e)$ be a group and let $a, b, b', e' \in G$.
  \begin{enumerate}
    \item if $a \cdot b = e$ then $b \cdot a = e$. (Right inverses are left inverses)
    \item $e \cdot a = a$ (The right identity is also left identity)
    \item If $a, b, b'$ are such that $a \cdot b = e = a \cdot b'$ then $b = b'$ (Inverses are unique)
    \item If $a \cdot e' = a$ then $e' = e$ (The identity is unique)
  \end{enumerate}
\end{proposition}
\begin{proof}[\textbf{i}]
  Working from the assumption that $a \cdot b = e$:
  \begin{align*}
    a \cdot b &= e \\
    b \cdot (a \cdot b) &= b \cdot e \\
    b \cdot (a \cdot b) &= b \text{ by right identity} \\
    (b \cdot a) \cdot b &= b \text{ by associativity}
  \end{align*} 
  Now because of inverses $\exists c \in G$ such that $b \cdot c = e$:
  \begin{align*}
    ((b \cdot a) \cdot b) \cdot c &= b \cdot c \\
    (b \cdot a) \cdot (b \cdot c) & = b \cdot c \text{ by associativity}\\
    (b \cdot a) \cdot e &= e \text{ by right inverses}\\
    b \cdot a &= e
  \end{align*}
  So we have that $a \cdot b = e = b \cdot a$ and therefore right inverses are also left inverses.
\end{proof}
\begin{proof}[\textbf{ii}]
  From \textbf{i} we have that $a \cdot b = e = b \cdot a$:
  \begin{align*}
    a \cdot b &= b \cdot a \\
    a \cdot (a \cdot b) &= a \cdot (b \cdot a) \\
    a \cdot (a \cdot b) &= (a \cdot b) \cdot a \text{ by associativity}\\
    a \cdot e &= e \cdot a \text{ by inverses}\\
    a &= e \cdot a \text{ by right identity}
  \end{align*}
  So we have that $e \cdot a = a$ and therefore the identity is both a right and left identity.
\end{proof}
\begin{proof}[\textbf{iii}]
  Working from the assumption that $a \cdot b = e = a \cdot b'$:
  \begin{align*}
    a \cdot b &= a \cdot b' \\
    b \cdot (a\cdot b) &= b \cdot (a \cdot b') \\
    (b \cdot a) \cdot b &= (b \cdot a) \cdot b' \text{ by associativity}\\
    e \cdot b &= e \cdot b' \text{ by \textbf{i}} \\
    b &= b' \text{ by \textbf{ii}} 
  \end{align*} 
  So $b = b'$ and therefore inverses are unique.
\end{proof}
\begin{proof}[\textbf{iv}]
  Let $b$ be the inverse of $a$, working from the assumption that $a \cdot e' = a$:
  \begin{align*}
    a \cdot e' &= a \\
    b \cdot (a \cdot e') &= b \cdot a\\
    (b \cdot a) \cdot e' &= b \cdot a \text{ by associativity} \\
    e \cdot e' &= e \text{ by \textbf{i}} \\
    e' &= e
  \end{align*} 
  So $e = e'$ and therefore the identity is unique.
\end{proof}
\end{document}
