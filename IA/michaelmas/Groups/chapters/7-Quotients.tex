\documentclass[../main.tex]{subfiles}
\begin{document}
\chapter{Quotient Groups}
\section{Normal Subgroups}
\begin{definition}[Normal Subgroup]
  $H \leq G$ is called \textit{normal}, if:
  \[
    ghg^{-1} \in H
  \]
  for every $h \in H$ and $g \in G$.
  If so, we write $H \lhd G$.
\end{definition}
\begin{remark}[Intuition]
  A subgroup is normal if whenever it contains an element it also contains all of the conjugates of that elements.
  It is a union of conjugacy classes.
\end{remark}
\begin{example}
  \begin{enumerate}
    \item $1 \lhd G$ and $G \lhd G$ for all $G$.
    \item If $G$ is abelian and $H \leq G$, then $H \lhd G$.
      That is, for an abelian group, all subgroups are normal.
    \item $\langle r \rangle \lhd D_{2n}$ since:
      \begin{align*}
        r^{l}r^{k}r^{-l} &= r^{k} \in \langle r \rangle \\
        sr^{k}s^{-1} &= r^{-k} \in \langle r \rangle
      \end{align*}
      by the dihedral relation.
      But $\langle s \rangle$ is \textbf{not} normal, because:
      \[
        rsr^{-1} = sr^{-2} \notin \langle s \rangle
      \]
    \item Suppose $\phi: G \to G'$ is a homomorphism.
      If $h \in \ker \phi$, and $g \in G$ then:
      \begin{align*}
        \phi(ghg^{-1}) &= \phi(g)\phi(h)\phi(g^{-1}) \\
                       &= \phi(g)\phi(g)^{-1} \\
                       &= e
      \end{align*}
      So $ghg^{-1} \in \ker \phi$.
      Therefore $\ker \phi \lhd G$.
  \end{enumerate}
\end{example}
\end{document}
