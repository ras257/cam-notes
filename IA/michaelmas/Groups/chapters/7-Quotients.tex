\documentclass[../main.tex]{subfiles}
\begin{document}
\chapter{Quotient Groups}
\section{Normal Subgroups}
\begin{definition}[Normal Subgroup]
  $H \leq G$ is called \textit{normal}, if:
  \[
    ghg^{-1} \in H
  \]
  for every $h \in H$ and $g \in G$.
  If so, we write $H \lhd G$.
\end{definition}
\begin{remark}[Intuition]
  A subgroup is normal if whenever it contains an element it also contains all of the conjugates of that elements.
  Equivalently, a subgroup is normal if it is the union of the conjugacy classes of all of its elements in $G$.
\end{remark}
\begin{example}
  \begin{enumerate}
    \item $1 \lhd G$ and $G \lhd G$ for all $G$.
    \item If $G$ is abelian and $H \leq G$, then $H \lhd G$.
      That is, for an abelian group, all subgroups are normal.
    \item $\langle r \rangle \lhd D_{2n}$ since:
      \begin{align*}
        r^{l}r^{k}r^{-l} &= r^{k} \in \langle r \rangle \\
        sr^{k}s^{-1} &= r^{-k} \in \langle r \rangle
      \end{align*}
      by the dihedral relation.
      But $\langle s \rangle$ is \textbf{not} normal, because:
      \[
        rsr^{-1} = sr^{-2} \notin \langle s \rangle
      \]
    \item Suppose $\phi: G \to G'$ is a homomorphism.
      If $h \in \ker \phi$, and $g \in G$ then:
      \begin{align*}
        \phi(ghg^{-1}) &= \phi(g)\phi(h)\phi(g^{-1}) \\
                       &= \phi(g)\phi(g)^{-1} \\
                       &= e
      \end{align*}
      So $ghg^{-1} \in \ker \phi$.
      Therefore $\ker \phi \lhd G$.
  \end{enumerate}
\end{example}
\begin{lemma}
  \label{quotientCosets}
  Suppose $H \leq G$.
  Then $H \lhd G$ if and only if
  \[
    gH = Hg
  \]
  for all $g \in G$.
\end{lemma}
\begin{proof}
  \begin{proofdirection}{Assume $H \lhd G$}
    Let $h \in H$ and $g \in G$ since $H \lhd G$, $ghg^{-1} \in H$.
    Therefore:
    \[
      gh = (ghg^{-1})g \in Hg
    \]
    So $gH \subseteq Hg$.
    Similarly:
    \[
      hg = g(g^{-1} h g) =g(g^{-1} h(g^{-1})^{-1}) \in gH
    \]
    So $Hg \subseteq gH$.
    Therefore, $gH = Hg$.
  \end{proofdirection}
  \begin{proofdirection}{Assume $gH = Hg$}
    Let $h \in H$, then $gh \in gH  = Hg$. So there is a $h' \in H$ such that:
    \[
      gh = h'g \implies ghg^{-1} = h' \in H
    \]
    So $H \lhd G$.
  \end{proofdirection}
\end{proof}
\section{Quotient Groups}
\begin{theorem}[Quotient Groups]
  \label{quotientTheorem}
  If $H \lhd G$, the set of (left) cosets $G / H$ is a group with operation:
  \[
    (g_1H)(g_2H) = g_1g_2H
  \]
\end{theorem}
\begin{proof}
  We need to check that the operation is well defined and satisfies the group axioms.

  \textbf{Well Defined -} Suppose that $g_1 H = g_1' H$ and $g_2H = g_2'H$.
  By \cref{quotientCosets}, $g_2H = g_2'H \iff Hg_2 = Hg_2'$ as $H \lhd G$.
  So there are $h_1, h_2 \in H$ such that $g_1 = g_1'h_1$ and $g_2 = h_2 g_2'$.
  Therefore:
  \begin{align*}
    g_1g_2 &= g_1'h_1 h_2 g_2' \\
           &= g_1'(h_1 h_2) g_2' \\
           &= g_1' g_2' h_3 \text{ for $h_3 \in H$ as $Hg_2' = g_2'H$}
  \end{align*}
  Therefore, $g_1g_2 \in g_1'g_2'H$, since cosets partition, $g_1g_2H = g_1'g_2'H$.
  So for $g_1H = g_1'H$ and $g_2H = g_2'H$:
  \[
    (g_1H)(g_2H) = g_1g_2H = g_1'g_2'H = (g_1'H)(g_2'H)
  \]
  so the operation is well defined.

  \textbf{Associativity -} Apparent from definition and associativity of the operation of $G$.

  \textbf{Closure -} $g_1g_2H \in G / H$.

  \textbf{Identity -} Provided by the trivial coset, $H$.

  \textbf{Inverses}
  \[
    (gH)(g^{-1}H) = (gg^{-1})H = H
  \]
  so $(gH)^{-1} = g^{-1}H$.
\end{proof}
\begin{definition}[Quotient Groups]
  If $H \lhd G$, then the group $G / H$ provided by \cref{quotientTheorem} is the \textit{quotient} of $G$ by $H$.
\end{definition}
\begin{example}
  \begin{enumerate}
    \item $G / 1 \cong G$, $G / G \cong 1$.
    \item Since $\Z$ is abelian, $n\Z \lhd \Z$ for any $n$ and $\Z/n\Z$ is a cyclic group of order $n$ (provided $n \neq 0$) with generator $1 + n\Z$ so:
      \[
        \Z / n\Z \cong C_n
      \]
    \item Let $G$ be a group and $H \leq G$ and suppose that $|G:H| = 2$.
      Then for any $g \notin H$:
      \[
        gH = G \setminus H = Hg
      \]
      and $eH = He$ so $H \lhd G$ by \cref{quotientCosets}.
      Furthermore, there is exactly 2 cosets so $|G / H| = 2$.
      Therefore, $G /H \cong C_2$ as there is only one group of order 2 up to isomorphism.
    \item In particular, $C_n \cong \langle r \rangle$ and $\langle r \rangle \lhd D_{2n}$.
      Since $|D_{2n} : \langle r \rangle| = 2$:
      \[
        D_{2n} / C_n \cong C_2
      \]
  \end{enumerate}
\end{example}
\begin{remark}[Warning]
Notice that:
\[
  C_4 / C_2 \cong C_2 \text{ and } K_4 / C_2 \cong C_2
\]
but $C_4 \centernot\cong K_4$

In general, for groups $A, B, C$:
\[
  A / B \cong C \centernot\implies A \cong B \times C
\]
\end{remark}
\subsection{Isomorphism Theorem}
\begin{theorem}[First Isomorphism Theorem]
  \label{isomorphismTheorem}
  If $\phi: G \to H$ is a homomorphism, then:
  \[
    G /\ker\phi \cong \im \phi
  \]
\end{theorem}
\begin{proof}
  Since $\ker \phi \lhd G$, the quotient $G / \ker \phi$ is indeed a group.
  Define $\overline{\phi}: G / \ker \phi \to \im \phi$, $g \ker \phi \mapsto \phi(g)$.

  \textbf{Well Defined -} Suppose that $g_1 \ker \phi = g_2 \ker \phi$.
  Therefore for some $k \in \ker \phi$:
  \[
    g_1 = g_2 k
  \]
  Then we see that:
  \begin{align*}
    \overline{\phi}(g_1 \ker \phi) &= \phi(g_1) \\
                                   &= \phi(g_2 k) \\
                                   &= \phi(g_2)\phi(k) \text{ as $\phi$ is a homomorphism} \\
                                   &= \phi(g_2) \\
                                   &= \overline{\phi}(g_2 \ker \phi)
  \end{align*}
  So $\overline{\phi}$ is well defined.

  \textbf{Homomorphism -} For $g_1, g_2 \in G$.
  \begin{align*}
    \overline{\phi}((g_1 \ker \phi)(g_2 \ker \phi)) &= \overline{\phi}(g_1 g_2 \ker \phi) \\
                                                    &= \phi(g_1g_2) \\
                                                    &= \phi(g_1)\phi(g_2) \\
                                                    &= \overline{\phi}(g_1 \ker \phi)\overline{\phi}(g_2 \ker \phi)
  \end{align*}
  So $\overline{\phi}$ is a homomorphism.

  \textbf{Injectivity -} If $g \ker \phi \in \ker \overline{\phi}$ then $\overline{\phi}(g \ker \phi) = e$.
  Then:
  \[
    \overline{\phi}(g \ker \phi) = \phi(g) = e
  \]
  So $g \in \ker \phi$.
  Therefore,
  \[
    g \ker \phi = \ker \phi
  \]
  That is, $g \ker \phi \in \ker \overline{\phi} \implies g \ker \phi = \ker \phi$, the trivial coset.
  Therefore, $\ker \overline{\phi} = \{\ker \phi\}$ so $\ker \overline{\phi}$ is trivial and thus $\overline{\phi}$ is injective.

  \textbf{Surjectivity -} A typical element of $\im \phi$ is $\phi(g)$ for some $g \in G$.
  By definition:
  \[
    \phi(g) = \overline{\phi}(g \ker \phi)
  \]
  so $\overline{\phi}$ is surjective.

  Therefore, $\overline{\phi}$ is a well defined bijective homomorphism so $G / \ker \phi \cong \im \phi$
\end{proof}
\begin{example}
  \begin{enumerate}
    \item Because $\phi: \Z \to \C_\times$, $k \mapsto e^{\frac{2\pi i k}{n}}$ is a homomorphism with image $C_n$ and kernel $n\Z$, we get that:
      \[
        \Z / n\Z \cong \im \phi \cong C_n
      \]
    by \cref{isomorphismTheorem}.
  \item Similarly $\phi: \R \to \C$, $t \mapsto e^{2\pi i t}$ is a homomorphism with:
    \begin{align*}
      \im \phi &= \{z \in \C: |z| = 1\} = U(1) \\
      \ker \phi &= \Z
    \end{align*}
    Then, by \cref{isomorphismTheorem}:
    \[
      \R / \Z \cong U(1)
    \]
  \end{enumerate}
\end{example}
\begin{corollary}
  If $|G| < \infty$ and $\phi: G \to H$ is a homomorphism, then $|G| = |\im \phi||\ker \phi|$.
\end{corollary}
\begin{proof}
  By Lagrange's theorem (\cref{lagrangesTheorem}), $|G / \ker \phi||\ker \phi| = |G|$.
  From \cref{isomorphismTheorem}, $|G / \ker \phi| = |\im \phi|$.
  Thus:
  \[
    |G| = |\im \phi||\ker \phi|
  \]
\end{proof}
\end{document}
