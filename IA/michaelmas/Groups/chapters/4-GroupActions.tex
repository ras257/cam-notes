\documentclass[../main.tex]{subfiles}
\begin{document}
\chapter{Group Actions}
Groups become groups of symmetries when they ``act''.
\section{Definition and Examples}
\begin{definition}[Group Action]
  An \textit{action} of a group $G$ on a set $X$ is a map $G \times X \to X$, $(g, x) \mapsto gx$ such that:
  \begin{enumerate}
    \item $ex = x$ for all $x \in X$
    \item $(g \cdot h)x = g(hx)$ for all $g, h \in G$ and $x \in X$
  \end{enumerate}
  \label{actionDef}
\end{definition}
We write $G \acts X$.
\begin{example}
  \label{actionExamples}
  \begin{enumerate}
    \item For any group $G$ and set $X$, then $gx = x$ defines the \textit{trivial action}.
    \item $\Sym(X) \acts X$ for any set $X$ by $fx = f(x)$.
    \item If $G \acts X$ and $H \leq G$ then $H \acts X$ by just considering elements in $H$.
    \item In particular, $\Isom(\C) \leq \Sym(\C)$ acts on $\C$.
    \item Similarly, $D_{2n}$ acts on $X_n$ (the regular $n$-gon).
      It also acts on its set of vertices, $\{z \in \C : z^{n} = 1\}$.
    \item Every group $G$ acts \textit{on itself} by $g \gamma = g \cdot \gamma$.
      This is called the \textit{left regular action}.
  \end{enumerate}
\end{example}
\section{Actions and Symmetry}
\begin{theorem}
  An action of a group $G$ on a set $X$ is equivalent to a homomorphism $\phi : G \to \Sym(X)$.
  \label{actionHomTheorem}
\end{theorem}
\begin{proof}
  \begin{proofdirection}{Suppose $G \acts X$}
    Consider $t_g : X \to X$, $x \mapsto gx$ for any $g \in G$.
    Now
    \begin{align*}
      t_{g^{-1}}(t_g(x)) &= t_{g^{-1}}(gx) \\
                         &= g^{-1}(gx) \\
                         &= (g^{-1} \cdot g)x \text{ (by \cref{actionDef}, \textbf{ii})}\\
                         &= ex \\
                         &= x \text{ (by \cref{actionDef}, \textbf{i})}
    \end{align*}
    So $t_{g^{-1}} \circ t_g  = \id_X$.
    Similarly $t_g \circ t_{g^{-1}} = \id_X$ so $t_g$ is invertible and therefore a bijection.
    Thus $t_g$ is a permutation so $t_g \in \Sym(X)$.
    So we may define $\phi: G \to \Sym(X)$, $g \mapsto t_g$.

    We now need to prove that $\phi$ is a homomorphism for $g, h \in G$, $x \in X$.
    \[
      (t_g \circ t_h)(x) = t_g(t_h(x)) = t_g(hx) = g(hx) = (g \cdot h)x = t_{gh}(x)
    \]
    So $t_g \circ t_h = t_{gh}$.
    Therefore $\phi(g \cdot h) = t_{gh} = t_g \circ t_h = \phi(g)\phi(h)$ so $\phi$ is a homomorphism.
  \end{proofdirection}
  \begin{proofdirection}{Suppose we have a homomorphism $\phi: G \to \Sym(X)$}
    We may define an action of $G$ on $X$ by $gx = \phi(g)(x)$.
    We now need to check that this is an action:
    \begin{enumerate}
      \item $ex = \phi(e)(x) = \id_X(x) = x$
      \item $(gh)x = \phi(gh)(x) = (\phi(g) \circ \phi(h))(x) = \phi(g)(\phi(h)(x)) = \phi(g)(hx) = g(hx)$
    \end{enumerate}
    So it satisfies both axioms of \cref{actionDef}.
  \end{proofdirection}
\end{proof}
\begin{remark}[Intuition]
  This means that given a group action of $G$ on $X$ then there is a permutation representation of the action given by $\phi: G \to \Sym(X)$.
  That is, for any $g \in G$ there is a corresponding permutation of $X$, $\phi(g)$, such that $gx = \phi(g)(x)$.

  Similarly if we have a homomorphism $\phi: G \to \Sym X$, there is a corresponding group action of $G$ on $X$ given by $gx = \phi(g)(x)$.

  This correspondence is one-to-one.
\end{remark}
\begin{theorem}[Cayley's Theorem]
  Every group $G$ is isomorphic to a subgroup of some $\Sym(X)$.
  Furthermore, if $G$ is finite then we may choose $X$ to also be finite.
\end{theorem}
\begin{proof}
  Let $X = G$.
  \Cref{actionExamples} \textbf{vi} defines the regular action  of $G$ on $G$, given by $g\gamma = g \cdot \gamma$ for all $g, \gamma \in G$.
  By \cref{actionHomTheorem}, this is equivalent to a homomorphism $\phi: G \to \Sym(G)$.

  Let $H = \im \phi$.
  From \cref{imKerHom}, $H \leq \Sym G$.
  So we may think of $\phi$ as a surjective homomorphism $\phi: G \to H$.

  We now need to prove that $\phi$ is injective.
  By \cref{surjectiveInjectiveProp}, we can do this by proving that $\ker \phi  = \{e\}$.
  Take $g \in \ker \phi$ so $\phi(g) = \id_G$.
  Therefore:
  \[
    g \gamma = \phi(g)(\gamma) = \id_G(\gamma) = \gamma
  \]
  for all $\gamma \in G$.
  In particular, we can pick $\gamma = e$ so $ge = e$.
  But from the definition of the regular action, we also have $g\gamma = g \cdot \gamma$ so $ge = g \cdot e = g$.
  Thus $g = e$ so $\ker \phi = \{e\}$.
  Therefore $\phi$ is injective.

  Hence $\phi$ is bijective and a homomorphism so $G \cong H \leq \Sym(G) = \Sym(X)$  as required.
  Finally, since $G = X$ it is certainly true that $|G| < \infty \implies |X| < \infty$.
\end{proof}
\section{Orbits and Stabilisers}
We can learn a huge amount about groups by thinking about actions.
Orbits and stabilisers can help us do this.
\begin{definition}[Orbit]
  If $G \acts X$ and $x \in X$ then the \textit{orbit} of $x$ is the set:
  \[
    Gx = \{y \in X : \exists g \in G \text{ such that } y = gx\}
  \]
\end{definition}
\begin{definition}[Stabiliser]
  If $G \acts X$ and $x \in X$ then the \textit{stabiliser} of $x$ is the set:
  \[
    \Stab_G(x) = \{g \in G : gx = x\}
  \]
\end{definition}
\begin{remark}[Warning]
  Some sources write $G_x$ instead of $\Stab_G(x)$.
\end{remark}
\begin{definition}[Transitive]
  If $Gx = X$, then we say the action is \textit{transitive}.
\end{definition}
\begin{definition}[Faithful]
  If every element $g \in G$ except $e$ has an $x \in X$ such that $gx \neq x$ then the action is \textit{faithful}.
\end{definition}
\begin{remark}
  If $G$ acting on $X$ is faithful if the associated homomorphism $G \to \Sym(X)$ is injective.
\end{remark}
\begin{proposition}
  Suppose $G \acts X$ then:
  \begin{enumerate}
    \item $\Stab_G(x) \leq G$ for any $x \in X$.
    \item The set of all orbits $\{Gx : x \in X\}$ form a \textit{partition} of $X$.
  \end{enumerate}
\end{proposition}
\begin{proof}
  \begin{enumerate}
    \item We need to check that $\Stab_G(x)$ satisfies the definition of a subgroup.
      \begin{itemize}
        \item \textbf{Closure - }
          If $g, h \in \Stab_G(x)$ then:
          \begin{align*}
            (g \cdot h)x &= g(hx) \\
                    &= gx \text{ (Since $h \in \Stab_G(x)$)}\\
                    &= x \text{ (Since $g \in \Stab_G(x)$)}
          \end{align*}
        \item \textbf{Identity - } $ex = x$ by definition of an action so $e \in \Stab_G(x)$.
        \item \textbf{Inverses - }
          For $g \in \Stab_G(x)$,
          \begin{align*}
            g^{-1}x &= g^{-1}(gx) \\
                    &= (g^{-1} \cdot g)x \\
                    &= ex \\
                    &= x
          \end{align*}
          So $g^{-1} \in \Stab_G(x)$.
      \end{itemize}
      Thus $\Stab_G(x) \leq G$.
    \item Similarly to in the proof of \cref{cosetsCover}:

      Notice that $x = ex \in Gx$ so every element is in an orbit, therefore orbits cover $X$.

      If $Gx_1 \cap Gx_2 \neq \emptyset$ then there is some $y = g_1 x_1 = g_2 x_2$ for some $g_1, g_2 \in G$.
      Hence:
      \begin{align*}
        x_1 &= ex_1 \\
            &=(g^{-1}_{1} \cdot g_1) x_1 \\
            &= g^{-1}_1 (g_1 x_1) \\
            &= g^{-1}_1 (g_2 x_2) \text{ (As $g_1 x_1 = g_2 x_2$)} \\
            &= (g^{-1}_{1} \cdot g_2)x_2 \in Gx_2
      \end{align*}
      Now, any $gx_1$ can now be written as $gx_1 = g((g^{-1}_{1} \cdot g_2)x_2) = (g \cdot (g^{-1}_{1} \cdot g_2))x_2 \in Gx_2$.
      Hence $Gx_1 \subseteq Gx_2$.
      Likewise, by relabelling, $Gx_2 \subseteq Gx_1$.
      Thus $Gx_1 = Gx_2$.
      So every element is contained in exactly one orbit so the orbits partition $X$.
  \end{enumerate}
\end{proof}
\begin{remark}[Note]
  By \textbf{ii}, an action is transitive ($Gx = X$) if and only if there is only one orbit, so transitivity is independent of the choice of $x$.
\end{remark}
\begin{example}
  \label{dihedralAction}
  Consider $D_{2n}$ acting on $X_n$, the regular $n$-gon and let $x = 1$.
  Since $D_{2n}$ is a subset of $\Sym(\C)$, it acts by $fx = f(x)$.

  For any $j$, $r^{j}s(x) = r^{j}(x) = e^{\frac{2 \pi i j}{n}} \cdot 1 = e^{\frac{2\pi i j}{n}}$.
  So the action of $r^{j}$ or $r^{j}s$ (i.e. all elements of $D_{2n}$) on $x$ gives $e^{\frac{2 \pi i j}{n}}$.
  Therefore the orbit of $x$, $D_{2n}x = \{e^{\frac{2 \pi i j}{n}} : 0 \leq j < n\}$.

  The above calculation also shows that if $gx = x$ then $j = 0$ so $g$ is either $r^{0} = e$ or $r^{0}s = s$ so $\Stab_{D_{2n}}(x) = \{e, s\}$.
\end{example}
\begin{theorem}[Orbit-Stabilser Theorem]
  Suppose $G \acts X$ and let $x \in X$.
  The formula:
  \[
    g \Stab_G(x) \mapsto gx
  \]
  defines a well-defined bijection:
  \[
    G / \Stab_G(x) \to Gx
  \]
\end{theorem}
\begin{proof}
  For brevity, let $S = \Stab_G(x)$ and define $\Phi(gS) = gx$.
  We have to check several things about $\Phi$.

  Firstly, we need to check that $\Phi$ is \textit{well-defined}.
  That is, for any $g_1, g_2 \in G$ such that $g_1S = g_2S$ (i.e. they are in the same coset), we need to check that $g_1S$ and $g_2S)$ both get sent to the same thing under the assignment $\Phi$.

  To do this we need to check that $g_1x = g_2x$.
  Now, since $g_1$ and $g_2$ are in the same coset, there is an $s \in S$ such that $g_1 = g_2 \cdot s$.
  Since $s$ is in the stabiliser, $sx = x$ and so:
  \[
    g_1x = (g_2 \cdot s)(x) = g_2(sx) = g_2 x
  \]
  So $\Phi$ is well-defined.

  We now need to check that $\Phi$ is surjective.
  For any $gx \in Gx$,
  \[
    \phi(gS) = gx
  \]
  by definition, so $\Phi$ is surjection.

  We now need to check that $\Phi$ is injective.
  Suppose $\Phi(g_1S) = \Phi(g_2S)$, so $g_1x = g_2x$.
  Now we need to prove that $g_1S = g_2S$.
  Let $s = g^{-1}_{2} \cdot g_1$.
  \begin{align*}
    sx &= (g^{-1}_{2} \cdot g_1)x \\
       &= g^{-1}_{2}(g_1 x) \\
       &= g^{-1}_{2}(g_2 x) \text{ (as $g_1x = g_2x$)}\\
       &= (g^{-1}_{2} \cdot g_2)x \\
       &= ex \\
       &=x
  \end{align*}
  So $s \in S$.
  Therefore:
  \begin{align*}
    g_1 &= (g_2 \cdot g^{-1}_{2}) \cdot g_1 \\
        &= g_2 \cdot (g^{-1}_{2} \cdot g_1) \\
        &= g_2 \cdot s \in g_2S
  \end{align*}
  So $g_1 \in g_2S$ so the cosets $g_1S$ and $g_2S$ overlap but since cosets partition, it follows that $g_1S = g_2S$.
  So $\Phi$ is a well-defined bijection.
\end{proof}
\begin{corollary}
  If $G \acts X$ and $x \in X$ then:
  \[
    |G| = |Gx||\Stab_G(x)|
  \]
\end{corollary}
\begin{proof}
  Orbit-Stabiliser gives a bijection $G/\Stab_G(x) \to Gx$, therefore:
  \[
    |Gx| = |G / \Stab_G(x)|
  \]
  and by Lagrange's theorem (\cref{lagrangesTheorem}):
  \[
    |G| = |G/\Stab_G(x)||\Stab_G(x)|
  \]
  Thus $|G| = |Gx||\Stab_G(x)|$.
\end{proof}
\begin{example}
  Consider again $D_{2n}$ acting on $X_n$ from \cref{dihedralAction}.
  We saw that, if $x = 1$ then $|D_{2n}x| = n$ and $|\Stab_{D_{2n}}(x)| = 2$.
  So by Orbit-Stabiliser, $|D_{2n}| = 2n$.

  This proof is a circular at the moment as we used the group structure to determine the orbits and stabilisers but we could do this without knowing the group structure for an easier proof of the size of $D_{2n}$.
\end{example}
\begin{example}[Symmetries of a Cube]
  Let $G$ be the group of isometries of a cube and let $x$ be the centre of a face.

  The centre of a face can be sent to the centre of any of the other $6$ faces so the orbit of $x$ is the centres of all of the faces, so $|Gx| = 6$.

  To find the stabiliser of $x$, we need to think of isometries that fix the centre of the face, so they must send the face to itself.
  This is just the group of symmetries of a square, $D_{8}$, so $\Stab_G(x) \cong D_8$ and $|\Stab_G(x)| = 8$.

  Therefore $|G| = |Gx||\Stab_G(x)| = 6 \cdot 8 = 48$.
\end{example}
\end{document}
