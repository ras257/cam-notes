\documentclass[../main.tex]{subfiles}
\begin{document}
\chapter{Group Actions}
Groups become groups of symmetries when they ``act''.
\section{Definition and Examples}
\begin{definition}[Group Action]
  An \textit{action} of a group $G$ on a set $X$ is a map $G \times X \to X$, $(g, x) \mapsto gx$ such that:
  \begin{enumerate}
    \item $ex = x$ for all $x \in X$
    \item $(g \cdot h)x = g(hx)$ for all $g, h \in G$ and $x \in X$
  \end{enumerate}
  \label{actionDef}
\end{definition}
We write $G \acts X$.
\begin{example}
  \label{actionExamples}
  \begin{enumerate}
    \item For any group $G$ and set $X$, then $gx = x$ defines the \textit{trivial action}.
    \item $\Sym(X) \acts X$ for any set $X$ by $fx = f(x)$.
    \item If $G \acts X$ and $H \leq G$ then $H \acts X$ by just considering elements in $H$.
    \item In particular, $\Isom(\C) \leq \Sym(\C)$ acts on $\C$.
    \item Similarly, $D_{2n}$ acts on $X_n$ (the regular $n$-gon).
      It also acts on its set of vertices, $\{z \in \C : z^{n} = 1\}$.
    \item Every group $G$ acts \textit{on itself} by $g \gamma = g \cdot \gamma$.
      This is called the \textit{left regular action}.
  \end{enumerate}
\end{example}
\section{Actions and Symmetry}
\begin{theorem}
  An action of a group $G$ on a set $X$ is the same as a homomorphism $\phi : G \to \Sym(X)$.
  \label{actionHomTheorem}
\end{theorem}
\begin{proof}
  \begin{proofdirection}{Suppose $G \acts X$}
    Consider $t_g : X \to X$, $x \mapsto gx$ for any $g \in G$.
    Now
    \begin{align*}
      t_{g^{-1}}(t_g(x)) &= t_{g^{-1}}(gx) \\
                         &= g^{-1}(gx) \\
                         &= (g^{-1} \cdot g)x \text{ (by \cref{actionDef}, \textbf{ii})}\\
                         &= ex \\
                         &= x \text{ (by \cref{actionDef}, \textbf{i})}
    \end{align*}
    So $t_{g^{-1}} \circ t_g  = \id_X$.
    Similarly $t_g \circ t_{g^{-1}} = \id_X$ so $t_g$ is invertible and therefore a bijection.
    Thus $t_g$ is a permutation so $t_g \in \Sym(X)$.
    So we may define $\phi: G \to \Sym(X)$, $g \mapsto t_g$.

    We now need to prove that $\phi$ is a homomorphism for $g, h \in G$, $x \in X$.
    \[
      (t_g \circ t_h)(x) = t_g(t_h(x)) = t_g(hx) = g(hx) = (g \cdot h)x = t_{gh}(x)
    \]
    So $t_g \circ t_h = t_{gh}$.
    Therefore $\phi(g \cdot h) = t_{gh} = t_g \circ t_h = \phi(g)\phi(h)$ so $\phi$ is a homomorphism.
  \end{proofdirection}
  \begin{proofdirection}{Suppose we have a homomorphism $\phi: G \to \Sym(X)$}
    We may define an action of $G$ on $X$ by $gx = \phi(g)(x)$.
    We now need to check that this is an action:
    \begin{enumerate}
      \item $ex = \phi(e)(x) = \id_X(x) = x$
      \item $(gh)x = \phi(gh)(x) = (\phi(g) \circ \phi(h))(x) = \phi(g)(\phi(h)(x)) = \phi(g)(hx) = g(hx)$
    \end{enumerate}
    So it satisfies both axioms of \cref{actionDef}.
  \end{proofdirection}
\end{proof}
\begin{theorem}[Cayley's Theorem]
  Every group $G$ is isomorphic to a subgroup of some $\Sym(X)$.
  Furthermore, if $G$ is finite then we may choose $X$ to also be finite.
\end{theorem}
\begin{proof}
  Let $X = G$.
  \Cref{actionExamples} \textbf{vi} defines the regular action  of $G$ on $G$.
  By \cref{actionHomTheorem}, this is equivalent to a homomorphism $\phi: G \to \Sym(G)$.

  Let $H = \im \phi$.
  From \cref{imKerHom}, $H \leq \Sym G$.
  So we may think of $\phi$ as a surjective homomorphism $\phi: G \to H$.

  We now need to prove that $\phi$ is injective.
  By \cref{surjectiveInjectiveProp}, we can do this by proving that $\ker \phi  = \{e\}$.
  Take $g \in \ker \phi$ so $\phi(g) = \id_G$.
  Therefore $g \gamma = \id_G(\gamma) = \gamma$ for all $\gamma \in G$.
  In particular, we can pick $\gamma = e$ so $ge = e$.
  By \cref{actionDef} \textbf{i} and uniqueness of the identity, $ge = e \implies g = e$.
  Thus $\phi$ is injective.

  Hence $\phi$ is bijective and a homomorphism so $G \cong H \leq \Sym(G) = \Sym(X)$  as required.
  Finally, since $G = X$ it is certainly true that $|G| < \infty \implies |X| < \infty$.
\end{proof}
\section{Orbits and Stabilisers}
We can learn a huge amount about groups by thinking about actions.
Orbits and stabilisers can help us do this.
\begin{definition}[Orbit]
  If $G \acts X$ and $x \in X$ then the \textit{orbit} of $x$ is the set:
  \[
    Gx = \{y \in X : \exists g \in G \text{ such that } y = gx\}
  \]
\end{definition}
\begin{definition}[Stabiliser]
  If $G \acts X$ and $x \in X$ then the \textit{stabiliser} of $x$ is the set:
  \[
    \Stab_G(x) = \{g \in G : gx = x\}
  \]
\end{definition}
\begin{remark}[Warning]
  Some sources write $G_x$ instead of $\Stab_G(x)$.
\end{remark}
\begin{definition}[Transitive]
  If $Gx = X$, then we say the action is \textit{transitive}.
\end{definition}
\begin{definition}[Faithful]
  If every element $g \in G$ except $e$ has an $x \in X$ such that $gx \neq x$ then the action is \textit{faithful}.
\end{definition}
\end{document}
