\documentclass[../main.tex]{subfiles}
\begin{document}
\chapter{Partial Differentiation}
\section{Functions of Several Variables}
Our goal is to generalise differentiation to functions of more than one independent variables, that is, \textit{multivariate functions}.
\begin{example}
  Examples of when we might need a multivariate function:
  \begin{itemize}
    \item Height of terrain, $h(x, y)$
    \item Temperature in room, $T(x, y, z, t)$
    \item Pressure of a gas as a function of volume and temperature
  \end{itemize}
\end{example}
To draw a contour plot you draw contours where $f(x, y) = \text{constant}$, like a topographic map:

\begin{center}
\begin{tikzpicture}[scale=2]
  \draw[->] (0,0) -- (2.5,0) node[right] {$x$};
  \draw[->] (0,0) -- (0,2.5) node[above] {$y$};

  \foreach \r/\label in {0.5/0.25, 1/1, 1.5/2.25, 2/4} {
    \draw (0,\r) arc[start angle=90, end angle=0, radius=\r];
    \node at ({\r/sqrt(2) + 0.1}, {\r/sqrt(2) + 0.1}) {\small $\label$};
  }

  \filldraw[black] (0.7, 1.5) circle (0.03) node[above right] {$A$};
  \node[below left] at (0,0) {$0$};
\end{tikzpicture}
\end{center}

The slope at the point $A$ depends on the direction.
\section{Partial Derivatives}
Generally speaking, partial derivatives are the derivatives of a multivariate function with respect to a single variable whilst holding other variables fixed.
\begin{definition}[Partial Derivative]
  Given a function of several variables, e.g. $f(x, y)$, the \textit{partial derivative} of $f$ with respect to $x$ at fixed $y$ is:
  \[
    \at{\pderiv{f}{x}}{y} = \lim_{\delta x \to 0} \frac{f(x + \delta x, y) - f(x, y)}{\delta x}
  \]
  This is the slope of $f$ when moving in the positive $x$ direction whilst keeping $y$ fixed.
\end{definition}
\begin{example}
  \begin{align*}
    f(x, y) &= x^2 + y^3 + e^{xy^2} \\
    \at{\pderiv{f}{x}}{y} &= 2x+0+y^2e^{xy^2} \\
    \at{\pderiv{f}{y}}{x} &= 0 + 3y^2 + 2xye^{xy^2}\\
  \end{align*}
  We can also take higher order derivatives similarly to normal derivatives.
  \[
    \at{\pderiv[2]{f}{x}}{y} = 2 +y^4e^{xy^2}
  \]
  We can also take mixed derivatives:
  \begin{align*}
    \at{\pderiv{}{x}\left(\at{\pderiv{f}{y}}{x}\right)}{y} &= \frac{\partial^2 f}{\partial x \partial y} =2ye^{xy^2} + 2xy^3 e^{xy^2} \\
    \at{\pderiv{}{y}\left(\at{\pderiv{f}{x}}{y}\right)}{x} &= \frac{\partial^2 f}{\partial y \partial x} =2ye^{xy^2} + 2xy^3 e^{xy^2}
  \end{align*}
\end{example}
\begin{remark}[Notation]
  We usually omit the evaluated at bar $\at{}{x}$ and implicitly assume that all other variables are being held fixed.
\end{remark}
\begin{remark}[Alternative Notation]
  \[
    f_x \equiv \pderiv{f}{x}, f_{xy} \equiv \frac{\partial^2 f}{\partial y \partial x}
  \]
  In the second case the derivative with respect to $x$ is carried out \textbf{first}.
\end{remark}
\begin{theorem}[Schwarz's Theorem]
  If $f$ has continuous mixed 2nd derivatives then:
  \[
    \frac{\partial^2 f}{\partial y \partial x} = \frac{\partial^2 f}{\partial x \partial y}
  \]
  That is, partial derivatives commute.
\end{theorem}
\begin{remark}
  If we have $f(x, y, z)$ then:
  \[
    \pderiv{f}{x} \equiv \at{\pderiv{f}{x}}{y, z} \neq \at{\pderiv{f}{x}}{y} \text{(in general)}
  \]
\end{remark}
\section{Multivariate Chain Rule}
Given a path $x(t), y(t)$ and $f(x, y)$ what is $\deriv{f}{t}$ along the path?
Consider the change in $f$ under a small change in $(x, y)$, that is $(x, y) \mapsto (x + \delta x, y + \delta y)$.
\begin{align*}
  \delta f &= f(x + \delta x, y + \delta y) - f(x, y) \\
           &= [f(x + \delta x, y + \delta y) - f(x + \delta x, y)] + [f(x + \delta x, y) - f(x, y)]
\end{align*}
Using \cref{taylorsThm} (Taylor's Theorem) as $\delta x \to 0$:
\[
  f(x + \delta x, y) - f(x, y) = f_x(x, y) \delta x + o(\delta x)
\]
Again using \cref{taylorsThm} first as $\delta y \to 0$ then again as $\delta x \to 0$:
\begin{align*}
  f(x + \delta x, y + \delta y) - f(x + \delta x, y) &= f_y(x + \delta x, y)\delta y + o(\delta y) \\
                                                     &= [f_y(x, y) + f_{yx}(x, y)\delta x + o(\delta x)] \delta y + o(\delta y)
\end{align*}
Substituting back into the original expression for $\delta f$:
\begin{align*}
  \delta f &= [f_y(x, y) + f_{yx}(x,y)\delta x + o(\delta x)]\delta y + f_x(x, y)\delta x + o(\delta y) + o(\delta x) \\
           &= f_x(x, y)\delta x + f_y(x, y)\delta y + f_{yx}(x, y)(\delta x)(\delta y) + o(\delta x)\delta y + o(\delta y) + o(\delta x) \\
           &= f_x(x, y)\delta x + f_y(x, y)\delta y + o(\delta x, \delta y)
\end{align*}
Taking the limit as $\delta x, \delta y \to 0$ yields:
\begin{theorem}[Differential Form of the Chain Rule for Partial Derivatives]
  The differential $\d{f}$ of $f(x, y)$ is:
  \[
    \d{f} = \pderiv{f}{x}\d{x} + \pderiv{f}{y}\d{y}
  \]
\end{theorem}
So for the path $x(t), y(t)$:
\begin{align*}
  \deriv{}{t}(f(x(t), y(t))) &= \lim_{\delta x, \delta y, \delta t \to 0} \left[\pderiv{f}{x} \frac{\delta x}{\delta t} + \pderiv{f}{y} \frac{\delta y}{\delta t}\right] \\
  &= \pderiv{f}{x} \deriv{x}{t} + \pderiv{f}{y} \deriv{y}{t}
\end{align*}
If we instead parametrise the path by the $x$ coordinate then:
\[
  \deriv{}{x}f(x, y(x)) = \pderiv{f}{x} \underbrace{\deriv{}{x}(x)}_{=1} + \pderiv{f}{y} \deriv{y}{x} = \pderiv{f}{x} + \pderiv{f}{y}\deriv{y}{x}
\]
\begin{theorem}[Integral Form of the Multivariate Chain Rule]
  \[
    \Delta f = \int  \d{f} = \int \deriv{f}{x} \d{x} + \int \pderiv{f}{y} \d{y}
  \]
  Where $\Delta f$ is the change in $f$ between the endpoints of the path.
\end{theorem}
For $f(x(t), y(t))$:
\[
  \Delta f = \int \left(\pderiv{f}{x}\deriv{x}{t} + \pderiv{f}{y}\deriv{y}{t}\right) \d{t} = \int \deriv{f}{t} \d{t}
\]
This means that we will just get the difference between the values of $f$ at the endpoints and thus the result does not depend on the particular path for a given pair of endpoints.
\section{Applications of the Multivariate Chain Rule}
\subsection{Change in Variables}
It is often useful to write a differential equation in a different coordinate system.

\end{document}
