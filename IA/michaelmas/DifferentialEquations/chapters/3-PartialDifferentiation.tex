\documentclass[../main.tex]{subfiles}
\begin{document}
\chapter{Partial Differentiation}
\section{Functions of Several Variables}
Our goal is to generalise differentiation to functions of more than one independent variables, that is, \textit{multivariate functions}.
\begin{example}
  Examples of when we might need a multivariate function:
  \begin{itemize}
    \item Height of terrain, $h(x, y)$
    \item Temperature in room, $T(x, y, z, t)$
    \item Pressure of a gas as a function of volume and temperature
  \end{itemize}
\end{example}
To draw a contour plot you draw contours where $f(x, y) = \text{constant}$, like a topographic map:

\begin{center}
\begin{tikzpicture}[scale=2]
  \draw[->] (0,0) -- (2.5,0) node[right] {$x$};
  \draw[->] (0,0) -- (0,2.5) node[above] {$y$};

  \foreach \r/\label in {0.5/0.25, 1/1, 1.5/2.25, 2/4} {
    \draw (0,\r) arc[start angle=90, end angle=0, radius=\r];
    \node at ({\r/sqrt(2) + 0.1}, {\r/sqrt(2) + 0.1}) {\small $\label$};
  }

  \filldraw[black] (0.7, 1.5) circle (0.03) node[above right] {$A$};
  \node[below left] at (0,0) {$0$};
\end{tikzpicture}
\end{center}

The slope at the point $A$ depends on the direction.
\section{Partial Derivatives}
Generally speaking, partial derivatives are the derivatives of a multivariate function with respect to a single variable whilst holding other variables fixed.
\begin{definition}[Partial Derivative]
  Given a function of several variables, e.g. $f(x, y)$, the \textit{partial derivative} of $f$ with respect to $x$ at fixed $y$ is:
  \[
    \at{\pderiv{f}{x}}{y} = \lim_{\delta x \to 0} \frac{f(x + \delta x, y) - f(x, y)}{\delta x}
  \]
  This is the slope of $f$ when moving in the positive $x$ direction whilst keeping $y$ fixed.
\end{definition}
\begin{example}
  \begin{align*}
    f(x, y) &= x^2 + y^3 + e^{xy^2} \\
    \at{\pderiv{f}{x}}{y} &= 2x+0+y^2e^{xy^2} \\
    \at{\pderiv{f}{y}}{x} &= 0 + 3y^2 + 2xye^{xy^2}\\
  \end{align*}
  We can also take higher order derivatives similarly to normal derivatives.
  \[
    \at{\pderiv[2]{f}{x}}{y} = 2 +y^4e^{xy^2}
  \]
  We can also take mixed derivatives:
  \begin{align*}
    \at{\pderiv{}{x}\left(\at{\pderiv{f}{y}}{x}\right)}{y} &= \frac{\partial^2 f}{\partial x \partial y} =2ye^{xy^2} + 2xy^3 e^{xy^2} \\
    \at{\pderiv{}{y}\left(\at{\pderiv{f}{x}}{y}\right)}{x} &= \frac{\partial^2 f}{\partial y \partial x} =2ye^{xy^2} + 2xy^3 e^{xy^2}
  \end{align*}
\end{example}
\begin{remark}[Notation]
  We usually omit the evaluated at bar $\at{}{x}$ and implicitly assume that all other variables are being held fixed.
\end{remark}
\begin{remark}[Alternative Notation]
  \[
    f_x \equiv \pderiv{f}{x}, f_{xy} \equiv \frac{\partial^2 f}{\partial y \partial x}
  \]
  In the second case the derivative with respect to $x$ is carried out \textbf{first}.
\end{remark}
\begin{theorem}[Schwarz's Theorem]
  If $f$ has continuous mixed 2nd derivatives then:
  \[
    \frac{\partial^2 f}{\partial y \partial x} = \frac{\partial^2 f}{\partial x \partial y}
  \]
  That is, partial derivatives commute.
\end{theorem}
\begin{remark}
  If we have $f(x, y, z)$ then:
  \[
    \pderiv{f}{x} \equiv \at{\pderiv{f}{x}}{y, z} \neq \at{\pderiv{f}{x}}{y} \text{(in general)}
  \]
\end{remark}
\section{Multivariate Chain Rule}
Given a path $x(t), y(t)$ and $f(x, y)$ what is $\deriv{f}{t}$ along the path?
Consider the change in $f$ under a small change in $(x, y)$, that is $(x, y) \mapsto (x + \delta x, y + \delta y)$.
\begin{align*}
  \delta f &= f(x + \delta x, y + \delta y) - f(x, y) \\
           &= [f(x + \delta x, y + \delta y) - f(x + \delta x, y)] + [f(x + \delta x, y) - f(x, y)]
\end{align*}
Using \cref{taylorsThm} (Taylor's Theorem) as $\delta x \to 0$:
\[
  f(x + \delta x, y) - f(x, y) = f_x(x, y) \delta x + o(\delta x)
\]
Again using \cref{taylorsThm} first as $\delta y \to 0$ then again as $\delta x \to 0$:
\begin{align*}
  f(x + \delta x, y + \delta y) - f(x + \delta x, y) &= f_y(x + \delta x, y)\delta y + o(\delta y) \\
                                                     &= [f_y(x, y) + f_{yx}(x, y)\delta x + o(\delta x)] \delta y + o(\delta y)
\end{align*}
Substituting back into the original expression for $\delta f$:
\begin{align*}
  \delta f &= [f_y(x, y) + f_{yx}(x,y)\delta x + o(\delta x)]\delta y + f_x(x, y)\delta x + o(\delta y) + o(\delta x) \\
           &= f_x(x, y)\delta x + f_y(x, y)\delta y + f_{yx}(x, y)(\delta x)(\delta y) + o(\delta x)\delta y + o(\delta y) + o(\delta x) \\
           &= f_x(x, y)\delta x + f_y(x, y)\delta y + o(\delta x, \delta y)
\end{align*}
Taking the limit as $\delta x, \delta y \to 0$ yields:
\begin{theorem}[Differential Form of the Chain Rule for Partial Derivatives]
  The differential $\d{f}$ of $f(x, y)$ is:
  \[
    \d{f} = \pderiv{f}{x}\d{x} + \pderiv{f}{y}\d{y}
  \]
\end{theorem}
So for the path $x(t), y(t)$:
\begin{align*}
  \deriv{}{t}(f(x(t), y(t))) &= \lim_{\delta x, \delta y, \delta t \to 0} \left[\pderiv{f}{x} \frac{\delta x}{\delta t} + \pderiv{f}{y} \frac{\delta y}{\delta t}\right] \\
  &= \pderiv{f}{x} \deriv{x}{t} + \pderiv{f}{y} \deriv{y}{t}
\end{align*}
If we instead parametrise the path by the $x$ coordinate then:
\[
  \deriv{}{x}f(x, y(x)) = \pderiv{f}{x} \underbrace{\deriv{}{x}(x)}_{=1} + \pderiv{f}{y} \deriv{y}{x} = \pderiv{f}{x} + \pderiv{f}{y}\deriv{y}{x}
\]
\begin{theorem}[Integral Form of the Multivariate Chain Rule]
  \[
    \Delta f = \int  \d{f} = \int \deriv{f}{x} \d{x} + \int \pderiv{f}{y} \d{y}
    \label{multiChainRule}
  \]
  Where $\Delta f$ is the change in $f$ between the endpoints of the path.
\end{theorem}
For $f(x(t), y(t))$:
\[
  \Delta f = \int \left(\pderiv{f}{x}\deriv{x}{t} + \pderiv{f}{y}\deriv{y}{t}\right) \d{t} = \int \deriv{f}{t} \d{t}
\]
This means that we will just get the difference between the values of $f$ at the endpoints and thus the result does not depend on the particular path for a given pair of endpoints.
\section{Applications of the Multivariate Chain Rule}
\subsection{Change in Variables}
It is often useful to write a differential equation in a different coordinate system.
For example, it might be useful to convert from Cartesian $(x, y)$ to polar $(r, \theta)$.

We can do this by thinking of a function $f(x, y)$ as $f(x(r, \theta), y(r, \theta))$.
Then from the multi-variate chain rule (\cref{multiChainRule}):
\begin{align*}
  \at{\pderiv{f}{r}}{\theta} &= \at{\pderiv{f}{x}}{y} \at{\pderiv{x}{r}}{\theta} + \at{\pderiv{f}{y}}{x} \at{\pderiv{y}{r}}{\theta} \\
  \at{\pderiv{f}{\theta}}{r} &= \at{\pderiv{f}{x}}{y} \at{\pderiv{x}{\theta}}{r} + \at{\pderiv{f}{y}}{x} \at{\pderiv{y}{\theta}}{r}
\end{align*}
\subsection{Implicit Differentiation}
Consider the surface in 3D space given by $f(x, y, z) = \text{constant}$.
This implicitly defines $z = z(x, y)$, $y = y(x, z)$, $x = x(y, z)$ but we may not be able to find these forms explicitly.
\begin{example}
\begin{equation}
  xy + y^2 z + z^5 = 1
  \label{implicitEx}
\end{equation}
We cannot find $z$ as a function of $x$ and $y$.
But we can still evaluate partial derivatives.
If we take the derivative of \cref{implicitEx} with respect to $x$ holding $y$ fixed we get:
\[
  y + y^2 \at{\pderiv{z}{x}}{y} + 5z^4 \at{\pderiv{z}{x}}{y} = 0
\]
and so:
\[
  \at{\pderiv{z}{x}}{y} = -\frac{y}{y^2 + 5z^4}
\]
\end{example}
\begin{remark}[Warning]
  In cases like these we need to be clear on what variables are being held fixed.
\end{remark}
In general, if $f(x, y, z) = \text{constant}$ then the multi-variate chain rule gives:
\[
  0 = \d{f} = \at{\pderiv{f}{x}}{y, z} \d{x} + \at{\pderiv{f}{y}}{x, z} \d{y} + \at{\pderiv{f}{z}}{x, y} \d{z}
\]
\begin{remark}[Warning]
We can't vary $x$, $y$, and $z$ independently and stay on the surface.
\end{remark}
If we want to find $\at{(\partial z/\partial x)}{y}$ we first start by differentiating both sides with respect to $x$ holding $y$ fixed:
\begin{align*}
  0 = \at{\pderiv{f}{x}}{y} &= \at{\pderiv{f}{x}}{y,z} \underbrace{\at{\pderiv{x}{x}}{y}}_{=1} + \at{\pderiv{f}{y}}{x,z} \underbrace{\at{\pderiv{y}{x}}{y}}_{=0} + \at{\pderiv{f}{z}}{x, y} \at{\pderiv{z}{x}}{y} \\
                            &=\at{\pderiv{f}{x}}{y,z} + \at{\pderiv{f}{z}}{x, y} \at{\pderiv{z}{x}}{y} \\
\end{align*}
We can now rearrange to get:
\[
  \at{\pderiv{z}{x}}{y} = -\frac{\at{(\partial f/\partial x)}{y, z}}{\at{(\partial f/\partial z)}{x, y}}
\]
We can carry out a similar process to get:
\[
  \at{\pderiv{x}{y}}{z} = -\frac{\at{(\partial f/\partial y)}{x, z}}{\at{(\partial f/\partial x)}{y, z}}
\]
and:
\[
  \at{\pderiv{y}{z}}{x} = -\frac{\at{(\partial f/\partial z)}{x, y}}{\at{(\partial f/\partial y)}{x, z}}
\]
From this it follows that for any arbitrary surface with $f(x, y, z) = c$ that:
\[
  \at{\pderiv{x}{y}}{z} \at{\pderiv{y}{z}}{x} \at{\pderiv{z}{x}}{y} = -1
\]
\begin{proposition}[Reciprocal Rule]
The reciprocal rule for derivatives applies if same variables are being held fixed:
\[
  \at{\pderiv{x}{z}}{y} = -\frac{\at{(\partial f/\partial z)}{x, y}}{\at{(\partial f/\partial x)}{y, z}} = \frac{1}{\at{(\partial x/\partial z)}{y}}
\]
\end{proposition}
\begin{remark}[Warning]
  To apply the reciprocal rule you need to ensure that the same variables are being held fixed.
  For example, in general:
  \[
    \at{\pderiv{r}{x}}{y} \neq \frac{1}{\at{(\partial x/\partial r)}{\theta}}
  \]
\end{remark}
\section{Differentiation Under the Integral Sign}
Consider a family of functions $f(x, c)$ where $c$ is a parameter.
Define:
\[
  I(c) = \int_{a(c)}^{b(c)} f(x, c) \d{x}
\]
\begin{theorem}[Differentiation Under The Integral Sign - D.U.T.I.S]
  \[
    \deriv{I}{c} = \int_{a(c)}^{b(c)} \pderiv{f}{c} \d{x} + f(b(c), c) \deriv{b}{c} - f(a(c), c) \deriv{a}{c}
    \label{dutis}
  \]
\end{theorem}
\begin{proof}
  From the definition of the derivative we have that:
  \begin{align*}
    \deriv{I}{c} &= \lim_{\delta c \to 0} \frac{1}{\delta c} \left[\int_{a(c + \delta c)}^{b(c + \delta c)} f(x, c + \delta c) \d{x} - \int_{a(c)}^{b(c)} f(x, c) \d{x}\right] \\
  \end{align*}
  The first integral can be split up into three integrals:
  \[
    \int_{a(c + \delta c)}^{b(c + \delta c)} = \int_{a(c)}^{b(c)} + \int_{b(c)}^{b(c + \delta c)} - \int_{a(c)}^{a(c + \delta c)}
  \]
We can first handle the integrals with matching bounds:
\begin{align*}
  &\lim_{\delta c \to 0} \frac{1}{\delta c} \left[\int_{a(c)}^{b(c)} f(x, c + \delta c) \d{x} - \int_{a(c)}^{b(c)} f(x, c) \d{x}\right] \\
  &= \lim_{\delta \to 0} \frac{1}{\delta c} \int_{a(c)}^{b(c)} (f(x, c + \delta c) - f(x, c)) \d{x} \\
  &= \int_{a(c)}^{n(c)} \lim_{\delta c \to 0} \frac{f(x, c + \delta c) - f(x, c)}{\delta c} \d{x} \\
  &= \int_{a(c)}^{b(c)} \pderiv{f}{c} \d{x}
\end{align*}
Now for the integral from $b(c)$ to $b(c + \delta c)$, by the Mean Value Theorem (\cref{meanValue}) we have that for some $b(c) \leq \bar{x} \leq b(c + \delta c)$:
\[
  \int_{b(c)}^{b(c + \delta c)} f(x, c + \delta c) \d{x} = (b(c + \delta c) - b(c))f(\bar{x}, c + \delta c)
\]
so:
\begin{align*}
  \lim_{\delta c \to 0} \frac{1}{\delta c} \left[\int_{b(c)}^{b(c + \delta c)} f(x, c + \delta c) \d{x}\right] &= \lim_{\delta c \to 0} \frac{1}{\delta c} \left[(b(c + \delta c) - b(c))f(\bar{x}, c + \delta c)\right] \\
  &= \left(\lim_{\delta c \to 0} \frac{b(c + \delta c) - b(c)}{\delta c} \right)\left(\lim_{\delta c \to 0} f(\bar{x}, c + \delta c)\right) \\
  &= \deriv{b}{c} f(b(c), c)
\end{align*}
We can apply a similar process for the other remaining integral yielding:
\[
  \lim_{\delta c \to 0} \frac{1}{\delta c} \left[\int_{a(c)}^{a(c + \delta c)} f(x, c + \delta c) \d{x}\right] = \deriv{a}{c} f(a(c), c)
\]
Combining all of the above integrals gives the desired result.
\end{proof}

\begin{example}
  Consider the integral:
  \[
    I(\lambda) = \int_{0}^{\lambda} e^{-\lambda x^2} \d{x}
  \]
  We can now apply \cref{dutis}:
  \begin{align*}
    \deriv{I}{\lambda} &= \int_{0}^{\lambda} \pderiv{}{\lambda}\left(e^{-\lambda x^2}\right) \d{x} + e^{-\lambda \cdot (\lambda)^2} \deriv{}{\lambda}(\lambda) \\
                       &= \int_{0}^{\lambda} x^2 e^{-\lambda x^2} \d{x} + e^{-\lambda^3}
  \end{align*}
\end{example}
\begin{example}
  Suppose we want to evaluate:
  \[
    J_n = \int_{0}^{\infty} x^{n}e^{-x} \d{x}
  \]
  For $\lambda > 0$, let
  \[
    I(\lambda) = \int_{0}^{\infty} e^{-\lambda x} \d{x} = \frac{1}{\lambda}
  \]
  Differentiating both forms of $I(\lambda)$ $n$ times gives:
  \[
    \deriv[n]{I}{\lambda} = \int_{0}^{\infty} (-x)^{n} e^{-\lambda x} \d{x} = (-1)^{n}\frac{n!}{\lambda^{n+1}}
  \]
  Now we an set $\lambda = 1$ to get $J_n$:
  \begin{align*}
    (-1)^{n} J_n &= (-1)^{n} n! \\
    J_n &= n! \\
  \end{align*}
\end{example}
\end{document}
