\documentclass[../main.tex]{subfiles}
\begin{document}
\chapter{Elementary Number Theory}
\section{Primes}
\begin{definition}[Divisibility]
  Give $a, b \in \Z$, we say that ``a divides b'' if $\exists c \in \Z$ such that $b = ac$.
  We might also say that ``$a$ is a divisor of b'', ``$a$ is a factor of $b$'' or ``$b$ is a multiple of $a$''.
  We denote this $a \mid b$.
\end{definition}
For any $b \in \Z$, $\pm1$ and $\pm b$ are always factors.
All other factors of $b$ are called \textit{proper factors} (or \textit{non-trivial}).
\begin{definition}[Prime Number]
  A natural number $n \geq 2$ is \textit{prime} if it has no proper factors, that is, its only factors are $\pm 1$ and $\pm n$.
  If $n \geq 2$ is not prime then it is \textit{composite}.
\end{definition}
\end{document}
