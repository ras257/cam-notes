\documentclass[../main.tex]{subfiles}
\begin{document}
\chapter{Elementary Number Theory}
\section{Primes}
\begin{definition}[Divisibility]
  Give $a, b \in \Z$, we say that ``a divides b'' if $\exists c \in \Z$ such that $b = ac$.
  We might also say that ``$a$ is a divisor of b'', ``$a$ is a factor of $b$'' or ``$b$ is a multiple of $a$''.
  We denote this $a \mid b$.
\end{definition}
For any $b \in \Z$, $\pm1$ and $\pm b$ are always factors.
All other factors of $b$ are called \textit{proper factors} (or \textit{non-trivial}).
\begin{definition}[Prime Number]
  A natural number $n \geq 2$ is \textit{prime} if it has no proper factors, that is, its only factors are $\pm 1$ and $\pm n$.
  If $n \geq 2$ is not prime then it is \textit{composite}.
\end{definition}
\begin{proposition}
  Every natural number $n \geq 2$ can be written as a product of primes.
  \label{prodPrimes}
\end{proposition}
\begin{proof}
  \induction
  {$n = 2$}{
    Clearly true for $2$ as $2$ is prime.
  }
  {$n < k$}{}
  {$n = k$}{
     If $k$ itself is prime then we are done.

     Otherwise, if $k$ is composite we can write $k = ab$ with $1 < a, b < k$.
     Then by the hypothesis, $a$ and $b$ can be written as a product of primes.
     Thus since $k$ is a product of $a$ and $b$, it can also be written as a product of primes.
  }
\end{proof}
\begin{theorem}[Infinity of Primes]
  There are infinitely many primes.
\end{theorem}
\begin{proof}[Euclid]
  Suppose there are finitely many primes, say: $p_1, p_2, \ldots, p_n$. Consider:
  \[
    N = p_1 p_2 \cdots p_n + 1
  \]
  Then we see that $p_1 \nmid N$ else $p_1 \mid (N - p_1 p_2 \cdots p_n) = 1$.
  By the same argument none of the other primes divide $N$.
  This contradicts the fact that $N$ can be written as a product of primes (See \cref{prodPrimes}).
\end{proof}
\begin{remark}[Note]
  We are not claiming that $N$ is prime here, just that it cannot be written as a product of primes.
\end{remark}
We do not currently know if a number's prime factorisation is unique.
To prove that a number's prime factorisation is unique, we need to introduce a few other things.
\begin{proposition}[Euclid's Lemma]
  If $p$ is a prime and $p \mid ab$ then $p \mid a$ or $p \mid b$.
\end{proposition}
We first need to introduce the highest common factor.
\begin{definition}[Highest Common Factor]
  Given $a, b \in \N$, we say that $c \in \N$ is the \textit{highest common factor} ($\hcf$) or \textit{greatest common divisor} ($\gcd$) of $a$ and $b$ if:
  \begin{enumerate}
    \item $c \mid a$ and $c \mid b$ ($c$ is a common divisor of $a$ and $b$).
    \item $d \mid a$ and $d \mid b$ then $d \mid c$ (Every common divisor of $a$ and $b$ divides $c$).
  \end{enumerate}
  We denote this $c = \hcf(a, b) = \gcd(a, b)$.
\end{definition}
\begin{example}
  The factors of $12$ are $1, 2, 3, 4, 6, 12$. The factors of $18$ are $1, 2, 3, 6, 9, 18$.
  So the common factors are $1, 2, 3, 6$.
  Hence $\hcf(12, 18) = 6$.
\end{example}
We now need to show that that $\hcf(a, b)$ always exists.
\section{Euclid's Algorithm}
\begin{proposition}[Division Algorithm]
  Let $n, k \in \N$.
  Then we can write $n = qk + r$ for some $q, r \in \Z$ and $0 \leq r \leq k - 1$.
\end{proposition}
\begin{remark}[Notation]
  We use ``$q$'' to mean \textit{quotient} and ``$r$'' to mean \textit{remainder}.
\end{remark}
\begin{proof}
  \induction
  {$n = 1$}{
    If $k = 1$ then $q = 1$ and $r = 0$, if $k > 1$ then $q = 0$ and $r = 1$
  }
  {$n = p - 1$}{
    So $p - 1= qk + r$ with $0 \leq r \leq k-1$.
  }
  {$n = p$}{
    \begin{proofcases}
      \begin{case}{$r < k - 1$}
        Then $p = qk + (r + 1)$ and $0 < r + 1 < k$ which is in the correct form.
      \end{case}
      \begin{case}{$r = k - 1$}
        Then $p = qk + k = k(q + 1) + 0$ which is in the correct form.
      \end{case}
    \end{proofcases}
  }
\end{proof}
\begin{remark}
  The quotient and remainder obtained for each $n$ are unique.
\end{remark}
\begin{proof}
  Suppose we have $n = qk + r = q' k + r'$, then $(q - q')k = r' - r$.
  Since $0 \leq r, r' \leq k -1$, we have $-k < r' - r < k$.
  The LHS is a multiple of $k$ so we must have $r - r' = 0$ and $q - q' = 0$.
  Thus $r = r'$ and $q = q'$ so the quotient and remainder are unique.
\end{proof}
Euclid's algorithm is a method to find the highest common factor of a number.
We repeatedly apply the division algorithm until we get a remainder of 0.
\begin{center}
\begin{tabular}{c|c|c|c}
  Step & Generic form (a,b) & Bound on $r_{i}$ & Example (534, 372) \\
\hline
1 & $a = q_1 b + r_1$ & $0 \leq r_1 \leq b-1$ & $534 = 1(372) + 162$ \\
2 & $b = q_2 r_1 + r_2$ & $0 \leq r_2 \leq r_1 - 1$ & $372 = 2(162) + 48$ \\
3 & $r_1 = q_3 r_2 + r_3$ & $0 \leq r_3 \leq r_2 - 1$ &$162 = 3(48) + 18$ \\
$\cdots$ & $\cdots$ & $\cdots$ &$48 = 2(18) + 12$ \\
$n$ & $r_{n-2} = q_n r_{n-1} + r_n$ & $0 \leq r_n \leq r_{n-1} - 1$&$18 = 1(12) + 6$ \\
n + 1 & $r_{n-1} = q_{n+1} r_n + 0$ & $r_{n + 1} = 0$ &$12 = 2(6) + 0$ \\
\hline
Output & $r_n$& -- & 6 \\
\end{tabular}
\end{center}
\begin{remark}[Note]
  The algorithm terminates in $n < b$ steps since $b > r_1 > r_2 > \cdots > r_n > 0$.
\end{remark}
\begin{theorem}
  The output of Euclid's algorithm with input $a, b$ is $\hcf(a, b)$.
\end{theorem}
\begin{proof}
  We need to show that $r_n$ satisfies both conditions to be $\hcf(a, b)$.
  \begin{enumerate}
    \item We have $r_n \mid r_{n - 1}$ (as $r_{n + 1} = 0$).
      So by induction:
      \begin{center}
      \begin{tabular}{c|c|c|c|c|c|c}
        Step & $n + 1$ & $n$ & $\cdots$ & 3 & 2 & 1 \\
        \hline
        Conclusion & $r_n \mid r_{n - 1}$ & $r_n \mid r_{n-2}$ & $\cdots$ & $r_n \mid r_1$ & $r_n \mid b$ & $r_n \mid a$
      \end{tabular}
      \end{center}
      Hence $r_n \mid b$ and $r_n \mid a$.
    \item Given $d$ such that $d \mid a$ and $d \mid b$.
      So $d \mid r_1$ and therefore, by induction:
      \begin{center}
      \begin{tabular}{c|c|c|c|c|c|c}
        Step & 1 & 2 & 3 & $\cdots$ & $n - 1$ & $n$ \\
        \hline
        Conclusion & $d \mid r_1$ & $d \mid r_2$ & $d \mid r_3$ & $\cdots$ & $d \mid r_{n - 1}$ & $d \mid r_n$
      \end{tabular}
      \end{center}
      Hence $d \mid r_n$.
  \end{enumerate}
  Therefore $r_n$ satisfies both conditions to be $\hcf(a, b)$.
\end{proof}
\begin{example}
  Suppose we want $\hcf(87, 52)$, we use Euclid's algorithm:
  \begin{center}
  \begin{tabular}{c|c}
  Step & Calculation \\
  \hline
  1 & $87 = 1 \cdot 52 + 35$ \\
  2 & $52 = 1 \cdot 35 + 17$ \\
  3 & $35 = 2 \cdot 17 + 1$ \\
  4 & $17 = 17 \cdot 1 + 0$
  \end{tabular}
  \end{center}
  As $1$ is the last non-zero remainder, $\hcf(87, 52) = 1$.
\end{example}
\end{document}
