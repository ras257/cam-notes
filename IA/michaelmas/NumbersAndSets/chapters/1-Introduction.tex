\documentclass[../main.tex]{subfiles}
\begin{document}
\chapter{Introduction}
\section{Foundations of Proof}
To introduce university style maths, we will look at:
\begin{itemize}
  \item Precise definitions
  \item Rigorous proofs
  \item Foundational questions
\end{itemize}
\begin{definition}[Mathematical Statement]
  A sentence that can have a true or false value. 
\end{definition}
\begin{definition}[Proof]
  A sequence of true statements without logical gaps establishing some conclusion.
\end{definition}
We want to prove things because:
\begin{itemize}
  \item We want to know they are true
  \item We want to gain insight into why something is true
  \item The proof might be elegant/cool
\end{itemize}
We usually denote the end of a proof with the symbol \qed\;or ``Q.E.D.''

\subsection{Examples}
Here is an example proof:
\begin{proposition}
  For all positive integers $n$, $n^3 - n$ is always a multiple of 3.
\end{proposition}
\begin{proof}
  $\forall n \in \N, n^3 - n = n(n + 1)(n - 1) = (n - 1)n(n + 1)$.
  
  Notice that this is a product of three consecutive integers so one must be a multiple of three and thus their product must be a multiple of three.
\end{proof}
Here is an example non-proof:
\begin{proposition}
  For any positive integer $n$, if $n^2$ is even then so is $n$.
  \label{nEven}
\end{proposition}
\begin{proof}
  Given an positive integer $n$ which is even, we can write $n = 2k$.
  Hence $n^2 = (2k)^2 = 2(2k^2)$, which is even.
\end{proof}
This is incorrect as we have instead proved the converse of the statement.
$A \implies B$ is not sufficient to conclude that $B \implies A$.

Statements may contain more assertions than immediately obvious:
\begin{proposition}
  The solution to $x^2 - 5x + 6 = 0$ is $x = 2$ or $x = 3$.
\end{proposition}
This is in fact 2 assertions:
\begin{enumerate}
  \item $x = 2$ and $x = 3$ are solutions
  \item There are no other solutions
\end{enumerate}
We could prove this by individually proving both assertions:
\begin{proof}
  \begin{enumerate}
    \item Suppose $x = 2$ or $x = 3$ then $x - 2 = 0$ or $x - 3 = 0$ so:
      \begin{align*}
        (x-2)(x-3) &= 0 \\
        x^2 - 5x + 6 &= 0
      \end{align*}
    \item Suppose $x^2 - 5x + 6 = 0$ 
      \begin{align*}
        x^2 - 5x + 6 &= 0 \\
        (x - 2)(x - 3) &= 0 \\
        \text{ so } x = 2 \text{ or } x &= 3
      \end{align*}
  \end{enumerate}
\end{proof}
Doing it as above is tedious so we can instead use ``if and only if logic'':
\begin{proof}
  \begin{align*}
    x = 2 \text{ or } x &= 3 \\
    \iff x - 2 = 0 \text{ or } x - 3 &= 0 \\
    \iff (x - 2)(x - 3) &= 0 \\
    \iff x^2 - 5x + 6 &= 0
  \end{align*}
\end{proof}

We also need to ensure that any assumptions we make are reasonable:
\begin{proposition}
  Every positive real is $\geq 1$
\end{proposition}
\begin{proof}
  Let $r$ be the least such positive real.
  By trichotomy either $r = 1$ or $r < 1$ or $r > 1$.
  \begin{proofcases}
    \begin{case}{$r < 1$}
      If $r < 1$, then $0 < r^2 < r$ which contradicts the assumption that $r$ is the least such positive real
    \end{case}
    \begin{case}{$r > 1$}
      If $r > 1$, then $0 < \sqrt{r} < r$ which contradicts the assumption again
    \end{case}
    As both other cases lead to an contradiction $r = 1$ is the only remaining case and so 1 is the least positive real.
  \end{proofcases}
\end{proof}
This is clearly nonsense! We don't know that there is a least positive real.
To avoid errors like this we need to ensure that \textbf{every} assumption is justified when suitable.
\subsection{Proof by contradiction}
To show $A \implies B$ we can show that there is no case where $A$ is true and $B$ is false. i.e:
\[
  (A \implies B) \iff (\lnot B \implies \lnot A)
\]
For example, here is the correct proof for \cref{nEven}.
\begin{proof}
  Assume on the contrary that $n^2$ is even but $n$ is odd.

  Thus we have $n = 2k + 1$ so $n^2 = (2k + 1)^2 = 4k^2 + 4k + 1 = 2(2k^2 + 2k) + 1$ which is odd.

  Thus we have arrived at a contradiction.
\end{proof}
\subsection{Disproof by counter example}
In general, to show that $A \centernot\implies B$ it is enough to show that there is a case where $A$ is true and $B$ is false.
This is the idea that one counter-example is enough to disprove a statement.
For example:
\begin{proposition}
  For any positive integer $n$, if $n^2$ is a multiple of 9, then so is $n$.
\end{proposition}
This is clearly not true. If we consider $n = 3$ then $n^2 = 9$ which is a multiple of 9 but $n$ is not.
\subsection{Combining Claims}
If $A$ and $B$ are assertions, we can (but usually don't) write $A \land B$ for the statement ``A AND B'', $A \lor B$ for the statement ``A OR B'', and $\lnot A$ for ``NOT A''.

Here is the truth table for the combinations depending on $A$ and $B$:\par
\begin{center}
\begin{tabular}{c|c|c|c|c|c}
  $A$ & $B$ & $A \land B$ & $A \lor B$ & $\lnot A$ & $A \implies B$ \\
\hline
  F & F & F & F & T & T \\
  F & T & F & T & T & T \\
  T & F & F & T & F & F \\
  T & T & T & T & F & F 
\end{tabular}
\end{center}\par
Note that $\lnot (A \land B) \iff (\lnot A) \lor (\lnot B)$ and $\lnot (A \lor B) \iff (\lnot A) \land (\lnot B)$.

$A \implies B$ is equivalent to $(\lnot A) \lor B$, $B \lor (\lnot A)$ and hence also to $\lnot B \implies \lnot A$.

$\lnot B \implies \lnot A$ is called the ``contrapositive'' of the statement $A \implies B$.

\subsection{Quantifiers}
The quantifier ``for all'' is denoted $\forall$ and the quantifier ``there exists'' is denoted $\exists$.
Quantifiers can be negated:
\[
  \lnot (\forall x A(x)) \iff \exists x\; \lnot A(x) 
\]
\[
  \lnot (\exists x B(x)) \iff \forall x\; \lnot B(x) 
\]

\section{Number Systems}
Here are informal definitions of a few number systems covered within this course:
\begin{definition}[Natural Numbers]
  We denote the set of \textit{natural numbers} $\N$. It is the set of numbers $\{1, 2, 3, \ldots\}$
\end{definition}
\begin{definition}[Integers]
  We denote the set of \textit{integers} $\Z$. It is the set of numbers $\{\ldots, -3, -2, -1 , 0, 1, 2, 3, \ldots\}$
\end{definition}
\begin{definition}[Rational Numbers]
  We denote the set of all \textit{rational} numbers $\Q$. It is the set of all numbers of the form $a/b$ where $a, b \in \Z$ and $b \neq 0$.
\end{definition}
\begin{definition}[Algebraic Numbers]
  A real number is \textit{algebraic} if it is the root some polynomial with integer coefficients.
\end{definition}
\begin{definition}[Transcendental Numbers]
  A real number is \textit{transcendental} if it is not algebraic. 
\end{definition}
\end{document}
