\documentclass[../main.tex]{subfiles}
\begin{document}
\chapter{Real Numbers}
\section{Constructions using \texorpdfstring{$\N$}{the naturals}}
Recall the construction of $\N$ using the Peano axioms from \cref{natualDef}.
We can build on this construction to formally define the \textit{integer} and \textit{rational} numbers.
\subsection{Integers\texorpdfstring{ -- $\Z$}{}}
We obtain $\Z$ from $\N$ by allowing for subtraction, formally, $\Z$ is the equivalence classes of $\N \times \N$ under the equivalence relation:
\[
  (a, b) \rel (c, d) \iff a + d = b + c
\]
That is, $\Z$ is the quotient $\Z = (\N \times \N) / R$.
Think of $(a, b)$ as representing the operation $a - b$.
We write $0$ for $[(1, 1)]$ and $-a$ for $[(1, 1 + a)]$.
We also define addition and multiplication operations by:
\begin{align*}
  (a, b) + (c, d) &= (a + c, b + d) \\
  (a, b) \times (c, d) &= (ac + bd, bc + ad)
\end{align*}
which satisfy the usual rules of arithmetic.

\subsection{Rationals\texorpdfstring{ -- $\Q$}{}}
We obtain $\Q$ from $\Z$ by allowing for division.
Formally, $\Q$ is the set of equivalence classes of $\Z \times \N$ under an equivalence relation:
\[
  (a, b) \rel (c, d) \iff ad = bc
\]
That is, $\Q$ is the quotient $\Q = (\Z \times \N) / R$.
We write $\frac{a}{b}$ for $[(a, b)]$.
We also define additional and multiplication by:
\begin{align*}
  (a, b) + (c, d) &= (ad + bc, bd) \\
  (a, b) \times (c, d) &= (ac, bd)
\end{align*}
which satisfy the usual rules of arithmetic.

We can also define an order on $\Q$, denoted $<$, which has the properties:
\begin{enumerate}
  \item \textbf{Trichotomy -} If $a, b \in \Q$ then exactly one of:
    \[
      a < b \qquad a = b \qquad b < a
    \]
    holds.
  \item \textbf{Transitivity -} If $a < b$ and $b < c$ then $a < c$.
\end{enumerate}
\end{document}
