\documentclass[../main.tex]{subfiles}
\begin{document}
\chapter{Modular Arithmetic}
\section{Definition}
\begin{definition}[Integers Modulo $n$]
Let $n \geq 2$ be a natural number.
Then \textit{the integers modulo $n$}, denoted $\Z_n$ or $\Z/n\Z$, consists of the integers, with two integers regarded the same if they differ by a multiple of $n$.
\end{definition}
\begin{definition}[Congruent]
  Integers $x$ and $y$ are said to be \textit{congruent} modulo $n$ if $n \mid x - y$.
  We write $x \equiv y \pmod{n}$ and say ``$x$ is \textit{congruent} to $y$ under mod $n$''.
\end{definition}
\begin{remark}[Note]
  $x \equiv y \pmod{n} \iff n \mid x - y \iff x = y + kn$ for some $k \in \Z$.
\end{remark}
\begin{example}
  In $\Z_7$, $16 \equiv 2 \pmod{7}$.
  In $\Z_3$, $-1 \equiv 2 \pmod{3}$.
  In $\Z_2$, $2k + 1 \equiv 1 \pmod{2}$.
\end{example}
\section{Arithmetic}
We can carry out addition and multiplication as usual in $\Z_n$.
\begin{proposition}[Arithmetic of mod $n$]
  If $a \equiv a' \pmod{n}$ and $b \equiv b' \pmod{n}$ then
  \begin{enumerate}
    \item $a + b \equiv a' + b' \pmod{n}$
    \item $ab \equiv a'b' \pmod{n}$
  \end{enumerate}
\end{proposition}
\begin{proof}
  By definition, $a \equiv a' \pmod{n} \iff n \mid (a - a')$ and $b \equiv b' \pmod{n} \iff n \mid (b - b')$.
  \begin{enumerate}
    \item So:
      \[
        n \mid (a - a') + (b - b') = (a + b) - (a' + b')
      \]
      Thus, $a + b \equiv a' + b' \pmod{n}$.
    \item So:
      \[
        n \mid (a - a')\cdot b + a' (b - b') \cdot a = ab - a'b'
      \]
      Thus, $ab \equiv a'b' \pmod{n}$
  \end{enumerate}
\end{proof}
\begin{example}
  Does $2a^2 + 3b^3 = 1$ have a solution with $a, b \in \Z$?

  Suppose it does have a solution, then $2a^2 \equiv 1 \pmod{3}$.
  We only need to check values of $a$ with $0 \leq a < 3$ as any other values would be congruent to one of these under mod $n$.
  \[
    2 \cdot 0^{2} \equiv 0,\ 2 \cdot 1^{2} \equiv 2,\ 2 \cdot 2^{2} \equiv 2 \pmod{n}
  \]
  So we can never get a remainder of $1$.
  Therefore there is no solution.
\end{example}
\section{Solving Congruences}
\begin{example}
  Solve $7x \equiv 2 \pmod{10}$.
  We note that $3\cdot 7 \equiv 21 \equiv 1 \pmod{10}$ so $3 \cdot 7x \equiv 6 \pmod{10}$ so $x \equiv 6 \pmod{10}$.
\end{example}
\begin{definition}[Unit Modulo $n$]
  Given $a, b \in \Z$, we say that $b$ is an \textit{inverse of $a$ modulo $n$} or that $a$ is a \textit{unit modulo $n$}, if it has an inverse.
\end{definition}
\begin{example}
  In $\Z_{10}$, 3 is an inverse of 7, and both 3 and 7 are units modulo 10.
  But 4 is \textbf{not} a unit modulo 10 since $4x \centernot\equiv 1 \pmod{10}$ regardless of $x$.
\end{example}
\begin{proposition}
  If $a$ is a unit modulo $n$, then it has a \textbf{unique} inverse modulo $n$.
\end{proposition}
\begin{proof}
    Let us suppose that there is two inverses $b, b'$ such that $ab \equiv ab' \equiv 1 \pmod{n}$, then $b \equiv bab \equiv bab' \equiv b' \pmod{ n}$.
\end{proof}
\begin{remark}[Notation]
  Since the inverse of $a$ is unique, we write $a^{-1}$ to mean the inverse of $a$.
\end{remark}
If $ab \equiv ac \pmod{n}$, then $b \equiv c \pmod{n}$.
That is, we can cancel units by multiplying both sides by $a^{-1}$.
However, this is not true in general.
For example, as $4$ is not a unit, $4 \cdot 3 \equiv 4 \cdot 8 \pmod{10}$ but $3 \centernot\equiv 8 \pmod{10}$.
\begin{proposition}
  Let $p$ be prime, then every non-zero residue modulo $p$ is a unit modulo $p$.
\end{proposition}
\begin{proof}
  As $p$ is prime and the residue is non-zero, We must have $\hcf(a, p) = 1$, so by \cref{bezoutsIdentity} (B\'ezout's Identity) there exists $x, y \in \Z$ such that $ax + py = 1$.
  Hence $ax = 1 - py$ so $ax \equiv 1 \pmod{p}$ for some $x \in \Z$.
\end{proof}
\begin{remark}[Note]
  A \textit{residue} modulo $n$ is the value of the number under modulo $n$.
\end{remark}
\begin{proposition}
  Let $n \geq 2$.
  Then $a$ is a unit modulo $n$ if and only if $\hcf(a, n) = 1$.
\end{proposition}
\begin{proof}
  $\hcf(a, n) = 1 \iff ax + ny = 1$ for some $x, y \in \Z$
  $\iff ax = 1 - ny \iff ax \equiv 1 \pmod{n}$
\end{proof}
\begin{corollary}
  If $\hcf(a, n) = 1$ then the congruence $ax \equiv b \pmod{n}$ has a unique solution.
  In particular, if $\hcf(a, n) = 1$ then there is a unique inverse of $a$ modulo $n$.
  \label{congUniqueSolution}
\end{corollary}
\begin{example}
  Suppose we want to figure out which days of the week New Years Day can fall on.
  If we set New Years Day as day 0 and our week has 7 days in it, then we need to solve $7x + k = 365y$, that is, $365y \equiv k \pmod{7}$.
  $365 = 7 \cdot 52 + 1$ so $365 \equiv 1 \pmod{7}$, thus $k \equiv y \pmod{7}$ so we can get any day of the week.
\end{example}
What if $ax \equiv b \pmod{n}$ with $\hcf(a, n) = d > 1$?
\begin{proposition}
  The congruence $ax \equiv b \pmod{n}$ where $\hcf(a, n) = d > 1$ has a solution if and only if $d \mid b$.
\end{proposition}
\begin{proof}
  \begin{proofdirection}{Assume there is a solution}
    Then $n \mid ax - b$, $d \mid n$, and $d \mid a$ so we must have $d \mid b$.
  \end{proofdirection}
  \begin{proofdirection}{Assume $d \mid b$}
    Then $n = dn'$, $a = da'$, and $b = db'$.
    Note now what $\hcf(a', n') = 1$ so $a'x \equiv b \pmod{n'}$ has a unique solution by \cref{congUniqueSolution}.
    Thus for some $k \in \Z$:
    \[
      a'x -  b = kn' \implies da'x - db' = kdn' \implies ax - b = kn \implies ax \equiv b \pmod{n}
    \]
    So there is a solution to $ax \equiv b \pmod{n}$.
  \end{proofdirection}
\end{proof}
So if $\hcf(a, n) = d > 1$, then the congruence $ax \equiv b \pmod{n}$ has no solution unless $d \mid b$, in which case the solutions are exactly $\frac{a}{d}x \equiv \frac{b}{d} \pmod{\frac{n}{d}}$.
The solution to this congruence will be unique modulo $\frac{n}{d}$ but there will be $d$ solutions modulo $n$.
\begin{example}
  \begin{enumerate}
    \item Suppose we want to solve $7x \equiv 4 \pmod{30}$.
      By Euclid's algorithm we have $\hcf(7, 30) = 1$ and $13 \cdot 7 - 3 \cdot 30 = 1$.
      Hence $13 \cdot 7 \equiv 1 \pmod{30}$ so $7^{-1} \equiv 13 \pmod{30}$.
      Thus $x \equiv 13 \cdot 4 \equiv 22 \pmod{30}$.

      Suppose that $x'$ is also a solution to the congruence $7x' \equiv 4 \pmod{30}$, thus $7x \equiv 7x' \pmod{30}$ and since $7$ is a unit, $x \equiv x' \pmod{30}$.

      For brevity we would just write:
      \begin{align*}
        &7x \equiv 4 \pmod{30} \\
        \iff& 13 \cdot 7x \equiv 13 \cdot 4 \pmod{30} \\
        \iff& x \equiv 22 \pmod{30}
      \end{align*}
      Note that is works both ways as $13$ and $7$ are both units modulo 30.
    \item Solve $10x \equiv 12 \pmod{34}$.
      $\hcf(10, 34) = 2$, $2 \mid 12$ so there will be solutions given by:
      \[
        \frac{10}{2}x \equiv \frac{2}{2} \pmod{34/2} \iff 5x \equiv 1 \pmod{17}
      \]
      We can also convert out of modular arithmetic instead of using the formula above:
      \begin{align*}
        &10x \equiv 12 \pmod{34} \\
        \iff& 10x = 12 + 34y \\
        \iff& 5x = 6 + 17y \\
        \iff& 5x \equiv 6 \pmod{17}
      \end{align*}
      Note that $\hcf(5, 17) = 1$ so we are back in the original case and can solve as before.
  \end{enumerate}
\end{example}
\subsection{Simultaneous Congruences}
Suppose we wish to solve the simultaneous congruences $x \equiv 1 \pmod{4},\ x \equiv 2 \pmod{3}$.
\begin{align*}
  x \equiv 1 \pmod{4} \to x = 1, 5, 9, \ldots \\
  x \equiv 2 \pmod{3} \to x = 2, 5, 8, \ldots
\end{align*}
So $x \equiv 5 \pmod{12}$.

What about $x \equiv 1 \pmod{4},\ x \equiv 2 \pmod{6}$?
\begin{align*}
  x &\equiv 1 \pmod{4} \to x = 1, 5, 9, 13, 17\ldots \\
  x &\equiv 2 \pmod{6} \to x = 2, 8, 14, 20 \ldots
\end{align*}
In fact there is no such $x$ in both lists.

The Chinese Remainder Theorem can help us solve simultaneous congruences.
\begin{theorem}[Chinese Remainder Theorem]
  Let $m, n$ be coprime, and $a, b \in \Z$.
  Then there is a unique solution modulo $mn$ to the simultaneous congruences:
  \[
    x \equiv a \pmod{m} \text{ and }
    x \equiv b \pmod{n}
  \]
\end{theorem}
\begin{remark}[Note]
  By ``unique'' modulo $mn$, we mean that $y$ is a solution to the same system if and only if $x \equiv y \pmod{mn}$.
\end{remark}
\begin{proof}
  \begin{itemize}
    \item \textbf{Existence}
      Since $\hcf(m, n) = 1$, there exists $s, t \in \Z$ such that $sm + tn = 1$.
      Note that $sm \equiv 1 \pmod{n}$ and $tn \equiv 1 \pmod{m}$.
      Trivially, we also have $sm \equiv 0 \pmod{m}$ and $tn \equiv 0 \pmod{n}$.
      Hence,
      \[
        x = a(tn) + b(sm) \equiv \begin{cases}
          a & \pmod{m} \\
          b & \pmod{n}
        \end{cases}
      \]
      \item \textbf{Uniqueness -}
      Suppose $y$ is also a solution, that is, $y \equiv a \pmod{m}$ and $y \equiv b \pmod{n}$.
      \begin{align*}
        &y \equiv a \pmod{n} \text{ and } y \equiv b \pmod{n} \\
        \iff& y \equiv x \pmod{m} \text{ and } y\equiv x \pmod{n} \\
        \iff& m \mid y - x \text{ and } n \mid y - x \\
        \iff& mn \mid y - x \text{ as $\hcf(m, n) = 1$} \\
        \iff& y \equiv x \pmod{mn}
      \end{align*}
  \end{itemize}
\end{proof}
\begin{remark}
  The Chinese remainder theorem can be extended, by induction, to more than 2 moduli.

  If $m_1, m_2, \ldots m_k$ are pairwise coprime, then for all $a_1, a_2, \ldots a_k \in \Z$ there exists an $x \in \Z$ such that:
  \begin{align*}
    x &\equiv a_1 \pmod{m_1} \\
    x &\equiv a_2 \pmod{m_2} \\
    &\cdots \\
    x &\equiv a_k \pmod{m_k} \\
  \end{align*}
\end{remark}
\section{Exponentiation and Factorials}
\subsection{FLT and Fermat-Euler}
\begin{definition}[Euler Totient Function]
  The \textit{Euler totient function}, $\varphi(n)$, counts the number of integers with $1 \leq a \leq n$ such that $\hcf(a, n) = 1$.
  That is, $\varphi(n)$ is the number of units modulo $n$.

  By convention, $\varphi(1) = 1$.
\end{definition}
\begin{example}
  \begin{enumerate}
    \item  If $m = 9 \to 1, 2, 4, 5, 7, 8$ are coprime so $\varphi(n) = 6$.
    \item When $p$ is prime, $\varphi(p) = p - 1$ as all naturals less than $p$ are coprime to $p$ and $\varphi(p^2) = p^2 - p$ as there are $p$ multiples of $p$ less than or equal to $p^2$.
    \item When $p, q$ are distinct primes, $\varphi(pq) = pq - p - q + 1$.
      The $-p$ accounts for the $p$ multiples of $q$ and the $-q$ accounts for the $q$ multiples of $p$, since $pq$ is then double counted, we need to add 1.
  \end{enumerate}
\end{example}
How do the powers of an integer behave modulo $p$?
\begin{example}
  \begin{enumerate}
    \item For example, in $\Z_7$: $2^{1} \equiv 2, 2^{2} \equiv 4, 2^{3} \equiv 1, 2^{4} \equiv 2$, then it repeats $2, 4, 1, 2, 4, 1, 2, \ldots$
    \item For example, in $\Z_{11}$: $2^{1} \equiv 2, 2^{2} \equiv 4, 2^{3} \equiv 8, 2^{4} \equiv 5, 2^{5} \equiv 10, 2^{6} \equiv 9, 2^{7} = 7,2^{8} = 3, 2^{9} \equiv 6, 2^{10} \equiv 1$, then repeats.
  \end{enumerate}
\end{example}
\begin{theorem}[Fermat's Little Theorem]
  Let $p$ be a prime.
  Then $a^{p} \equiv a \pmod{p}$ for all $a \in \Z$.

  Equivalently $a^{p - 1} \equiv 1$ for all $a \centernot\equiv 0 \pmod{p}$
\end{theorem}
\begin{proof}
  If $a \centernot\equiv 0 \pmod{p}$, then $a$ is a unit mod $p$.
  So:
  \[
    ax \equiv ay \pmod{p} \iff x \equiv y \pmod{p}
  \]
  The numbers $1, 2, \ldots, p - 1$ are all pairwise incongruent and non-zero mod $p$.
  Hence the numbers $a, 2a, \ldots, (p-1)a$ are pairwise incongruent and non-zero mod $p$.
  So they must be $1, 2, 3, \ldots, p - 1$ in some order.

  Hence:
  \[
    a \cdot 2a \cdot 3a \cdots (p-1)a \equiv 1 \cdot 2 \cdot 3 \cdots(p-1) \pmod{p}
  \]
  so $a^{p-1}(p-1)! \equiv (p-1)! \pmod{p}$.
  But $(p - 1)!$ is a product of units so is also a unit mod $p$ so we can cancel it to obtain $a^{p-1} \equiv 1 \pmod{p}$.
\end{proof}
\begin{theorem}[Fermat-Euler Theorem]
  If $a, m \in \Z$ such that $\hcf(a, m) = 1$, then $a^{\varphi(m)} \equiv 1 \pmod{m}$.
\end{theorem}
\begin{proof}
  Let $\mathcal{U} = \{x \in \Z : 0 < x < m,\ \hcf(x, m) = 1\}$, this is the $\varphi(m)$ numbers less than $m$ that are coprime to $m$.
  These must all be units mod $m$ as they are coprime to $m$.
  Label then $u_1, u_2, \ldots, u_{\varphi(m)}$, similarly to before, we have that $au_1, a_2, \ldots au_{\varphi(n)}$ are all distinct and invertible modulo $m$, since $a$ is a unit.
  So they must be $u_1, u_2, \ldots u_{\varphi(n)}$ in some order.
  Hence
  \[
    au_1 \cdot au_2 \cdots au_{\varphi(m)} \equiv u_1 \cdot u_2 \cdots u_{\varphi(m)} \pmod{m}
  \]
  So $a^{\varphi(m)} z \equiv z \pmod{m}$ where $z = u_1 u_2 \cdots u_{\varphi(m)}$.
  $z$ is a product of units so is also a unit mod $m$.
  Thus we may cancel it to obtain $a^{\varphi(m)} \equiv 1 \pmod{m}$.
\end{proof}
\subsection{Wilson/Al-Haytham Theorem}
What is $(p - 1)!$ modulo $p$?
\begin{example}
  \begin{enumerate}
    \item When $p = 5$, $4! \equiv 24 \equiv -1 \pmod{5}$.
    \item When $p = 7$, $6! = 720 \equiv -1 \pmod{7}$
  \end{enumerate}
\end{example}
We first need to prove a lemma about which integers are self-inverse mod $p$:
\begin{lemma}
  Let $p$ be a prime.
  Then $x^2 \equiv 1 \pmod{p}$ if and only if $x \equiv 1 \pmod{p}$ or $x \equiv -1 \pmod{p}$
\end{lemma}
\begin{proof}
  \begin{align*}
    &x^2 \equiv 1 \pmod{p} \\
    \iff& x^2 - 1 \equiv 0 \pmod{p} \\
    \iff& (x - 1)(x + 1) \equiv 0 \pmod{p} \\
    \iff& p \mid (x - 1)(x + 1)
  \end{align*}
  We know from \cref{euclidsLemma} (Euclid's Lemma) that if $p$ is prime and $p \mid ab$ then $p \mid a$ or $p \mid b$.
  The converse also holds trivially so:
  \begin{align*}
    &p \mid (x - 1)(x + 1) \\
    \iff& p \mid x - 1 \text{ or } p \mid x + 1 \\
    \iff& x - 1 \equiv 0 \pmod{p} \text{ or } x + 1 \equiv 0 \pmod{p} \\
    \iff& x \equiv 1 \pmod{p} \text{ or } x \equiv -1 \pmod{p}
  \end{align*}
  So $x^2 \equiv 1 \pmod{p} \iff x \equiv 1 \text{ or } x \equiv -1 \pmod{p}$.
\end{proof}
\begin{remark}
  More generally, a non-zero polynomial of degree $k$ over $\Z_p$ has at most $k$ roots in $\Z_p$.
\end{remark}
\end{document}
