\documentclass[../main.tex]{subfiles}
\begin{document}
\chapter{Complex Numbers}
\section{Basic Definitions}
\begin{definition}[Complex Number]
  We construct $\C$ by adding the element $i$ to $\R$, where $i^2 = -1$.

  Then any complex number $z \in \C$ can be written as:
  \[
    z = x + iy \quad x, y \in \R
  \]
\end{definition}
We denote the real part of $z$ as $\Re(z) = x$ and the imaginary part as $\Im(z) = y$.

For complex numbers $z_1 = x_1 + i y_1$ and $z_2 = x_2 + i y_2$, we can carry out addition component-wise:
\[
  z_1 \pm z_2 = (x_1 \pm x_2) + i(y_ 1 \pm y_2) 
\]
and noting that $i^2 = -1$, we can carry out multiplication:
\begin{equation}
  z_1 z_2 = (x_1 x_2 - y_1 y_2) + i(x_1 y_2 + x_2 y_1)
  \label{complexMult}
\end{equation}
Both addition and multiplication of complex numbers are associative and commutative.
Moreover, the distributive property is satisfied. i.e. for $z_1, z_2, z_3 \in \C$:
\[
  (z_1 + z_2)z_3 = z_1 z_2 + z_2 z_3
\]

For $z \neq 0$ the inverse of $z$ is given by:
\begin{equation}
  z^{-1} = \frac{x - iy}{x^2 + y^2}
  \label{inverse}
\end{equation}
which satisfies $zz^{-1} = 1$.
\begin{definition}[Complex Conjugate]
  For any $z \in \C$ with $z = x + iy$ we define its \textit{complex conjugate} by:
  \[
    \bar{z} = z^{*} = x - iy
  \]
\end{definition}
Note that this allows us to write:
\[
  \Re(z) = \frac{z + z^{*}}{2} \text{ and }
  \Im(z) = \frac{z - z^{*}}{2i}
\]
Some properties satisfied by the complex conjugate are:
\begin{itemize}
  \item $(z^{*})^{*} = z$
  \item $(z_1 + z_2)^{*} = z^{*}_{1} + z^{*}_{2}$
  \item $(z_1 z_2)^{*} = z^{*}_{1} z^{*}_{2}$
\end{itemize}
\begin{definition}[Modulus of a Complex Number]
  For any $z \in \C$ with $z = x + iy$ we define its \textit{modulus} by a real and non-negative number $|z|$, such that:
  \[
    |z|^2 = x^2 + y^2
  \]
  The modulus of $z$ is sometimes denoted by $r$.
\end{definition}
Note that $|z|^2 = z z^{*}$.
We can also use the modulus to write a second form for \cref{inverse}:
\[
  z^{-1} = \frac{z^{*}}{|z|^2}
\]
\begin{definition}[Argument of a Complex Number]
  For any $z \in \C \setminus \{0\}$ we define its \textit{argument} by a real number denoted $\theta$ or $\arg(z)$ such that:
  \[
    z = r(\cos \theta + i \sin \theta)
  \]
\end{definition}
Writing $z$ as above is called the ``polar form'' of $z$.
From this it follows that (for $z \neq 0$):
\[
  \cos \theta = \frac{x}{r} = \frac{x}{\sqrt{x^2 + y^2}}\text{ and }\sin \theta = \frac{y}{r} = \frac{y}{\sqrt{x^2 + y^2}}
\]
and thus:
\[
  \tan \theta = \frac{y}{x}
\]
If $\theta$ is the argument of $z$ then any $\theta + 2n\pi, n \in \Z$ is also an argument of $z$.
To make the argument of $z$ unique, we restrict the range of $\theta$ to $-\pi < \theta \leq \pi$ and call this argument the ``principal argument''.

\begin{remark}[Notation]
  We denote the principle argument $\Arg(z)$ where $-\pi < \Arg(z) \leq \pi$ and the multi-valued argument $\arg(z)$ where $\arg(z) = \{\Arg(z) + 2n\pi\;\forall n \in \Z\}$. 
\end{remark}

\section{The Argand Diagram}
The Argand diagram is sometimes called the ``Complex plane''.

To plot a complex number $z = x + iy$ on the Argand diagram, we plot the real part of $z$ along the $x$-axis and the imaginary part of $z$ along the $y$-axis, i.e. as a coordinate $(x, y)$ on a 2D-plane.
\begin{center}
\begin{tikzpicture}[scale=0.8]
  \draw (-3.5,0) -- (3.5,0) node[right] {\(\Re\)};
  \draw (0,-3.5) -- (0,3.5) node[right] {\(\Im\)};

  \coordinate (Z) at (2,2.5);

  \filldraw (Z) circle (1pt) node[above right] {\(z = x + iy\)} node[below right] {\((x, y)\)};
\end{tikzpicture}
\end{center}

By representing complex numbers in the Argand diagram then the addition of complex numbers can be treated as vector addition (tip-to-tail):
\begin{center}
\begin{tikzpicture}[scale=0.8]
  \draw (-3.5,0) -- (3.5,0) node[right] {\(\Re\)};
  \draw (0,-3.5) -- (0,3.5) node[right] {\(\Im\)};

  \coordinate (Z1) at (1,1.5);
  \coordinate (Z2) at (1,0.5);
  \coordinate (sum) at (2, 2);

  \draw[->] (0,0) -- (Z1);
  \draw[->] (0,0) -- (Z2);
  \draw[->] (0, 0) -- (sum);

  \draw[dotted] (Z1) -- (sum);
  \draw[dotted] (Z2) -- (sum);

  \filldraw (Z1) node[above left] {\(z_1\)};
  \filldraw (Z2) node[below right] {\(z_2\)};
  \filldraw (sum) node[above right] {\(z_1 + z_2\)};
\end{tikzpicture}
\end{center}

Note that in the Argand diagram the complex conjugate of a complex number $z$ is its reflection in the real axis:
\begin{center}
\begin{tikzpicture}[scale=0.8]
  \draw (-3.5,0) -- (3.5,0) node[right] {\(\Re\)};
  \draw (0,-3.5) -- (0,3.5) node[right] {\(\Im\)};

  \coordinate (Z) at (2,2.5);
  \coordinate (Zconj) at (2,-2.5);

  \draw[dashed] (Z) -- (Zconj);

  \filldraw (Z) circle (1pt) node[above right] {\(z = x + iy\)};
  \filldraw (Zconj) circle (1pt) node[below right] {\(z^{*} = x - iy\)};
\end{tikzpicture}
\end{center}

This makes it easier to visually derive the following properties:
\begin{itemize}
  \item $(z_1 + z_2)^{*} = z^{*}_{1} + z^{*}_{2}$
  \item $(z_1 z_2)^{*} = z^{*}_{1} z^{*}_{2}$
  \item $|z^{*}| = |z|$
\end{itemize}

\section{Properties and Consequences}
\begin{remark}[Other remarks about $\C$]
  \begin{itemize}
    \item $\R \subset \C$, since if we take $a \in \R$ then we can construct $a + i \cdot 0 \in \C$.
    \item A complex number in the form $0 + i \cdot b$ (i.e. $\Re(z) = 0$) is called ``purely imaginary''.
    \item The identity for the operation $+$ in $\C$ is 0 and for $\cdot$ it is 1.
    \item $(\C, +, 0)$ is an abelian group.
    \item $(\C \setminus \{0\}, \cdot, 1)$ is an abelian group.
    \item $(\C, +, \cdot)$ is a field.
  \end{itemize}
\end{remark}
\begin{proposition}
  The representation of a complex number in terms of its real and imaginary part is unique.
\end{proposition}
\begin{proof}
  Assume that there is two different representations of a complex number $z$, that is, $z = x_1 + iy_1 = x_2 + iy_2$.
  \begin{align*}
    x_1 + iy_1 &= x_2 + iy_2 \\
    x_1 - x_2 &= i(y_2 - y_1) \\
    (x_1 - x_2)^2 &= -(y_1 - y_2)^2
  \end{align*}
  The left hand side is $\geq 0$ but the right hand side is $\leq 0$ so both sides must be equal to 0 and thus $x_1 = x_2$ and $y_1 = y_2$.
  So any two representations are the same.
\end{proof}
\begin{theorem}[Fundemental Theorem of Algebra]
  A polynomial of degree $n$ with coefficients $c_i \in \C, c_n \neq 0$ can be written as a product of $n$ linear factors. i.e.:
  \[
    p(z) = c_n z^n + \cdots + c_0 = c_n(z - \alpha_n) \cdots (z - \alpha_0)
  \]
\end{theorem}
Hence $p(z) = 0$ has at least one root in $\C$ and $n$ roots $\alpha_i \in \C$ counted with multiplicity.
\begin{proposition}
  The modulus satisfies the following properties:
  \begin{enumerate}
    \item $|z_1 z_2| = |z_1||z_2|$
    \item $|z_1 + z_2| \leq |z_1| + |z_2|$ (Triangle Inequality)
    \item $|z_1 - z_2| \geq ||z_1| - |z_2||$
  \end{enumerate}
\end{proposition}
\begin{proof}[\textbf{i}]
  \begin{align*}
      |z_1 z_2| &= \sqrt{(x_1 x_2 - y_1 y_2)^2 + (x_1 y_2 + x_2 y_1)^2}\\
                &= \sqrt{x^{2}_{1} x^{2}_{2} -2x_1x_2y_1y_2 + y^{2}_{1}y^{2}_{2} + x^{2}_{1}y^{2}_{2} + 2x_1x_2y_1y_2 + x^{2}_{2}y^{2}_{1}} \\
                &= \sqrt{x^{2}_{1} x^{2}_{2} + y^{2}_{1} y^{2}_{2} + x^{2}_{1} y^{2}_{2} + x^{2}_{2} y^{2}_{1}} \\
                &= \sqrt{\left(x^{2}_{1} + y^{2}_{1}\right)\left(x^{2}_{2} + y^{2}_{2}\right)} \\
                &= \sqrt{x^{2}_{1} + y^{2}_{1}}\sqrt{x^{2}_{2} + y^{2}_{2}} \\
                &= |z_1||z_2|
  \end{align*}
\end{proof}
\begin{proof}[\textbf{ii}]
  Since both sides of the inequality are non-negative, $|z_1 + z_2| \leq |z_1| + |z_2| \iff |z_1 + z_2|^2 \leq (|z_1| + |z_2|)^2$.

  Consider:
  \begin{align*}
      (|z_1| + |z_2|)^2 - |z_1 + z_2|^2 &= x^{2}_{1} + y^{2}_{1} + 2|z_1|| z_2| + x^{2}_{2} + y^{2}_{2} \\
                                        & \quad - (x^{2}_{1} + 2x_1x_2 + x^{2}_{2} + y^{2}_{1} + 2y_1y_2 + y^{2}_{2})\\
                                        &= 2(|z_1||z_2| - x_1 x_2 - y_1 y_2)
  \end{align*}
  We want to prove that the above is positive:
  \begin{align*}
    |z_1||z_2| - x_1 x_2 - y_1 y_2 &\geq 0 \\
    |z_1 z_2| &\geq x_1 x_2 + y_1 y_2
  \end{align*}
  Either $x_1 x_2 + y_1 y_2 \leq 0$ and we are done or:
  \begin{align*}
    (x^{2}_{1} + y^{2}_{1})(x^{2}_{2} + y^{2}_{2}) &\geq (x_1 x_2 + y_1 y_2)^2 \\
    x^{2}_{1} x^{2}_{2} + x^{2}_{1} y^{2}_{2} + x^{2}_{2} y^{2}_{1} + y^{2}_{1} y^{2}_{2}&\geq x^{2}_{1} x^{2}_{2} + 2x_1 x_2 y_1 y_2 + y^{2}_{1} y^{2}_{2} \\
    x^{2}_{1} y^{2}_{2} - 2x_1 x_2 y_1 y_2 + x^{2}_{2} y^{2}_{1} &\geq 0 \\
    (x_1 y_2 - x_2 y_1)^2 &\geq 0
  \end{align*}
  Thus $(|z_1| + |z_2|)^2 \geq |z_1 + z_2|^2$ and therefore $|z_1 + z_2| \leq |z_1| + |z_2|$.
\end{proof}
\begin{proof}[\textbf{iii}]
  In the triangle inequality, set $z_2 \to z_3 - z_1$.
  This leads to:
  \begin{align*}
    |z_1 + z_3 - z_1| &\leq |z_1| + |z_3 - z_1|\\
    |z_3| - |z_1| &\leq |z_3 - z_1|
  \end{align*}
  Now we have that $|z_1 - z_2| \leq |z_1| - |z_2|$.

  Notice that if we swap $z_1$ and $z_2$ we have that $|z_2 - z_1| \leq |z_2| - |z_1|$.
  But $|z_1 - z_2| = |z_2 - z_1|$ so $|z_1 - z_2|$ is greater than or equal to both $|z_1| - |z_2|$ and $-(|z_1| - |z_2|)$.

  This means it must be greater than the absolute value of $|z_1| - |z_2|$ so we have the required result $|z_1 - z_2| \geq ||z_1| - |z_2||$.
\end{proof}
\begin{theorem}[De Moivre's Theorem]
  For $n \in \Z$ then:
  \[
    (\cos \theta + i \sin \theta)^n = \cos n\theta + i \sin n\theta
  \] 
\end{theorem}
We first need to prove the following lemma:
\begin{lemma}
  If $z_1 = r_1(\cos \theta_1 + i \sin \theta_1)$ and $z_2 = r_2(\cos \theta_2 + i \sin \theta_2)$, then:
  \[
    z_1 z_2 = r_1 r_2 (\cos (\theta_1 + \theta_2) + i\sin(\theta_1 + \theta_2))
  \]
  \label{multLemma}
\end{lemma}
\begin{proof}
  Using \cref{complexMult} we have that:
  \begin{align*}
    z_1 z_2 &= r_1 r_2 (\cos \theta_1 \cos \theta_2 - \sin \theta_1 \sin \theta_2 + i(\cos \theta_1 \sin \theta_2 + \cos \theta_2 \sin \theta_1)) \\
            &= r_1 r_2 (\cos(\theta_1 + \theta_2) + i\sin(\theta_1 + \theta_2))
  \end{align*} 
\end{proof}
Now we can move onto the final proof:
\begin{proof}
  \begin{proofcases}
    \begin{case}{$n \geq 0$}
      \induction
      {$n = 0$}{
        $z^0 = 1 = \cos 0 + i \sin 0$ \tick
      }
      {$n = k$}{}
      {$n = k + 1$}{
        \begin{align*}
          z^{(k+1)} &= z^k z \\
                    &= (\cos(k\theta) + i\sin(k\theta))(\cos \theta + i\sin \theta) \text{ by induction hypothesis}\\
                    &= (\cos(k\theta + \theta) + i\sin(k\theta + \theta)) \text{ by \cref{multLemma}}\\
                    &= \cos(k + 1)\theta + i \sin(k + 1)\theta
        \end{align*} 
      }
    \end{case}
    \begin{case}{$n < 0$}
      We now write $n = -m$ with $m > 0$ then:
      \begin{align*}
        (\cos \theta + i \sin \theta)^n &= \frac{1}{(\cos \theta + i \sin \theta)^m} \\
                                        &= \frac{1}{\cos(m\theta) + i\sin(m\theta)} \text{ by Case 1}\\
                                        &= \frac{\cos(m\theta) - i\sin(m\theta)}{\cos^2(m\theta) + \sin^2(m\theta)} \text{ by \cref{inverse}}\\
                                        &= \cos(m\theta) - i\sin(m\theta) \\
                                        &= \cos(-m\theta) + i\sin(-m\theta) \\
                                        &= \cos(n\theta) + i\sin(n\theta)
      \end{align*}
    \end{case} 
  \end{proofcases}
  Shown to be true for $n \geq 0$ and $n < 0$ so true $\forall n \in \Z$.
\end{proof}
\section{Exponential and Trigonometric Functions}
\subsection{Definitions and Properties}
We define the exponential of a complex number $z$, $\exp(z) = e^z$ as:
\[
  \exp(z) = \sum_{n=0}^{\infty} \frac{1}{n!} z^n
\]
This converges $\forall z \in \C$ and has the following properties:
\begin{enumerate}
  \item $\exp(z)\exp(w) = \exp(z + w)$
  \item if $z \in \R$ then $e^z$ reduces to the usual exponential
  \item $e^0 = 1$
  \item $(e^z)^n = e^{nz}$
\end{enumerate}
We define cosine as:
\begin{align*}
  \cos(z) &= \frac{1}{2}(e^{iz} + e^{-iz}) \\
          &= \frac{1}{2}\left(\sum_{n=0}^{\infty} \frac{1}{n!} (iz)^{n} + \sum_{n=0}^{\infty} \frac{1}{n!} (iz)^{n}\right) \\
          &= \frac{1}{2} \left(2\sum_{n=0}^{\infty} \frac{1}{(2n)!} z^{2n} (-1)^{n}\right) \text{ as odd terms cancel}\\
          &= \sum_{n=0}^{\infty} \frac{1}{(2n)!} z^{2n} (-1)^{n}
\end{align*}
We define sine as:
\begin{align*}
  \sin(z) &= \frac{1}{2}(e^{iz} - e^{-iz}) \\
          &= \frac{1}{2}\left(\sum_{n=0}^{\infty} \frac{1}{n!} (iz)^{n} - \sum_{n=0}^{\infty} \frac{1}{n!} (iz)^{n}\right) \\
          &= \frac{1}{2} \left(2\sum_{n=0}^{\infty} \frac{1}{(2n + 1)!} z^{2n + 1} (-1)^{n}\right) \text{ as even terms cancel}\\
          &= \sum_{n=0}^{\infty} \frac{1}{(2n + 1)!} z^{2n + 1} (-1)^{n}
\end{align*}
Similarly to the exponential, if $z \in \R$ these definitions reduce to the analogous ones for real numbers.

From these definitions it follows immediately that $e^{iz} = \cos z + i\sin z$.
We also get that for $x \in \R$:
\[
  \Re(e^{ix}) = \cos x \text{ and } \Im(e^{ix}) = \sin x
\]
We can now express any $z \in \C$ as:
\[
  z = r(\cos \theta + i\sin\theta) = re^{i\theta}\quad r \geq 0, \theta \in \R
\]
De Moivre's theorem now follows immediately from this:
\[
  z^{n} = r^{n}e^{i n \theta}
\]
The complex exponential also has the following property that will be useful later:
\begin{lemma}
  $e^{z} = 1 \iff z = 2n \pi i, \forall n \in \Z$
  \label{exp1}
\end{lemma}
\begin{proof}
  \begin{proofdirection}{Assume $e^{z} = 1$}
    Write $z = x + iy$ then $e^{z} = e^{x} e^{iy} = e^{x} (\cos y + i \sin y) = 1$.

    Equating real parts gives $e^{x} \sin y = 0 \implies \sin y = 0$.

    Equating imaginary parts gives $e^{x} \cos y = 1$ and we know that the modulus is $e^{x} = 1$ so $\cos y = 1$.

    $\cos y = 1 \text{ and } \sin y = 0 \implies y = 2n\pi$ and $e^{x} = 1 \implies x = 0$ so $z = 2n\pi i$.
  \end{proofdirection}
  \begin{proofdirection}{Assume $z = 2n \pi i$}
    $e^{z} = e^{2 \pi n i} = \cos 2 \pi n + i \sin 2 \pi n = 1$
  \end{proofdirection}
\end{proof}
\subsection{Roots of Unity}
If $z^{N} = 1$ then $z$ is an ``$N$-th root of unity''.
If we let $z = re^{i\theta}$.
Then we have that $r^{N}e^{i \theta N} = 1$.
$r^{N}$ is the modulus so $r^{N} = 1$ and so $r = 1$.

So we require $\cos \theta N + i \sin \theta N = 1$ and so $e^{i \theta N} = 1$ and we know from \cref{exp1} that $\theta N = 2 \pi n$ and so $\theta = \frac{2n \pi}{N}$.

This means that we have $N$ distinct solutions given by $z = e^{\frac{2 n \pi i}{N}}$ for $n = 0, \ldots, N - 1$.
If we let $\omega = e^{\frac{2 \pi i}{n}}$ then we can rewrite this as $z = (e^{\frac{2 \pi i}{N}})^{n} = \omega^{n}$.
We call $\omega^{n}$ the $N$-th roots of unity.
\section{Logarithm and Complex Powers}
\subsection{Complex Logarithm}
\begin{definition}[Complex Logarithm]
  For $z \in \C, z \neq 0$, we define $w = \log z$ by all $w$ such that $e^{w} = z$.
\end{definition}
This means that log is the inverse of exp.
Note that because $e^{z} = e^{z + 2n\pi i}$, $\exp$ is a many-to-one correspondence and so log is \textbf{multi-valued}.
We can write:
\[
  z = re^{i\theta} = e^{\log r} e^{i \theta} = e^{\log r + i\theta}
\]
so
\[
  \log z = \log r + i\theta = \log r + i(\Arg(z) + 2n\pi)\quad\forall \theta \in \arg(z)
\]
\begin{remark}
  If we take $\theta \mapsto \theta + 2n\pi \text{ for } n \in \Z$ then $\log z \mapsto \log z + 2n\pi i$

  To make it unique we restrict it to $-\pi < \theta \leq \pi$ and call this the ``principle logarithm''.
\end{remark}
\begin{example}
  To determine $\log i$ we know that $|i| = 1$ and $i = e^{i\pi/2}$ so $\Arg(i) = \pi/2$ and $\arg(i) = \{\pi/2 + 2n\pi\;\forall n \in \Z\}$. So:
  \[
    \log i = i\left(\frac{\pi}{2} + 2n\pi\right)
  \]
\end{example}
\subsection{Complex Powers}
\begin{definition}[Complex Powers]
  For $z, \alpha \in \C, z\neq0$, we define $z^{\alpha} = e^{\alpha \log z}$.
  \label{cPower}
\end{definition}
This is multi-valued in general.
If we take $\theta \mapsto \theta + 2n\pi$ then $z^{\alpha} \mapsto z^{\alpha}e^{2n\pi i \alpha}$.
\begin{example}
  To determine $(1 + i)^{1/2}$ we can write $1 + i = \sqrt{2}e^{i\pi/4}$ so:
  \[
    \log(1 + i) = \log \sqrt{2} + i(\pi/4 + 2\pi n)
  \]
  Now using \cref{cPower} we get:
  \begin{align*}
    (1 + i)^{1/2} &= \exp\left[\frac{1}{2}\left(\log \sqrt{2} + i(\pi/4 + 2n\pi)\right)\right] \\
                  &= 2^{\frac{1}{4}} e^{i(\pi/8 + n\pi)} \\
                  &= 2^{\frac{1}{4}} e^{i\pi/8} (-1)^{n}
  \end{align*}
  So $(1 + i)^{1/2}$ has two distinct values.
\end{example}
\section{Lines and Circles}
\subsection{Lines}
In the complex plane we can use a point $z_0 \in \C$ and a direction $\omega \in \C$ to describe a line:
\[
  z = z_0 + \lambda \omega \quad \lambda \in \R
\]
We can eliminate $\lambda$ by taking conjugates:
\[
  z^{*} = z^{*}_{0} + \lambda \omega^{*}
\]
Now we can equate two expressions for $\lambda \omega \omega^{*}$:
\[
  \lambda \omega \omega^{*} = w^{*}z - \omega^{*}z_0 = \omega z^{*} - \omega z^{*}_{0}
\]
so for all points on the line we also have:
\[
  \omega^{*}z - \omega z^{*} = \omega^{*} z_0 - \omega z^{*}_{0}
\]
\subsection{Circles}
To construct a circle centered at $c \in \C$ and radius $\rho > 0$:
\[
  z = c + \rho e^{i\theta} \quad \theta \in \R
\]
Which is equivalent to:
\[
  |z - c| = \rho |e^{i\theta}| = \rho
\]
and if we then square both sides and use $|z-c|^2 = (z - c)^{*}(z - c)$ then:
\[
  |z|^2 - cz^{*}  - c^{*} z = \rho^2 - |c|^2
\]
\end{document}
