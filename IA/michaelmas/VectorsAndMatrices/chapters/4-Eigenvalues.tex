\documentclass[../main.tex]{subfiles}
\begin{document}
\chapter{Eigenvalues and Eigenvectors}
\section{Introduction}
\begin{theorem}[Fundemental Theorem of Algebra]
  Let $p(z)$ be a polynomial of degree $m \geq 1$:
  \[
    p(z) = \sum_{j = 0}^{m}  c_jz^{j}
  \]
  where $c_j \in \C$, and $c_m \neq 0$.
  Then $p(z) = 0$ has precisely $m$ roots (not necessarily distinct) counted with multiplicity in $\C$.
\end{theorem}
\begin{definition}[Multiplicity]
  The root $z = w$ has \textit{multiplicity} $k$ if $(z - w)^{k}$ is a factor of $p(z)$ but $(z - w)^{k + 1}$ is not.
\end{definition}
\begin{definition}[Eigenvector]
  Let $T: V \to V$ be a linear map.
  Then $v \in V$ with $\vec{v} \neq 0$ is an \textit{eigenvector} of $T$ if:
  \[
    T(\vec{v}) = \lambda \vec{v}
  \]
  for a scalar $\lambda$ called the \textit{eigenvalue}.
\end{definition}
If $V = \R^{n} \text{ or } \C^{n}$, and $T$ is given in terms of an $n \times n$ matrix $A$, then:
\[
  \vec{A}\vec{v} = \lambda \vec{v} \iff (A - \lambda I)\vec{v} = 0
\]
For a given $\lambda$, this holds for some vector $\vec{v} \neq 0$ if and only if $\det (A - \lambda I) = 0$.
This is because we require the kernel of $A - \lambda I$ to be nontrivial.

The equation obtained from this is called the \textit{characteristic equation}.
\begin{definition}[Characteristic Polynomial]
  For a matrix $A$, the \textit{characteristic polynomial} of degree $n$, $\chi_A$, is defined as:
  \[
    \chi_A(t) = \det (A - \lambda I)
  \]
\end{definition}
\end{document}
