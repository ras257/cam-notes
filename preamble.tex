%%%% Packages
\usepackage{amsmath} % Additional math environments and commands
\usepackage{amssymb} % Lots of extra symbols
\usepackage{fancyhdr} % Header & footer styling
\usepackage{lipsum} % Generates lorem ipsum text for testing
\usepackage{cancel} % Provides \cancel and \cancelto{value}{expr}
\usepackage{subfiles} % Allows for multiple files to be combined
\usepackage{import} % Needed for importing figures from inkscape
\usepackage{pdfpages} % Needed for importing figures from inkscape
\usepackage{transparent} % Needed for importing figures from inkscape
\usepackage[hidelinks]{hyperref} % Allows for customising hyperlink style
\usepackage[nameinlink,noabbrev]{cleveref} % Customising reference labels
\usepackage{pifont} % Used for dingbats font
\usepackage{lmodern} % Latin modern fonts
\usepackage{centernot} % Provides \centernot
\usepackage[british]{babel} % Set language to British English
\usepackage{etoolbox} % Useful tools for conditional expressions
\usepackage{changepage} % Needed for indented environments
\usepackage{enumitem} % Used to customise lists
\usepackage{framed} % Used for creating boxes
\usepackage[skip=5pt, indent=0em]{parskip} % Removes first line indentation
\usepackage[a4paper, top=1in, bottom=1in, left=1.4in, right=1.4in]{geometry} % Page geometry


%%% MISC DEFINITIONS/OPTIONS
\definecolor{shadecolor}{gray}{0.92}
\setlist[enumerate]{label=\textbf{\roman*.}} % Use roman numerals for enumerate

%%% HYPERLINK CUSTOMISATION
%\hypersetup {
%  % Make box underline
%  % pdfborderstyle = {/S/U/W 1},
%
%  % Make border a light blue
%  % linkbordercolor = {0.55 0.67 0.93},
%  
%  % Disable box around link
%  % pdfborder = {0 0 0}
%
%  % Use smallcaps for links
%  % frenchlinks = true,
%}


%%% COUNTERS
\newcounter{theorem}\counterwithin{theorem}{chapter}
\newcounter{corollary}\counterwithin{corollary}{theorem}
\newcounter{example}\counterwithin{example}{chapter}
\newcounter{subexample}\counterwithin{subexample}{example}
\newcounter{subtheorem}\counterwithin{subtheorem}{theorem}
\newcounter{definition}\counterwithin{definition}{chapter}
\newcounter{proofdirection} % Used for automating proof-direction arrows



%%% ENVIRONMENTS
\newenvironment{theoremlike}[2][] {
  \refstepcounter{theorem}
  \par\smallskip
  \noindent
  \textbf{#2 \thetheorem}\ifstrempty{#1}{}{ (#1)}:
  \nopagebreak
  \begin{shaded*}
  \ignorespaces
}{
  \end{shaded*}
}

\crefname{theorem}{theorem}{theorems}
\newenvironment{theorem}[1][] {
  \crefalias{theorem}{theorem}
  \begin{theoremlike}[#1]{Theorem}
}{
  \end{theoremlike}
}

\crefname{proposition}{proposition}{propositions}
\newenvironment{proposition}[1][] {
  \crefalias{theorem}{proposition}
  \begin{theoremlike}[#1]{Proposition}
}{
  \end{theoremlike}
}

\crefname{lemma}{lemma}{lemmas}
\newenvironment{lemma}[1][] {
  \crefalias{theorem}{lemma}
  \begin{theoremlike}[#1]{Lemma}
}{
  \end{theoremlike}
}

\crefname{corollary}{corollary}{corollaries}
\newenvironment{corollary}[1][] {
  \refstepcounter{corollary}
  \par\smallskip
  \textbf{Corollary \thecorollary}\ifstrempty{#1}{}{ (#1)}:
  \nopagebreak
  \begin{shaded*}
  \ignorespaces
}{
  \end{shaded*}
}

\newenvironment{proof}[1][] {
  \setcounter{proofdirection}{0}
  \begin{oframed}
  \textit{Proof}\ifstrempty{#1}{}{ (#1)}.
  \ignorespaces
}{
  \hfill\qed
  \end{oframed}
}

\newenvironment{proofdirection}[1]{
  \stepcounter{proofdirection}
  \par
  % Add arrow depending on if this is the first or second proofdirection
  \textbf{#1}~(\ifnumequal{\value{proofdirection}}{1}{$\Rightarrow$}{$\Leftarrow$})
  \begin{indentenvironment}
  \ignorespaces
}{
  \end{indentenvironment}
  % Add extra line if this is the second direction so that the qed symbol isn't on a blank line
  \ifnumequal{\value{proofdirection}}{1}{}{Shown in both directions, as required.}
}

\crefname{example}{example}{examples}
\newenvironment{example}[1][] {
  \refstepcounter{example}
  \par\smallskip
  \noindent
  \textbf{Example \theexample}\ifstrempty{#1}{}{ (#1)}:
  \begin{indentenvironment}
  \ignorespaces
}{
  \end{indentenvironment}
}

\crefname{defintion}{definition}{definitions}
\newenvironment{definition}[1][] {
  \refstepcounter{definition}
  \par\smallskip
  \noindent
  \textbf{Definition \thedefinition}\ifstrempty{#1}{}{ (#1)}:
  \begin{shaded*}
  \ignorespaces
}{
  \end{shaded*} 
}

\newenvironment{indentenvironment}[1][1em]{
  \begin{adjustwidth}{#1}{}
  \setlength{\parindent}{0pt} % Don't indent at start of new paragraphs for this env
}{
  \end{adjustwidth}
}

\crefname{case}{case}{cases}
\newcounter{casesdepth}
\newcounter{parentcase}
\newcounter{case}

\newenvironment{proofcases}{
  \stepcounter{casesdepth}
  \ifnumequal{\value{casesdepth}}{2}{
    \setcounter{parentcase}{\value{case}}
  }{
    \setcounter{parentcase}{0}
  }
  \setcounter{case}{0}
}{
  \addtocounter{casesdepth}{-1}
  \setcounter{case}{\value{parentcase}}
}

\newenvironment{case}[1] {
  \refstepcounter{case}
  \par\noindent
  \ifnumequal{\value{casesdepth}}{1}{
    \textbf{Case \arabic{case}:} (#1)
  }{
    \textbf{Case \arabic{parentcase}.\roman{case}:}  (#1)
  }
  \begin{indentenvironment}
  \ignorespaces
}{
  \end{indentenvironment}
}

\newenvironment{remark}[1][Remark]{
  \par\smallskip
  \begin{leftbar}
  \textbf{#1}:
  \ignorespaces
}{
  \end{leftbar}
  \par
}



%%% DIAGRAMS AND FIGURES
\newcommand{\incfig}[1]{
  \def\svgwidth{\columnwidth}
  \import{../figures/}{#1.pdf_tex}
}



%%% NEW COMMANDS
\newcommand{\induction}[6]{
  Proceed via induction.\par\noindent
  \textbf{Base case} (#1):\nopagebreak[4]
  \begin{indentenvironment}#2\end{indentenvironment}
  \textbf{Assume} true for #3\ifstrempty{#4}{.}{:}
  \ifstrempty{#4}{\par\noindent\nopagebreak}{\begin{indentenvironment}#4\end{indentenvironment}}
  \textbf{RTP} true for #5:
  \begin{indentenvironment}#6\end{indentenvironment} 
  Thus proven by by mathematical induction.
}

\newcommand{\nonexaminable}{
  \begin{remark}[Note]
    The content of this section is \textbf{non-examinable}.
  \end{remark}
}

% Roman d for integrals
\renewcommand{\d}[1]{\,\mathrm{d} #1}
% Derivative, optional order argument
\newcommand{\deriv}[3][\relax]{
  \ifstrempty{#1}{
    \frac{\mathrm{d} #2}{\mathrm{d} #3}
  }{
    \frac{\mathrm{d}^{#1} #2}{\mathrm{d} #3^{#1}}
  }
}
% Partial Derivative, optional order argument
\newcommand{\pderiv}[3][]{
  \ifstrempty{#1}{
    \frac{\partial #2}{\partial #3}
  }{
    \frac{\partial^{#1} #2}{\partial #3^{#1}}
  }
}

\newcommand{\at}[2]{\left. #1 \right|_{#2}} % Evaluate derivative at
\newcommand{\eval}[3]{\left[#1\right]_{#2}^{#3}} % Evaluate anti-derivative over
\newcommand{\abs}[1]{\left\lvert #1 \right\rvert} % absolute value
\newcommand{\floor}[1]{\left\lfloor #1 \right\rfloor} % absolute value
\newcommand{\ceil}[1]{\left\lceil #1 \right\rceil} % absolute value
\newcommand{\norm}[1]{\left\lVert #1 \right\rVert} % norm
% boldsymbol and mathbf used so that non-latin characters are also displayed in bold
\renewcommand{\vec}[1]{\boldsymbol{\mathbf{#1}}} % vector (bold notation)
\newcommand{\uvec}[1]{\hat{\mathbf{#1}}} % unit vector
\newcommand{\rel}{\mathrel{R}} % generic relational operator
\newcommand{\powerset}[1]{\mathcal{P}(#1)} % power set

% Important Sets
\newcommand{\C}{\mathbb{C}}
\newcommand{\N}{\mathbb{N}}
\newcommand{\R}{\mathbb{R}}
\newcommand{\Q}{\mathbb{Q}}
\newcommand{\Z}{\mathbb{Z}}
\newcommand{\F}{\mathbb{F}}

% Groups
\newcommand{\GL}{\mathrm{GL}}
\newcommand{\SO}{\mathrm{SO}}
\newcommand{\SU}{\mathrm{SU}}

% Use fraktur for imaginary and real parts
\renewcommand{\Im}{\mathfrak{I}}
\renewcommand{\Re}{\mathfrak{R}}

% Extra Operators
\DeclareMathOperator{\area}{area}
\DeclareMathOperator{\Res}{Res}
\DeclareMathOperator{\rank}{rank}
\DeclareMathOperator{\im}{im}
\DeclareMathOperator{\erf}{erf}
\DeclareMathOperator{\arsinh}{arsinh}
\DeclareMathOperator{\arcosh}{arcosh}
\DeclareMathOperator{\artanh}{artanh}
\DeclareMathOperator{\csch}{csch}
\DeclareMathOperator{\sech}{sech}
\DeclareMathOperator{\Isom}{Isom}
\DeclareMathOperator{\hcf}{hcf}
\DeclareMathOperator{\Sym}{Sym}

% Probability
\DeclareMathOperator{\poisson}{Po}
\DeclareMathOperator{\binomial}{B}
\DeclareMathOperator{\var}{Var}
\newcommand{\E}{\mathrm{E}}
\renewcommand{\P}{\mathrm{P}}

% Extra Symbols
\newcommand{\qed}{$\square$} % Symbol to end proof
\renewcommand{\emptyset}{\varnothing} % I prefer the look of \varnothing
\newcommand{\tick}{\ding{51}} % Handwritten tick symbol
\newcommand{\cross}{\ding{55}} % Handwritten cross symbol
\newcommand{\trans}{{\mkern-1.5mu\mathsf{T}}} % Matrix transpose

